\documentclass[12pt]{article}

\usepackage[margin=1in]{geometry}
\setlength{\parindent}{1cm}
\linespread{1.3}
\usepackage{indentfirst}
\title{College Memoir}
\author{Patrick Steadman}
\begin{document}
\maketitle


\section{The Bell Curve}

My mother wanted me to bring more stuff to school, like a down comforter,
washclothes, and posters.  I tried to explain that I didn't want to bring these
things and we argued.  I hadn't slept much, and couldn't communicate the
aesthetic I was imagining for my college life.  I had set my Cornell netID
password as "10Nianhan", a Chinese idiom that translated to "Ten Years by the
Cold Window", signifying a period of transformative, ascetic study.  My mother
didn't seem to get that.

I put on a red polo shirt from American Eagle, and my parents drove me to
Ithaca.  It was a beautiful, clear, cool day.  

Moving in, I saw a pretty Indian girl with long hair.  I felt a stab of anxiety,
wondering if I would realize my dreams over the next four years.  It was time.
I couldn't let things stay the same.

After registration, Mommy took my sisters to Collegetown.  I went to the dining
hall with my dad.  It was obvious that he was going to lecture me. 

I knew that my dad had struggled at Cornell, to some degree.  He had taken a few
semesters off to work as the chef for his fraternity.  Then at some point he met
my mom, on the sailing team.  

After we got chocolate milk from the steel udders my father took a napkin and
drew a bell curve on it.

``You know what this is, right?'' he asked.

``A bell curve?'' I said, bemused.

``Yeah!  The whole drive here, I was thinking about what to say to you, now.
It's tricky,'' Daddy said, seeming a
little embarassed.

``College is pretty long and exhausting,'' he said.

``What does a bell curve have to do with it?''

``Well, I guess you're used to being around \textit{here} on the bell curve,
right?'' said Daddy, indicating somewhere in the top 10\%.  ``Or at least that's
where I was used to being, I think.''  

``Yeah, I guess,'' I said.  I didn't like where this lecture was going.

``Well, at Cornell, you might have to get used to being around here on the
curve, or even here.  Everyone at Cornell comes from the top end of the curve.
You have to learn to be okay with being down here on the curve sometimes.''

This is cliched stuff, I thought.  I felt a little angry and stressed.  I'm not
like you, Daddy, I thought.  You don't understand.  I'm not even really
\textit{on} the bell curve, in some way.  I don't care about the bell curve.
I'm sorry you did.

``Sure,'' I said.  ``I get it.  I've heard this stuff before.''

``As long as you go to class, you should be okay.  You have to go to class.  You
just have to.''

``Look, remember, I calculated the cost per hour of college during that boring
orientation thing earlier?  \$7.70 an hour, even when I'm asleep.  I'm sure the
cost of class hours is, like, insane.  I'm going to make the best use of my time
and go to class, don't worry, okay?''

---

I went through all of the orientation stuff.  I had no interest in spending time
with the group of Communications majors I was assigned to, so I drifted between
groups. 

At some point, after my parents left and the activities were finished, we were
set free.  I suddenly realized that no one was expecting me home that
night, and felt a video game like sense of freedom.

---

I found Finn, a friend from my high school, and started following him to
Collegetown.  A lot of other people were following Finn too.  It was apparent to
me that Finn didn't really know where he was going, but he managed to chat with
a girl who was going to a party and just sort of co-opted that information.  

Finn used me as a way to create interesting conversation for the group.
``Steadman's crazy.  Tell them about the time you hitchhiked to Montana.  I
drove this fucking kid to the highway, and he just started hitchhiking there.''

``Wait whhat?'' someone would say.  I would just sort of nod and smile,
secretly pleased to get some attention, but also a little uneasy.  Then I would
answer typical questions about running away from home and hitchhiking.
Collegetown was full of students, many on the lawns of houses.

The party was at a Jewish fraternity's annex.  It was crowded.  Finn and I
pushed our way in.  Finn got beers for the remaining people in our group.  We
stood by the beer pong game.  Finn was grinning, and occassionaly we would have
a shouting conversation with a girl, who always turned out to be other
freshmen.  I would feel optimistic about each girl we talked to.   ``These are
going to be a great four years, man,'' said Finn.  ``We're going to kill it
here.''

***

I sat in the entrance to Goldwin-Smith hall, with my copy of Ulysses, my
notebook, and the book of annotations that helped explain Ulysses.  The sun was
setting over of the arts quad and the stage set up for the end-of-orientation
week DJ. I continued reading Ulysses, but I enjoyed the presence of the other
students and the music playing over the PA system.

Every line of Ulysses seemed to contain references to information that would
enable to become the sort of human being I wanted to be.  ???

I decided to move onto the quad.  On my way, I was stopped by two girls.  It was
immediately obvious that something was up.

``Hey, can you take us to a bathroom,'' said a short tan Asian girl, probably
Korean, who was slightly overweight.

I opened my mouth but for a second nothing came out; I had been in the library
all day.

``Sure, yeah,'' I said, making myself smile.  The other girl's hair was dyed in
a white streak down the center, making her look like a skunk, but I thought  she
was cute.

I led the pair through the crowd.  The Asian girl walked next to me, bumping
her hips into mine. ``So you guys are freshmen?'' I said.  

``Noo,'' said the Asian girl.  ``We go to Ithaca college.  We're sophomores.''

``Oh, cool,`` I said.   I took them into the art library, and we found a
bathroom in the basement.  They went in, and I could clearly hear their
conversation.

``Oh my god you always go for the nerdy ones,'' said skunk girl.

``He's cutee though,'' said the Asian. 

I went back outside with them, and we worked our way into the crowd.  The DJs
were on the stage.  They had something to do with Mario and video game music.
The group that did ``Like a G6'' played next.  The skunk girl had a plastic
water bottle of vodka, half full.  She shared it with the Asian girl but not
with me.

The Asian girl stood in front of me and then started grinding her ass into me.
I thought back to stuff I'd read online about how the best degree at Ithaca
College was a Mrs. from Cornell.  It felt fucked up.  She yelled things into my
ear but I didn't really understand anything she said.  ``Do you want to hook up
with me?'' she yelled.  

cold air that sometimes come in a hot crowd like this

I thought for a second.

``Uh not really,'' I said.  ``What?'' she yelled.  ``No,'' I said louder.

``Then why are you so \textit{hard}," she said into my ear, running her hand
over my dick.  I felt like I was drunk even though I hadn't had anything to
drink.  I looked at the skunk girl, dancing with her eyes closed.

"Like a G6" came on.  I danced with the Asian girl.  She was quiet.

Eventually I got to the point when I knew nothing would be gained by staying
with the two girls.  I said goodbye, laughing, and walked back to my dorm,
feeling bemused and happy. 

Over the next five years I often thought back on this event and wondered why I
didn't have sex with the Asian girl.  

---

At some point during the first few weeks of classes, I met George.  Paola
introduced me to him.  Paola and I had been in the same Buffalo-wide "Gifted
Math Class", and had been my only girlfriend in high school.  When we started
dating she told me that I should meet her friend George, but we broke up before
I had a chance to meet him. 

One evening, I went over to her suite and uneasily chatted with her and her
roommates until George showed up.

Despite how reserved George was, and how in many ways he seemed to be a
``typical Asian'', I quickly felt that our fates were aligned and we became
friends.

---

I was disappointed with the people in my dorm.  I tried to hang around in the
common room and talk to them, but most of the guys were in the Engineering
school.  They were interested in solving Rubik's cubes or talking about
Pokemon.  Some of the other guys were interested in athletics and participated
in the Judo club or went to the gym.  One time I tried doing an "ab ripper X"
video with these guys but couldn't get through it.  It seemed the the girls
were not very interested in me.

I would message George and we would meet on the balcony outside the common room.
He would usually listen to me play guitar, and then we would talk about music
or listen to it.  I wasn't very good at guitar or singing.  I could only learn
one or two songs at a time, and had a bad sense of rhythm.  George had studied
piano as a kid, but when I tried to get him to play at the piano on the top
floor of the dorm he would silently attempt to play a piece of classical music
for about fifteen minutes, and then suddenly give up.  

---

I took it upon myself to re-read the Odyssey before I started reading Ulysses.
It seemed like my only chance at understanding and originality was via
rigorously exploring of all of the dependencies of important pieces of
literature.  I was unsettled by my experience with the required reading book for
incoming freshman: ``Do Androids Dream of Electric Sheep?'' There was an essay
contest and a reading group.  I had put many hours and hours into my essay, and
had to run through the rain to turn it in, late to the group discussion.  But
listening to my peers talk about the book, which I loved, I felt confused and
defeated.  In the back of my head, I knew that the essay wasn't very good.

The night before the first day of classes, I posted an excerpt from the Odyssey
on Facebook: ``I will leap out of bed, sling a sharp sword over my shoulder,
strap a stout pair of boots onto my lissom feet, and go forth from my chamber
like a young god to face 8am calc.''

The Ulysses class met in one of the classic seminar classrooms in
Goldwin-Smith.  These classrooms had windows the size of a dining room table.
I was the only freshman in the class.  Most of the other students were
attractive: there were upperclassmen girls with golden jewlery, beautiful
sweaters, and slim bodies.  There was a handsome Irish student and a
crazy-looking woman with a hunchback. 

I considered joining an A Cappela group or the sailing team.

I was also in Advanced-Intermediate Chinese, a class on Globalization, and a
media Communications Class.  I was a Communications major.  I had fallen apart
writing my application essays alone on a foreign exchange in Taiwan.  I chose
to apply for the Communications major because I imagined running a
multinational media company.  Cornell was the only school I got into, besides my 
safety school.

--- 

I liked to talk to George about my schemes for college.  It seemed like George
had a quiet interest in my more weird or risky ideas.  For example, in high
school the Internet had inspired me to go urban exploring.  I used urban
exploring as a way to get people to hang out with me outside of school.  We
explored the basement of our high school, and then abandoned factories,
theaters, and hospitals.  I dreamed of exploring Cornell's tunnels and rooftops
until I found a place to build a secret room where people could meet.  I
imagined it would be a grimy version of the ``secret'' rooms that exclusive
societies like Quill and Dagger had. 

George helped me look for a suitable place to build this room.  We went out in
the heady warm nights of mid September, when the emerald green quads were empty
and bright.  George watched my back while I opened hatches that looked
promising.  I was impressed by his bravery and calm.  Unfortunately, we didn't
have much luck finding a good place.  I promised him that we'd eventually find a
good place, that it was like hitchhiking: the ride always came when you least
expected it.

---

I ate lunch with Seth, a ``nerdy'' friend from high school, who was planning to
be a doctor.  I was surprised when I found Seth and he was sitting with a bunch
of good-looking, articulate people. They had met during orientation, playing
ultimate frisbee.  I was instantly attracted to Morgan, who was tall, brunette,
and classically good looking in a somewhat masculine way.  She was from
Virginia.  After that, I made a point to eat lunch at the same time and place,
so I could join their table.  We sat in the back of Okensheild's, the old
faux-gothic dining hall overlooking the slope.  Morgan showed me Nicki Manaj,
saying she was the best female rapper since Lil' Kim, and would happily discuss
all of the stuff I was researching as part of reading Ulysses, like Gematria and
manichaeism.

---

\textit{need more Harry Potter like descriptions of the physical environment}

I got into the habit of doing homework in the laundry room of my dorm.  I liked
the big plastic table, the white noise, and the warmth.  I did a lot of calculus
homework there, or did readings for my classes classes.  I also read most of Tao
Lin's ``Richard Yates'' there.  

``Richard Yates'' had recently been published.  Conversations with George had
inspired me to actually read it.  George and I wanted to apply to Telluride
House, a ``intellectual community'' that had provided free housing to
neo-conservative thinkers like Allan Bloom, Francis Fukuyama, and Paul
Wolfowitz, back in the early 1960's.  It was now supposed to be much more
progressive.  George and I were already becoming reliant on each other for
discussion of the ideas most dear to us, like literature on the internet, ``the
classics'', and Hipster Runoff. 

The Telluride House essay required five essays.  George and I stayed in the
graduate English lounge late on the days leading up to the deadline.  I wasn't
making very much progress on my essays.  I wrote one that described the sounds
of the beginning of a class in middle or elementary school: the unzipping of
binders, etc.  I showed it to George and he sort of smiled and asked if it was
supposed to be like James Joyce.  I felt discouraged. 

George didn't know what to write about.  I was procrastinating by looking at
stuff on Hipster Runoff and asked if George had heard of Tao Lin.  ``Of
course,'' said George, surprising me.  ``Maybe you should write your essay about
Tao Lin,'' I said.  ``But I haven't read any of his books,'' said George.  ``So?
That's perfect.''

While George worked on his essays, I read everything that I could find online
about Tao Lin, or written by Tao Lin.  I had become aware of  Tao Lin when I was
living in Taipei as a foreign exchange student.  During winter break I was
living with a nervous old woman in the misty, hilly northern part of the city.
I was very isolated, and my mental health rapidly deteriorated as I tried to
write my college application essays before New Years, when they were due.  It
would be the start of a new decade, hopefully my decade.

I became obsessed with Hipster Runoff, feeling that it somehow cut closer to the
heart of my life than anything else.  I first felt contempt for the excerpts of
"Shoplifting from American Apparel" that I read, but I realized that some of his
writing accurately captured some of the strange feelings I was having in Taipei,
feelings that I had thought were unique.

I told that George that my essays were ``almost finished'', not wanting to
discourage him.  At midnight, I sent in an application that consisted only of
blank essays.  I told myself that I would finish them all the next day and
pretend that I had forgetten to attach them.  George seemed exhausted, and not
particularly happy with his essays.  Jokingly, I ordered a copy of ``Richard
Yates'' from Amazon.

---

I took my first Calculus prelim.  I was surprised at how hard it was.  I mostly
understood the material, I thought, but I was having more and more trouble
paying attention in lecture.  I got a score in the 70s, which was the mean for
the test.  I felt moderately angry, and promised myself that I'd study harder
next time.

Fall break came.  I decided to stay in Ithaca.  While jogging around Beebee
lake, Mommy called me.  I had a silver Motorola RAZR flip phone, my dad's old
phone.  I told her things were going pretty well, but that writing essays were
hard, and I hadn't applied to Telluride House.  ``Next year,'' she said.  I
walked to the one remaining dining hall, and ate while I read.  I still felt
optimistic.

---

One time I was working in the laundry room and Paola texted me, wondering what I
was doing.  I invited her over.  I wondered if she still felt attracted to me.
I high school relationship had collapsed simply due to awkwardness.  Maybe
things had changed.  

It was one of the first dark, cold nights.  Paola seemed happy to see me.  We
talked about our school work.  It seemed like the first month of college had
changed Paola.  I felt like I was still the same.  One of the things that had
made we want to break up with Paola was the fact that she was almost constantly
cheerful and bubbly, but her words now seemed to be heavier, as she spoke about
how difficult her statistical methods class was.  She was considering changing
her major, or studying abroad.  I said that calculus was hard too. 

``I'm really tired,'' she said, and put her head down on the folding table.

``This is a tricky problem set,'' I said, looking down at the book.  We were
silent in a way that reminded me of silences that sometime happen when drinking
and talking with friends. Then I realized that Paola was making crying sounds.
For a second I thought it might a noise from one of the laundry machines, but I
listened harder until I felt sure that it was from Paola.  I wondered if she
was going to tell me what was going on, or if she knew that I could hear her.
I felt paralyzed and disturbed.  

I imagined patting her back, hugging her, or saying something, but I couldn't
actually make myself do these things, because of some sort of strange
embarassment.  I felt ridiculous and weak, pretending that I was oblivious.

I hear her lift her head from her arms, and straighten herself.

``I think I'm going to go back to my dorm,'' she said in flat voice.  I looked
over in her direction, not meeting her eyes.   

``Oh yeah, good luck.'' I said.

``Good luck,'' she said on her way out the door.

---

I would go out with either people from my floor, or
Bebe Zeva

Increasing moments of desperation, denser and denser notes
color-coded and with two rows of words per line, allow for greater density of
notes
reading list became more and more ambitious
Halloween
craigslist and lockpick
---

I had an Oral exam in Chinese.  I had been up until 4am reading about Aristotle
and my face felt stiff.  I smiled and apologized when the teacher asked why I
was missing class.  We began going through the pre-defined dialogue, but I found
myself having more and more trouble.  I felt far away, dissociated.  We were
talking about paper cranes.  I was forgetting words, and felt transfixed by the
desire to excuse myself and leave.  The teacher gave my a pitying smile.  

Finally Morgan and I were studying together, like I studied with George.  I
fantaized about her joining George and I, and three three of us becoming a
dependable,  growing intellectual group.  Morgan and I studied on the top floor
of Uris library, where the sound of the belltower was noticably louder, and the
sound of the wind often made me sad.  I was starting to do research for my
midterm essay in the Ulysses class.  

---
going home
wedding

Halloween ``faggot'' 

I stayed up most of the night trying to write the Ulysses paper.   Around 3am, I
subconsciously gave up, but spent time reading on the internet for a few more
hours until going to sleep in disgust.  I slept through my calculus class but
made myself go to the Ulysses class, wearing a green polo shirt.  I smiled at
myself in the mirror.  I had only a few paragraphs written.  It was a Monday,
and I figured I'd be able to turn it in before the next class, and just take
some late points.  I didn't talk to the professor, and slipped out while
eveyrone else was turning in their essays.  

I felt strange the rest of the day.  I talked brightly with Morgan and David at
lunch, and worked on a crossword with them.  Everything that happened felt
critical.  I had two full days to finish the essay.  I would research Giordano
Bruno and turn in a paper that made original contributions to the study of
Ulysses. The quality I could add in the next two days would more than make up
for the lateness.  

when Morgan and mother and fate ---

Morgan, Seth, David and I always talked about going out, but it never really
happened.  Sometimes Morgan and David would talk jokingly about things that
happened on the weekend, but those nights out seemed to be with their
floormates.  We didn't actually all go out together until a warm night in the
beginning of November.  

I had high hopes for the night.  I figured something would happen between Morgan
and I.  I had no idea how to negotiate the physical space between her and I, and
saying something to her seemed impossible in her presence.  I figured alcohol
would make things happen.

We ended up going to a few fraternities around the new West Campus dorms.  I
watched Morgan play beer pong with Seth.  I had a few shots.  Morgan, Seth,
David and I danced.  Seth was being goofy, and Morgan laughed.  

I was starting to feel more tired than drunk.  A few times, Morgan and I talked
about our schoolwork, but when Morgan was drunk, talking to her about Ulysses
and her child development stuff felt hollow.  The converstations didn't last
long.

I was falling behind the group, which had swelled to include some other loud
guys, as we headed to an afterparty in Collegetown.  I had no connections at
these places so I was just along for the ride.  I realized suddenly that it was
November, and I didn't have the social connections that other people were
developing, despite my best efforts.  I stalked behind Morgan, listening to her
talk to a group of loud drunk guys.  I felt more and more drunk and unhappy, and
felt angry and disappointed with Morgan and also myself.

I dropped away behind a wall, and then ran away, and stopped again, feeling
ashamed.   I was inside the arches of the World War II memorial.  I waited,
breathing hard.  Suddenly Morgan appeared.  

``What are you doing, Patrick?'' she said, seeming annoyed, but a little amused.
I stared at her in mute shame, knowing that I didn't have words for what I was
feeling.  ``I think you should go home,'' she said.  ``Can you get home?'' she
said, in an overly concerned voice.

``Yeah,'' I said.  I set off on the direction of our dorms, but then turned
towards the suspension bridge.  I felt a sort of thrill at my unhappiness and
disappointment, watching my feet move quickly in the orange streetlight.  It
felt like something was broken and I was being pushed on.  I walked over the
bridge, thinking about climbing over the fence.  I told myself I was fine and
that I was free of Morgan now, and that tomorrow I'd write.

---

I stared at the Calculus prelim.  I didn't know how to do anything.  I had
forced myself to go to it, which was hard enough.  But still, I didn't know how
to do it.  Tears came to my eyes.  I looked through the test booklet for about
a half hour, attempting some of the problems on scrap paper.  Then I turned in
the empty test.  It was a beautiful night, and I walked home with other
testtakers, but feeling removed from them.  I felt that I might really be in
trouble now, grade wise.  I dreaded the thought of my parents.

---

Meeting mother for Mexican food

I went to a strange place in November.  Instead of going to get my hair cut, I
ordered hair clippers from Amazon.  I cut my own hair in the dorm bathroom, the
hot metal of the clippers burning the nape of my neck.  

A few days later I woke up in the middle of the night with a intense desire to
shave my head, or half of it.  I lay awake in bed, sweating.

I also ordered a set of lockpicks from a PayPal shop, and showed the excitedly
to George.  I thought that the lockpicks would reignite our stalled ``secret
room'' plan.  I practiced using the lockpicks on my own dorm room door.  When I
finally was able to lock and unlock the door, I excitedly showed my new skill to
the floormates. 

I was spending a lot more time with them.  It seemed like they enjoyed hanging
out with me now, amused by the things I was interested in.  I would stay up many
nights with the floormates, talking.  I showed a prefrosh the massive collection of
art pngs I had torrented.  My computer got viruses, and was booted from the
network, so I had to go to the computer lab, where I obsessively read articles
about Tao Lin, or talked to a 23 year old I had met on craigslist.  

``You know, in the beginning of the semester, I thought you were a scrub,'' said
Jonny, one of the athletic guys on the floor.  ``But you're like the opposite of
a scrub,'' he said.  

``What's a scrub?''  I asked, intuiting the meaning of the word instantly.

Jonny seemed embarassed.  ``You know, \textit{that} guy.  Because you like
didn't talk to any of us.''  

A bunch of us were sitting around, making ramen.  One of the Asian students was
cracking eggs into the boiling water.  It was a cold morning.  I had stayed up
the whole night before, reading Umberto Eco's ``Foucault's Pendulum''.  A
significant amount of snow had fallen in the dark.  I felt a sense of stillness
and doom, knowing that Thanksgiving break was next week.

I was spending most nights in the library, reading ebooks.  I decided that I
needed to finish reading all of Ulysses before I could write my essay.  I felt a
sense of awe at the last chapter of Ulysses, when I read it.  It seemed like the
perfect consummation in literature, the imaginations of two perfectly real
characters treading the same memories, affirming love. 

I read ``The Secret History'' and ``100 Years of Solitude''.  I felt a leap of
hope the night I stayed up all night reading about Aristotle and the hope he had
provided for Joyce, and then ended up sitting in the dorm shower and crying.

Email from professor

I bought Bulfinch's mythology, and read it, waiting for fall break.


---

My mother drove Paola, Finn and I back to Buffalo.  She was in a good mood and
took us to eat at Moosewood.  After, we walked around the Brentwood Mall, which
isn't really a mall, it's the basement of an old institutional building.  It
smelled like fall in Ithaca, a hazy smell of upstate New York and water in
basements. 

I bought a beautiful embossed version of "The Aeneid" at the rare and used book
store. 

In the car, I could feel the stress of the past months affecting how I talked
to Mommy. I  wanted her to stop talking to Finn about his physics major.  I
felt that he'd definitely flake out of it soon.  The way Finn talked about his
physics major, it seemed very obvious that he wanted to impress people with it. 

My plan would work out better in the long run.

As we drove through the amber fields overlooking Cayuga Lake Mommy stopped
talking to Finn and Paola and quietly asked me how things were going.

"I think I really hate Cornell," I said.

"Your classes still aren't going well, huh?" Mommy said. 

"They're going badly but it's not that.  It's the people.  I'm really bored and
so...disappointed."

I could tell what I was saying was hurting her.  But I was so sick of her
upbeatness.

"Patrick, where else would you go?  I think you're being unfair.  It'll get
better.  I remember how stressful it was.  I still have bad dreams sometimes
where I have a prelim and a bunch of homework I haven't turned in."

"You don't understand.  I don't know if I can go back there."

I said the communications classes were such bullshit and that I wished I was
studying something in the Arts and Sciences College.  

"You can't blame us for your problems," Mommy hissed at me, even though Finn
and Paola were probably asleep in the back.

We drove by an abandoned house that I noticed every time we made the drive from
Buffalo to Ithaca, a dark farmhouse close to the road.  I wanted to
explore it someday.

When we got home I put the Aeneid on my book stand.  The Aeneid was good and
important to read.  Of arms and men.  I could hear Mommy and Daddy arguing
downstairs as they made dinner.   The golden light of childhood autumns filled
my room.

I was called down for dinner a few times until I heard anger.  I sat down at the
table feeling shaky and tired, and drank the milk.  It felt chalky in my mouth.
Avery made a face when she saw my hair.

We started arguing around the end of dinnner, and my sisters left.  We ended up
in our "dining room".

"You have to figure out what you can do to recover your classes, Patrick.  Talk
to your professors," said Daddy.

"I can't, I can't." 

"What do you mean 'you can't'?  You don't really have a choice.  College is an
endurance thing, you just have to get through this semester."

"You can't take this for granted, Patrick.  Grandma and Grandpa have been saving
for a long time so you can go to Cornell and not the University at Buffalo.  We
thought you'd be happy, once you got to a place like Cornell, with challenging
classes and people", said Mommy, her voice self-pitying.

"Can we please not talk about money?  It doesn't help."

I paced around the dining room table.

"You guys don't understand.  I think I'm going to fail all of my classes, or at
least all but one or two," I said, finally.

"What?"

"Great," said Mommy.

"There are things you can do, Patrick," said Daddy after a few seconds.

"I've already tried them.  It's too late.  I just couldn't write my essays.  I'm
sorry."

"Then forget you essays!  Focus on the other classes.  Have you been going to
them?"

I felt the pain coming.

"Not since, October, maybe?  I was really focused on trying to write that essay
and was at the library all night.  I'm sorry.  It's a writing problem.  I think
I'm getting closer to figuring it out."
 
"Your essays aren't the issue here, Patrick.  You need to drop that class and
figure out how to pass the other classes."

"I'm going to fail them all, except for maybe one.  There's no chance.  I'm
almost sure."

I could see tears coming to the edges of Mommy's eyes.

"Oh great," she said.  "You really screwed up."

"Yeah, basically."

"You need to talk to your advisor," said Daddy.  "And figure out how to drop the
courses."

"It's past the drop deadline.  I'm just going to fail those classes."

"No, that's not acceptable."

"You're going to need to get a job," said Mommy.

"What?  No, I'm going back to try again."

"No, you're not," said Daddy.  "We can't let you go back like this."

"I can't \textit{not} go back, okay?"

"I know..." said Daddy, "that you're afraid that if you leave, you won't ever go
back.  That makes sense."

"I don't even want to go back, I'm so miserable there.  I need to figure out
this essay thing, though."

"It's not essays I think, Patrick.  You need to understand that you won't always be the
best in your class."

"I know that, stop saying that," I said.  "That's not my problem."

"I think you're not turning your essays in because you're afraid of not being
the best," said Mommy.

"I can't even really get them started.  I don't know why.  I've been trying
\textit{so hard}, I hope you understand that."

"You're going to have to pay for this semester you wasted," Mommy said.

"Please stop talking about money!  It's not helping...me write my essays.
Cornell is so freaking stressful."

"Tomorrow we need to call your advisor and figure out how to drop those
classes."

"First, I have never talked to my advisor, and second, it's way past the end of
the drop period."

"Why didn't you talk to your advisor?  That's just stupid, Patrick," said Mommy.

"Because he's in Communications, and I don't want to be a Communications major!
You have no idea how dumb my Communications class is.  It's like the opposite of
Legally Blonde.  I'm the only person with a black Windows computer surrounded by
a bunch of sorority girls and bros with MacBooks."

They laughed.

"Look, I know that might not be the best match for you, but that doesn't mean
you don't need to talk to your advisor or go to class.  You're being stubborn
and can't blame us," said Mommy.

"We can call the CALs office tomorrow and figure out what to do."

"He should call up La Scala's and see if he can get his job back," said Mommy to
Daddy, faux-seriously.

I scoffed, and made my way up to my room.  I could hear my parents arguing
downstairs.

***

The next day my parents called the College of Argiculture and Life Sciences.  I
listened to them talk from the head of the stairs.  I wasn't sure why they
didn't make me make the call, but I was vaguely grateful.  I was so sick of it.
I had shivered in bed last night, watching Mad Men.  My sister had moved into my
childhood bedroom and so all my stuff was crammed into a small room with large
windows that let in the cold.  

\section{Seroquel}
My family goes to Connecticut for Christmas, where my grandparents live.  My
mother and my grandmother (a middle school pincipal) made the decision that I
would visit a special psychologist in Connecticut.

I could feel that I was making my family miserable at my grandparent's house.
At home in Buffalo, I felt okay with my situation being on leave of absence.
After an initial reluctance to give up on the semester, I felt an intense sense
of relief once I realized I wasn't going back to school.  I lay in my bed with my
phone, listening to classical music.  It was truly winter, after Thanksgiving,
and I had a wired heating pad.  During Christmas I wanted to continue this
lifestyle.  I had made a list of 100 books I wanted to read in the new year,
and I wanted to spend eight hours a day reading.  I had my bookstand and a copy
of Harold Bloom's ``The Western Canon'', and I tried to set it up in the dining
room away from everyone.  My mom came in and told me sternly that this wasn't
okay.  I argued.  This happend for days.  I was making my family miserable.  The
thought of playing with my toddler cousins or other high-spirited activities was
painful.  My mother and even father accused me of being unempathetic and
selfish.  Very soon I was miserable as well.

We went to visit the psychologist on a workday between Christmas and New
Years.  She worked out of her own home. 

My parents went in and talked to her for at least an hour.  I had my notebook
with me, and drew the door to her office.   ...

When I returned to Buffalo, I began to row 5 kilometers most days.  The first
time I tried to row five kilometers, I stopped at about three kilometers in, but
then started up again and finished.  I wrote the total time in my Moleskeine
planner.  I also used the Moleskeine to write down the PHP/JavaScript/HTML
tutorials I completed and the books I planned to read.  This was basically how I
passed the winter of 2010-2011.  I had to withdraw from the literature classes I
took at 

Bookstore job.  

\section{Learning to Program}
Hipster Runoff personals.

\section{Robert Horning}

After landing, I got my bag and waited for Marie to arrive.  I finally found her
waiting outside.  When she saw me she made an inarticulate noise and hugged my
chest.

"Hi," I said.

I shivered from being on the plane.  We waited in a long line for a taxi to
Manhattan.  We were mostly silent, but it was not awkward. 

The movement and wind of the taxi made things feel less ominous.  I put my arm
around her and and enjoyed looking out the window at the orange, complicated
night outside.  It was my first time in New York without my parents.  In the
tunnel under the river, Marie kissed me.  It was not a great kiss, but it put
to rest the mild anxiety that things weren't going well.

When we got to the hotel I paid the \$50 for the cab, and we made our way to the
room.  It was a little more dingy than I had expected.  I could tell that we
both liked it, though.

We had planned to drink 2.5 beers before we had sex.  As we sat on the bed,
talking about the guy she had met on her flight, I felt more and more
comfortable with her voice.  The light of the room was very warm and home-like.
When room service came with our beer and food Marie smiled at me as I fumbled with the
tip.  

I drank one of the Budweisers quickly and watched Marie eat steak and eggs.  

"Do you want to have sex now?" Marie asked, placidly.

"Yes," I said, smiling.

"Let's take a shower first."

Marie went into the very small bathroom and took off her wrinkled button
down and shorts.  She dropped them on the floor, where they became damp with the
water from the tub.  I took off my clothes, shivering again.  We sat
facing each other in a few inches of warm water.

I touched her shoulders, which had been scarred by acne.  She ran her fingers
over the hair on my upper thighs.

"I want to give you a blowjob," she said.

"Alright."

"Could you, like, kneel?"

It was my first time getting a blow job, and I was surprised by the feel of
teeth and stray hair.  The fact that we were in a tub was somewhat funny, to me.

"You have really big balls," she said, when she stopped.

We got out of the tub and went to the bed.  I felt more confident.  We kissed
and then had sex.  As we had discussed before, I didn't use a condom.  I came
pretty quickly and Marie checked that I hadn't come inside of her.  

"Don't worry," I said. 

After a period of quiet conversation, Marie asked me to get her a pack of
cigarettes.  I was happy to take a walk alone, to process things in cool night
air.

I saw a smoke shop but walked past it, drawn towards Times Square.  It was late
enough that Times Square was relatively quiet, but still incredibly bright.  I
memorized the shapes of the black and reflecting buildings above me, most
seemingly from the late 80s and reminiscent of late 80s neoliberalism.  I felt
that I had lost my virginity the right way, for me.  

  
\section{Mainland China}

Sometimes after work I lay on my bed and grin uncontrollably, thinking about
the money I was making.

It was dangerous to lay on my bed and think about myself.

I felt good in the morning, usually. Before I went to sleep, I opened the
blinds. My room had a very clean wooden floor that�s warmed by the morning sun.

There is a certain part of me that always wanted discipline and order,
wanted to be organized and strong. I wanted a situation that would force me
to live properly, and I had failed to find it.

I�m better at bullshit, I realized. I had lost my ability to bullshit during my
first semester at Cornell, and I had been living in constant retreat since.

Breakfast was sort of like a meeting. I reported the information, prices, etc
that I got from America the night before, and offered my suggestions. Often we
discussed things for an hour or so, drinking coffee. Outside the window, the
workers were riding into the factory on scooters and bikes, checking in with the
security guard.

Sometimes around midday I thought silly things like "I�m going to the bathroom
and there I�ll think really hard, and after that I�ll just try to live the best
life I can and not worry about existential things."

There was a Japanese "technical advisor" living with us. He was in his sixties.
I appreciated his taste in aesthetic experiences.  After dark or early in the
morning, we would go on long cycling tours of the surrounding area. He owned a
pair of excellent racing bikes.  There wasN nighttime traffic, sprawling new
apartment housing developments, rice paddy canals, etc.  Asiatic cows.   It was
nice to appreciate these things with someone else.  One night there was the
moon.  He didnt�t speak English, and his Chinese wasn't very good.

I imagined buying an extremely clean apartment and furnish it simply and hiring
someone to clean and cook.

I felt calmer.  I had work and money, but not self-control.  I didn't live
regularly.  My social skills were somewhat improved, but only in a
Chinese/Japanese context.

"hmm what did we talk about back when we talked"

The day�s work was done at 5:30, and we went upstairs for dinner.

Hirata said: "I like rich people. I don�t like poor people."

~

Elma and I went to the customer�s factory. All the other
engineers underestimated Elma because she is a woman and ugly. She made \$270 a
month. We walked into the room where they were sewing. There were about three
hundred women operating sewing machines. Elma said haoduomeinv.  (look at all the
beauties)  They were all mother-aged. It was loud and they were playing Teresa
Teng over the loudspeakers. It smelled of sweat and fabric. I felt lightheaded
but the music made me happy and I felt invigorated or something, like I had
just washed my face.

easy to romanticize with the largeness of the machinery and the weightiness of
the product. The area we live in smelled like freshly-machined steel.

Elma did a good job getting the marker files we needed and now I think we are
friends.

I explained the concept of wage slavery to Joy as we walked.

"like how he talks about wage labor, like earning the money and then spending
it on food (lattes and pizza) needed to repair/replenish my body for another
day zzz this is like everyone of course, but that thought/concept keeps running
through my head."

So that day I was in a bad mood when we returned to our factory. But then
Japanese boss called me into his office and gave me the first envelope with
"unoffical lifestyle fee". I ran upstairs and counted the brand new bills on my
bed.  Thousands of dollars.  Then we all went out for Japanese food at the
Westin hotel downtown.

I was very happy.

\section{The University at Buffalo}
While I was very involved with the Internet in the spring of 2012, I also
started to have more a ``real life''.  I saw a flyer advertising the men's
rowing team, looking for people over six feet tall.   I wanted to
meet people and go to parties at UB. 

I went to the information session advertised on the poster.  I was impressed by
 broad chest of one of the rowers at the session.  I decided I
wanted to look like that.  I put my name down.

In the spring of 2012, I was taking programming, accounting, economics, and
math classes.  I cried bitterly after my first accounting test.  All of my
other classes were very easy and I got near-perfect grades.

In February I showed up to a few land workouts, where we ran stadiums or did
some erging.  I think I missed about half the practices, just as I missed about
half of the cross country, track, or wrestling practices in high school.
Something changed, though, in the beginning of March.

I drove myself to the boathouse (more of a boat quonset hut) early in the
morning.  I was late, so they didn't put me in a boat, but told me to go for a
run.  I had been up too late on the Internet most nights the past week.  I ran
behind one of the other rowers, barely keeping up.  I described the pain to
myself in words: I felt my shoulder muscles heat up and become loose, my chest
become a golden cage, a golden twinging cage.  Sweat dried onto my face and
burned the corners of my eyes.  When we finished, I felt good.  I told the coach
that I could go on the training trip during spring break.  

Then I got in my car and wrote a tumblr entry describing how I wanted to go to
Sports College: I needed a coach to tell me when to eat, when to sleep, when to
work.  I needed to learn to control my fatigue and pain.  I could take care of
my own mind, I thought.



\section{Writing Advice}
  One
weekend in February, I flew to New York to go to a reading by Marie Calloway,
Tao Lin, Spencer Madsen, and Megan Boyle.  It was my first time in New York
since I'd flown there to meet Marie when we were Internet boyfriend and
girlfriend.  

After arriving, I took the subway to Wall Street.  I was interested in Finance
as a career and Occupy Wall Street.  Zucotti Park was desolate, surrounded by
portable police watchtowers. A handful of people were protesting, or just there.
I admired a statue of a banker eating his lunch, staring into his slim
briefcase.

I wandered away from Wall Street, trying to find somewhere to charge my
Blackberry  so that I could keep trying to get in contact with Marie. I was
worried that I would be isolated and alone for the whole long weekend.  I was
becoming slightly more confident, and I didn't want to damage this confidence by
feeling ugly, disillusioned, and left out at the reading.

I spent the rest of the long, dark afternoon and evening wandering around
Williamsburg.  My duffle bag was chafing my legs.  Eventually I sort of gave up
on Marie messaging me back, and got a hotel room at the Brooklyn Sheraton.  I
felt calmed, and took a picture of the hotel room and put it on facebook.  I
wrote an email to the Yale cofounders of a location-based chat service that I
thought was interesting, and felt thrilled and excited when I recieved a
response back: yes, they'd be happy to set up a call.  With Marie off my mind, I
got ready for bed.  

I checked facebook one last time.  There were a bunch of messages from Marie.
She was upset and drunk and wondering where I was.  I felt excited, relieved: I
messaged her that'd I'd come to her if she gave me an address, and went down to
the hotel lobby to get a cab.  It was almost midnight.

The cab stopped at the address, around Union Square.  Marie was supposed to meet
me outside.  I got out of the car and saw her.  For a moment, it was like she
was pretending not to see me, or expecting me to come to her.  She was wearing
very short shorts and looked unstable.  I waved her towards the cab.  

She got in and put her head on my shoulder.

``Thank God that was so awful, I needed to get away,'' she said.  I felt sort of
validated: I didn't really trust the Internet women friends she was staying
with.  I told the cab driver to take us to the address of the first bar I found
on yelp in Brooklyn, and wondered how he interpreted Marie and I: did he think
she was a sex worker?  

Marie started crying, and I smoothed her hair.  She cried until we got to
Brooklyn.

We went into an empty bar, and Marie was suddenly more cheerful.  She talked
about the people she had met at the readings, and told me that I was special.
After a couple of sweet whiskey drinks we left the bar, and went to find 
food.  While crossing the street, Marie asked me ``Are we going to have sex,
Patrick?'' and I said ``I don't know...probably?''

Marie got chicken wings or something and we got a cab back to the Sheraton.  The
driver couldn't find it and I felt calm and detached, trying to give directions.  

When we got back to the hotel, Marie took a shower, and got onto the bed in the
doggy style position.  ``Fuck me,'' she said.  I took off my khakis, button
down, and sweater, and kneeled on the bed with a condom.  I felt ambiguously
turned on due to the alcohol and comfort of the hotel room, and put the condom
on.  When I saw the hair of her vagina and touched my body to her slightly wet
skin, trying to push my way into her, I was surprised to find that I wasn't
hard.  I had never even really considered this happening: the other times I'd
had sex with her, everything had happened without thought or effort, more or
less.  I tried to get myself hard for about thirty seconds, feeling more and
more abject, as Marie looked back at me.  Finally I gave up, hating myself for
giving up, and threw myself down on the bed next to her, hot tears in my eyes.

Marie seemingly ignored me for a while. Then she ran her hand through my hair.
I can't remember what she said, or if I said anything.  Before long we fell
asleep, together.

When we awoke I felt calmed and drained.  The shame of the night before would be
something I'd deal with later, I figured.  Brunch with literary people and the
reading were going to happen that day.  Marie woke up and we talked in bed,
mostly about literature.  She seemed to want to talk about my writing.  "You
should make a chapbook," she said.  I figured she was being sincere.  

``I don't think my writing is good, right now,'' I said.  I hoped that I would
soon not feel like this, that I would soon have a chapbook and I would have
something to my name.

``You need to stop worrying about making yourself look good in your writing,'' she
said.

\section{IMAX and tumblr} 
I felt strangely, deeply exhausted.  On Metro North I wrote short poems with the
words heavy on my mind and posted them on tumblr.  The men and women
riding the train looked different to me, now that the summer was over, 
in the early afternoon.  My consciousness flickered intermittently; 
I drank the Vitamin Water "energy" flavor.

I wondered if this new feeling would last.
...
I wondered if this new feeling would last.

  

\section{Real Life} 

Throughout the middle half of college, the concept of ``real life'' was
important to me.  I felt that ages 18-20 had been a series of almost
uninterrupted crises, and that I hadn't had a chance to build or live in any
sense.  Instead of girlfriends, I had fucked people from Internet for a week and
never seen them again.  I had created nothing, only learned and moved and
wasted.  ``Real life'' became a self-conscious fantasy that involved
timelessness, progression towards a goal, sanity, happniness, and certain
situations of the physical world of life.  I thought of a certain overcast
Sunday in Taiwan when my host parents took me and a German exchange student to a
pool club on the top of a skyscraper.  The place was pleasant and decorated for
Halloween.  I did a few laps in the pool, and then talked about amorphous career
and international things with the German student.  He always feigned surprised
at many things I said and would say "sorry?".  I'm not sure where our host
mother was during all of this.  It was a very grey day and the club and building
were almost all white, except for the dark blue room reserved for the hot and
cold water baths.  I went into this room alone and was in the hot room when an
older Taiwanese man started making an effort to show me how to wash myself with
the pumice stone, which eventually escalated into him scrubbing my balls, which
obviously made me uncomfortable, but I put up with it.  Eventually the host
mother picked us up.  On the way home, we stopped for Italian food.  I came to
love the Taiwanese approximation of Italian food more than ``real'' Italian
food.

The fantasy of real life was strongest when I returned to Cornell in 2012.
I felt I was ready to get into a grind that would take me somewhere I couldn't
anticipate, a place of beauty and continuity.  

\section{Casual Sex}

There was a storm in the late afternoon the day before Marie left New York.
When the storm came all of the people in the office went to the windows to
watch.  I found this charming and romantic and gave up on work for the day, but
stayed late nevertheless, until after most other people had gone and the
office's air conditioning had been turned off.  

I thought back to the night before, when I'd had dinner with Marie.  It had been
very unsatisfying.  We met in Union Square and went to a Japanese restaurant,
for lack of something better to do, and talked about the usual stuff.  Marie
used my phone to try to arrange her last twenty four hours in the city.  At some
point it seemed like she tired of me and was ignoring me.  For some reason,
possibly due to the humidity that would eventually build into the storm of the
next day, I felt insane and used my chopsticks to drop my wet tofu onto the
plastic placemat she was staring at.  It was an angry gesture and it made Marie
angry.  She told me she felt disrespected and left a few minutes later.  I
wondered if it'd be the last time I saw her.

I realized that Marie had forgotten to log out of her facebook on my phone.  I
considered logging her out, but then decided against it.

I wanted to have sex with Marie again.  My feelings towards her were stale and
confused, but I felt more sane and powerful than I did in February, and I didn't
want her to ignore and forget me now.  Sitting in the empty office, I read
through her recent messages, making sure she didn't have plans, and then
messaged her: "ur mad at me but do u want to have sex."

I had a mental image of face-fucking her and doing exactly what I wanted.  I had
never quite done that before.  I thought back to ??? and how I had desired sex
on a sunday afternoon with plenty of time and without any context.  

I sent her another text: "it will be good".  Marie responded, asking where and
when.

---

An hour and a half later, Marie arrived at the lobby of the Radisson.  She
smiled and touched my arm, and then followed me into the hotel bar.  

Marie wanted a Bloody Mary and I ordered it for her, getting a water for myself.
I felt slightly self-conscious with Marie, she was wearing the GABM baseball cap
and didn't look too good.

She started up the conversation: her publisher was going to pay for her book
tour, and Glamour was going to do an article on her.  This was the stuff we were
talking about the night before.  

"I'm not surprised," I said.  I felt proud of her.

Marie's voice was barely audible in the loud Midtown bar.  She said something
about not putting on makeup.

"I can't really tell the difference," I said tiredly, thinking that her skin and
lips actually looked fine.

"Look at me," said Marie, after a second.

"What?" I said.

"Look at me."  She was smiling a bit.  She slapped me, with force.

I continued to look at her for a few seconds, then looked away.  No one at the
bar was looking at us.

"Don't do that again," I said quietly.

"Don't insult me again," Marie said.

"When did I insult you?"

"About my clothes, and my makeup..."

"I didn't mean to insult you," I said, after a moment.  I was thinking about how
maybe I was going to regret this whole thing.  Marie slipped off her barstool
and walked away.

"Where are you going?" I said unemotionally, but she was gone. 

The barman gave me some Chex Mix.  I used my phone to figure out how to cancel
the hotel room.  (It had free cancellation.)  I considered running after Marie,
but I realized I still had to pay the bill.  And, I'd be okay.  I was going to
take the train home, maybe exercise, go on the Internet, feel a little more sad
and stupid, but I'd wake up the next morning for work.

Marie sat back down a few minutes later.

"Hi," I said.  I felt better.

"Hi."

A minute passed.  I chose my words.

"When I said I couldn't tell the difference, I thought you had said that you
hadn't put on makeup...you look good."

She smiled.  I wondered if she was taking what I was saying at face value.

"Is the room like ready now?" asked Marie.

"Yeah, probably."

---

Marie lay on the bed looking at the room services menu while I looked out at the
city.  It was still light out.  

"Don't order yet," I said.  "I have to leave for my train."

She put down the menu.  I paced the room, then took off my clothes except for my
socks, and joined her on the bed.  Marie lay there.  I was silent for a moment,
then placed her hand on my penis, and she moved her head to give me a blowjob.
I thrust into her face, moved her head.

When I pulled away, she inhaled sharply, and I was surpised.  I felt sort of
exhilarated and kissed her, but her mouth tasted sour and I didn't kiss her
again.  I slowly undid the buttons of her shirt and pulled it off, and then
pulled off her shorts.  I paused to look at her pubic area fully, and then
inserted into her inexpertly.  

She gave me head on her knees.  Her eyes were closed and I forced them open with
my fingers for a moment.  She struggled her arms for a moment and then stopped.  

At one point, shortly before I came, I watched her stomach contract I thrust,
not wanting to come, I realized that this experience was definitely worth it.

I came over her neck and collarbones, it felt good.  I got her tissue paper from
the bathroom.

---

I took a short shower.  It felt good to close my eyes and feel my hair become
wet.  The bathroom was decent and it had gone ok.

When I got out, I consulted with Marie and ordered room service, then lay down
on the bed.  Marie lay her head on my shoulder.  This didn't bother me at all.
We watched the Olympics coverage for about twenty minutes.  After room service
came and I ate my burger, I got completely dressed.  

I pressed my face into the bed for a few minutes, then got up.

"If you ever need anything," I said, "hit me up."  Then I left.

---

The next morning on the train, I read through Marie's messages.  It was a
tedious process on my blackberry and the phone became very hot. 

The first thing that became apparent was that two British guys came over
after I left, and they had a threesome.  The threesome had been somewhat
disturbing.  As a footnote, she noted to her friends that she had slept with me
and that it hadn't been good, I had done lame forced dom stuff.

I didn't really care about the fact that the dom stuff had been lame and forced,
I had enjoyed it.  But I felt sad and empty seeing confirmation that I was a small part of
Marie's life, that there was always more exciting more intense and literary
stuff happening to her.

I continued searching for my name, and didn't find much.  There were some
messages after the party, where Marie was talking to Evan, someone who followed
me on Tumblr.

"I was disappointed with Patrick, he was boring," said Evan.

"Mew," said Marie.

I felt angry.  Evan was boring.  I was not boring.  It would show in the long
run. 
            
\section{Hyatt 48 Lex}
Beach House concert at the beginning of summer

I felt strangely, deeply exhausted.  On Metro North I wrote short poems with the
words heavy on my mind and posted them on tumblr.  The men and women
riding the train looked different to me, now that the summer was over, 
in the early afternoon.  My consciousness flickered intermittently; 
I drank the Vitamin Water "energy" flavor.

I wondered if this new feeling would last.
...
I wondered if this new feeling would last.

\section{The Bell Curve Part II}


\section{Un-Quitting}
The combination of the yelling, pain and sight of others failing around me would
make me expect that someone with authority would end this.


\section{Reticule}
During my first fall on the Cornell rowing team, I had a constant sense that
all of my teammates were calling me ``faggot'' or ``pussy'' behind my back.  I
imagined that if I showed my ``true self'' their friendliness would turn into
open hate, like when you kill allied soldiers in a video game, and they start
attacking you and your crosshairs go from green to red.  


\section{Hadoop}
When I arrived at Cornell for my sophomore year, I was determined to set myself
up for a prestigious internship.  In early September I put on my blazer and went
to the career fair.  I spent most of the day in the hot indoor track building
talking to employers. 

I found that I enjoyed talking to the tech companies more than the banks.  When
talking to someone from a financial services company, it always seemed like both
the recruiter and I didn't really know what the work was.  With the tech
companies, I was able to stretch my superficial knowledge of programming and
have some interesting discussions with the recruiters.  Many of them seemed to
want to believe that they were doing something really special, technology-wise.
An overweight neckbeard representing an ad tech company made a lasting impression 
by asking if I knew what Hadoop was.  

``No, what is it?'' I asked.

``Oh, man.  The next big thing.  The current big thing,'' he said,
rocking back and forth.

``Like, what is it though?''

``It's Apache's distributed MapReduce framework.  It's how we can process
millions of ad slots a minute.  You should really look it up.''

I did Google it, and felt the familiar sense of wondering what was really going
on: were terms like Hadoop, and NoSQL, or NodeJS, just buzzwords, like the names
of bands in high school, or the different sectors at an investment bank?  Or did
they \textit{actually matter}?  Over a year later, I would smile when my Database Systems class
required us to implement the PageRank algorithm using the Hadoop framework.
Computer Science had turned out to be more than emptiness, for me.

And a year after \textit{that}, one of the men indirectly responsible for the
creation of Hadoop spoke in my Information Retrieval class.  I was fascinated,
because I had watched a video where he talked about the role of 9/11 in Google's
history.  I had developed a pet theory: that the attack on the Twin Towers had
led to the creation of the Map Reduce programming paradigm, and in turn the
Hadoop framework that enabled many of the ``big data'' applications of the later
2000s.

Amit Singhal, director of search quality at Google, was away at a conference on
September 11th, 2001. As the public searched for news about the attacks on the
Twin Towers and the Pentagon, Amit and his colleagues realized that Google was
dramatically failing to meet the nation's information need. Searches for ``World
Trade Center'' led to web pages detailing the architecture of the now-destroyed
buildings, or real estate listings.  This was due to the fact that Google was
only able to index the internet about once a month: the index used to fulfill
searches did not reflect the current, dramatically different reality.  Over a
conference call, Amit and the Google engineers decided on a hacky solution: they
simply added links to relevant news articles on Google's homepage.  This didn't
work: the massive amounts of traffic directed to these articles caused the news
network's servers to crash almost instantly.

Amit and Krishna Bararat, a search architecture engineer, were trapped at the
conference center in upstate New York until planes were allowed to fly again.
Over the next few days, they sketched out the architecture of what would become
Google News, a system that would index news websites constantly, ensuring that
Google would be able to provide information about events that had just happened.
Building this system would require rethinking Google's entire data pipeline.
New programming models for distributed systems would need to be perfected in
order to have enough computing power to simulataneously index thousands of news
websites.

Over the next few years, Google News was developed, but Amit and Google realized
that everything else also needed to be indexed in ``real time''.  In order to
index the whole internet, every day, the programming methods used to create
Google News would have to be formalized.  In addition, Google realized that the
rest of the internet had to catch up with Google.  To address both of these
concerns, in 2003 Krishna released ``the MapReduce paper'', which detailed the
abstractions used by Google to think about their complicated distributed
systems.  

Something about the nature of MapReduce always felt very current and zeitgeist
to me, perhaps because of the neckbeard at career fair, or perhaps because the
paradigm reminded me of the way ``real life'' felt: many many isolated units of
data being mapped, fragmented by a hash function, and then reduced to useful
key-value pairs.


\section{LA}


\section{Winter and Spring}

I stared at the math book in dismay.  I started to read the third of nine
chapters that were going to be tested.  I had to reread sentences, I was
not understanding.  After a few minutes every few sentences my eyes would close
and I'd shake myself awake.  My eyes closed for a while.

I soon woke up angry.  I couldn't do this.  It was stupid.  I bit my tongue and
took out a notebook, and restarted reading the paragraph, writing down the
progression of equations.  I felt more awake and filled with a little
adrenaline.   

Snow blew around outside as I worked on a Business Managment homework
assignment.  The next day was the Valentine's Day Massacre, a 10k erg followed
by a 4 mile run that included a steep hill.  I was afraid.  The assignment was
tiring and dull but I felt surrounded by beauty.

I sat next to Kyle for the 10k.  I started out holding a 1:45 split.  It
was not too hard to hold it.  Seeing that I was beating Kyle was motivating.
There was nothing terribly painful at all.  I felt warm and soon enough there
were 3 kilometers to go, and I ground through them and got off the erg.  I was
going to put on pants but someone told me to start running.

Kyle was a faster runner than me.  "Football", I thought.  At the same time I
passed a few of the really large rowers, like C and A.  They were
running very slowly.  

\section{IRAs}


\section{Priceline}
Shake Carlo.  


\section{Wall of Real Depression}
In August my relationship with my grandparents worsened, and I rowed less.  I
worked on programming until after 2AM most nights, and my movements downstairs
woke up my grandfather.  I would hear his footsteps on the stairs and dread what
he said: "Do you know what time it is, Patrick?  What on \textit{earth} are you
doing?".  His face was tired and angry. "Work," I'd reply, ashamed.

I was, in fact, working, and I was enjoying it.  I didn't enjoy much of anything
else.  Every week Pop invited me to do various things with him.  Did I want to
go stand-up paddleboarding?  Did I want to go waterskiing with my cousins-once-
removed?  Did I want to get dinner after work?  

I had little or no interest in doing these things, and it hurt me.  I was
starting to wonder where my life had gone.  It seemed like my new thing was
a single desire to become ``a real software engineer'', so that I didn't have to
deal with so much bullshit.  

The sense of hope I had felt about Connecticut and adult life was starting to
unravel.  I was least happy during my morning commute.  My air conditioner was
failing and would occassionally emit hot humid air.  Every morning I drove past
a group of Mexican immigrants waiting for contractor pickup trucks.  My
grandfather also picked up Mexicans.  When he drove up to the bridge that they
waited under, they would try to pile into the back of his truck, because they
thought he was a contractor.  I got angry each time my grandfather repeated this
story, or talked about how he repsected the Mexicans because they at least
worked, unlike the other residents of South Norwalk.  One time when we were
driving down the to yacht club to go stand-up paddleboarding, I said that
it probably made more economic sense for American citizens to collect welfare
than to do hazardous, low-paying, uninsured work, so maybe he should respect the
residents of South Norwalk for being smart, like bankers. 

On the weekend of July 4th, I helped my grandfather sail his boat to Newport,
Rhode Island.  Every year since I was five or six, I'd done the "overnight sail"
from Connecticut to Rhode Island.  

Hot tears, disbelief

Relief and doom. 
Hackathon?

Bradley Manning was sentenced to 35 years in prison.  I read this news walking
through the rows and rows of brightly-colored desks at Priceline.  I left work
early and drove back to my grandparents, and read the chat messages that
eventually got him caught.   I felt that ``real life'' might be talking to
someone you barely know late at night at the Internet when you're far away from
home and have witnessed things that are too big for you to swallow, or dooming
yourself because you want to talk.  When I met up with some of my childhood
hacker friends back in Buffalo around Thanksgiving, we were always careful to
correctly gender Manning.  

I was most happy at the office, programming and listening to EDM.  I didn't want
the summer to end.


Awkwardness that did not go away.

---

President Obama was visiting upstate New York in the driving rain and the I-90
was blocked completely.  I texted William and asked if I could sleep at the
Knoll, and he said sure.  I was excited to get back to Cornell.


There was a party at the Knoll the night I arrived, or maybe the night after.  I
drove Jim and V into downtown Ithaca to get kegs.  Seeing my teammates felt
awkward but good.  I'm not sure why I felt awkward.  It was difficult to go back
to the level of intimacy we had felt as a boat, or something.  Also my head was
still in New York, thinking about programming, to some degree.    

The party started.  I felt expectant.  I gradually got drunk and put my energy
into bro-ing out with my teammates.  Soon the house was totally packed.  But
nothing \textit{happened}, beyond being greeted happily by my teammates.  I
didn't talk to any girls.  As the night progressed I felt more and more
irrelevant and I told myself that the simple and good solution was to row
harder, get better.  Then things would happen.

***

I was having trouble pushing myself.  During 2013, even in the most miserable
workout the perfect EDM song might come on, and I would feel cold adrenaline go
down my neck and feel unexpectedly good and want to push more.  Something about
the reality of the situation made me embrace it, even if meant going to the
bathroom between pieces and seeing sweat roll off my legs my scared red face in
the mirror.  

***
First time I got high
Driving home
Rowing robotically the next day

\section{Bushwick}


\section{Anna}
and I dropped away behind a wall, and then ran away, and stopped again, feeling ashamed.

\section{Moat}


\section{Facebook}


\section{Graphics}


\section{Rowing Machine}
The rowing machine was the only sacred object in my apartment.  It was situated
so that I could see the Manhattan skyline while I rowed, and in the late fall
my heart was moved by the slow pulsing of the warning lights across the river.

My parents were confused by the young adult I became in college, and when I
walked on to the rowing team, they had clung to my identity as a rower as one of
the few positive facts they knew about me.  As a result, almost all of my
Christmas gifts during those years were rowing related.  These gifts hung on the
wall next to the rowing machine: a hand drawn poster from the 1920's
commemorating Ivy League crews, parts of antique rowing shells, and,
``embarassingly'', pictures of me racing at Eastern Sprints and the IRAs.  When
people made fun of my shrine, I reminded myself that these items were proof that
my parents loved me more than any of them.


\section{Abuse of Trust}
The methods used to accomplish most cracking-style hacks are almost embarassing
to me. 

Abuse of trust is one of the most effective ways to gain access to valuable
data.  Part of Facebook's culture is that Facebook ostensibly places a high
level of trust in its employees.  I think most people are unwilling to steal
data from their employers due to organizational loyalty, intense punishments
when caught, and the difficulty of finding a buyer for that information.  On a
personal level, something else seems to stop people from violating each other's
personal data.

What does it mean, that most people, if left alone in a room with one of their
friend's diaries, would respect that friend's privacy and leave it alone?  To
me, this is strange.  The information in that diary could potentially really
benefit them in the long run, perhaps deepen the relationship with that friend,
or make you realize that you should stop being friends with that person.  The
friend is unlikely to be hurt by your having read the diary.  Does this mean
that people would rather ~not~ know the truth about the relationships in their
lives, or believe that not knowing the truth is ideal?  Or, does this mean that
many people are fundamentally uninterested in the mental lives of their friends?
These explanations all seem plausible to me.  Or, there may just be very
effective taboos in place that prevent any well-socialized individuals from
reading private information.

\section{Old Messages}
While reading through the old messages of friends and lovers, I sometimes ended
up reading conversations from my own past.  I usually felt the familiar dread
and pain that came  with seeing myself as others saw me.  None of the messages
were old enough to allow for the possibility that I'd changed since sending
them.  A person with more willpower might have used these messages as motivation
to change himself.  A more humble person might have realized that she should be
kinder to others.  I felt pain but realized that I had to ignore it.

\section{Live the Dream of 2012 in 2014}



\section{Buffalo 2015}

I spent my last winter break in Buffalo, feeling mildly euphoric most of the
time.  I went to work with Daddy.  He had decided to buy a fully electric
Volkswagen Golf, and I went with him to pick it up.  We worked strange hours,
driving to the factory around noon and leaving at eight in the evening, when it
was freezing cold and dark.  During the drive, we discussed the project, and
then often continued talking in the kitchen when we got home.  Sometimes it
seemed like we were both a little embarassed by the amount of passion we
had for the project. 

My mother was angry that Daddy spent so much time at work.  One time she
came down and interrupted us.  ``Your father's workplace is so dysfunctional,''
she said.  ``It's given him a stomach ulcer.''

I think my father and I were motivated by the the vision of a Buffalo factory
that was not dysfunctional. 

My family went out for dinner a lot, to new restaurants in revitalized areas of
Buffalo.  They were nice restaurants.  Walking in, I was often silently amazed
that the brassy, dark, spacious glow of New York restaurants could be replicated
in Buffalo.  These restaurants had a selection of good beers and I would get two
and be quite buzzed.

One time, my family was waiting for a table at the bar.  My mother was talking
about her drink, part St. Germaine and part white wine.  I hoped she was
overhearing the handsome older men talking about real estate investments in
Colorado.  One man owned a warehouse that he was now leasing to medicinal
marajuana growers.  I imagined my friends from high school meeting for drinks in
this restaurant, by the edge of the old Erie canal, discussing the wealth we had
built over our lives, wealth that extended beyond Buffalo but was still somehow
still rooted in Buffalo.  I thought of sixty years from now.  The future
seemed...nice.

Many nights I had nightmares about rowing, where I missed practice and shouted
at Coach Kennett.  I had another strange dream about Ocean Palace.  It was
disturbing.  Somehow we became close, and she was at my place after going out.
I was touching her, we were about to have sex.  Then she stopped me and said
``stop, I'll never have sex with you, you disgust me, so many aspects of your
body and personality gross me out.''  Then I experienced a telepathic slideshow
of the parts of my body that she found disgusting.  I silently backed away, full
of pain, knowing it would be hard to foget this moment, and feeling betrayed.
At the same time, I knew it was somehow deserved.  Later on in the dream I ended
up having a threesome with Ocean and her friend.  I woke up and thought of the
dream I had in the fall about a beautiful relationship with Ocean, and felt a
sense of trepidation towards the coming semester.

Other things to write about in this section: IMG1772, Zoe from tinder, New York
trip

You guys need to give me more love, attention, and respect.

\section{Amazon}



\section{Marissa}
It was surprising that within a month of moving to New York, I was in a
relationship, sleeping with someone that really lit up my life.

Marissa was my first adult relationship, I think.  It was more than a little
surreal: us, on Christmas eve, dancing in front of our small Christmas tree,
staring into each other's eyes, only half drunk.  The amount of comfort and
pleasure we managed to enjoy during any given weekend never ceased to surprise
me.  We had perfectly achieved our ``goals'', and only dimly remembered our
dreams.

Marissa was a banker, but she made being a banker look easy.  She didn't really
buy into the whole ``insane work hours'' thing.  She was effective and composed,
and had a moderate, genuine interest in the financial services industry.  Also,
she was attractive.  Blonde.  She looked like a female banker. 

The only thing that hinted at her past was the fact that she wore too much
perfume.  I think I have a thing for women --- (I always make a point to call
the ``girls'' of our generation ``women'') --- I think I have a thing for women
who wear too much perfume.

New York social life.  I was sick of the fact that all of my friends besides
William didn't seem to care for me much at all, or were very flaky.  But I also
realized the paradox in the fact that I still wanted to be popular and have many
interesting impressive friends.  I felt that this postgrad time in New York was
my last chance to realize this ideal.  I had failed in high school and college.
In certain moments I saw I was just as calculating and cruel as the
acquaintances that disappointed me.  I told myself that I wanted a social life
that was brighter and more whole, but I also realized that this might just be a
cowardly excuse to avoid the social realities that came with my appearance,
income, lack of relevance or fame, and personality.

a world doomed by the power of sexual attraction, complexity, time, entropy,
death, luck

\section{End of Rowing}
Losing to Columbia.
Conversation re: loss to Japanese team at Henley.

\section{Showcase}


\section{The End}
Alumni regatta

``Come back to Donlon with us, we have some presents for you.''
The end would probably feel like this, but leaving bigger victories and defeats.

---

\section{Notes/Miscellaneous}
Wow, a lot of this, up to the middle of the book, is about Marie.  Interactions
with her serve as the sort of framework for the first half of the book.  Then it
goes to ???  basically a structureless time.  And the third part is whatever is
happening now.

Telling yourself that the only real thing is your work will make you depressed
and undermine your work.  

I've got to have a ``game plan'' for getting this done.

When you start to fall in love
and your stomach turns to syrup
and you don't fall in love often
so you can't identify with this

Why am I so basic and boring?

But really writing is just hard like erging was hard.  When I do write it is
okay, is moving you in the correct direction.  But it is painful and hard, and
hard to get started.

``I yearn to be fucked,''
jealousy

\section{Justification}
I'm done with college now, but I don't think I really understand the past five
years.  I have about a month before I move to New York and begin adult life.
I've decided to spend this time reading my notebooks, thinking about past
events, and trying to understand them.  I'm putting all of the writing on the
Internet, and hope it'll be relevant to someone out there.  People have never
seemed particularly 'into' my artistic or literary work, but I want to believe
that if I stay focused and honest that the writing will be beautiful and
useful.  If not, I will begin my working life knowing that I tried writing, and
that writing is a way for me to communicate with myself and the people in my
life, but is not a viable career option.

This writing is kind of impossible.  I feel very conscious that the time I have
now for writing, these bus rides and train rides across the state of New York,
are my chance to write.  If I let even one slip it is a problem.  Some walls you
can't get through, right?  Is there any point to getting through this wall?  The
only writing that seems to come naturally is this sort of journaling --- it's
what I've practiced the last five years.  But it is useless when it comes to
``real writing''.  The truth is that I have little understanding of what
experiences in my life were important and literary, and I also don't have an
understanding of how to present those experiences in a way that would be
interesting to others.  There is noise and distraction and life will slip by as
it always has in the past.  Any other approach seems to lead to insanity and
suffering.  Insanity and suffering to no end, due to lack of talent, willpower,
relevance, etc.  Okay, done.  The only hope is that I can be like the shitty
rowers who just keep trying and trying and a way that you'd never expect them to
succeed but they kind of have their plan, and eventually it works.  Is that me?
I don't know.  But it kind of \textit{needs} to be me.

The fantasy that you will just work really hard and then it'll be perfect will
not happen.  Your goal should be to get this to a state where it can be
critiqued.

Are you really going to do this?  It's like sitting there before a 2k.  Remember
your true last 2k, the one you did after you got kicked out of the boat, with
L.  Going out at a pace you knew you wouldn't hold.  It would be too
painful.  At 500?  600 to go, knowing it would be too painful.  "It's good you
stopped, and didn't have a shitty piece," said L.  And remember your second last
2k, after you got sick during spring break.  That one was a thing of beauty in a
way.  

Should this be limited to the scope of the thoughts you actually had in college?
moments of significance

why are you telling me this?  why is this interesting to me?
bluntly stating is not always so interesting really
the only way you have of making your writing ``better'' is deleting parts

now feeling that I might be writing the about the wrong things.
``I feel like I am respected in the majority of situations''
social context things vs. ``true self'', insofar as true self is internal, individual
internet as the outside/inside place, literary place

I can deal with it, that is the truth
you have to say it, and I have to believe it
I just have this feeling in my body I remember
what does

\end{document}
