\documentclass[12pt]{article}

\usepackage[margin=1in]{geometry}
\setlength{\parindent}{1cm}
\linespread{1.3}
\usepackage{indentfirst}
\title{College Memoir}
\author{Patrick Steadman}
\begin{document}
\maketitle


\section{The Bell Curve}

Mommy wanted me to take more stuff to school.  She wanted me to take a down
comforter, washcloths, ironing board, posters, etc.   I tried to explain that I
didn't want to bring these things and we ended up fighting.  I had been up all
night packing and looking at the things in my childhood bedroom.  I didn't want
to bring anything that would interfere with the austerity of the life I imagined
at college.  I didn't want to be like everyone else, with their mini-fridges and
stuff.

I put on a red polo shirt from American Eagle.  My parents drove me to Ithaca.
It was a beautiful, clear, cool day.  

Moving in, I saw a beautiful Indian girl with long hair, and I  wondered if
college would live up to my dreams or if it would be more of the same.  I
sincerely hoped that I would change and become something better.  College was
the time to do that.

After I got registered and bought my textbooks, Mommy took my sisters to
Collegetown.  I went to the dining hall with my dad.  It was obvious that he was
going to lecture me.  I knew that my dad had struggled at Cornell,  to some
degree.  He had taken a semester or two off because of his grades and had worked
as the chef for his fraternity.  Then at some point he met my mom, on the
sailing team.  

After we got our chocolate milk from the steel udders Daddy took a napkin and
drew a bell curve on it.

"You know what this is, right?" he asked.

"A bell curve?" I said, bemused.

"Yeah!  The drive here, I was thinking about what to say to you.  College
is pretty long and exhausting," Daddy said, seeming a little embarassed, as
he always did.

"What does that have to do with a bell curve?"

"Well, I guess you're used to being around \textit{here} on the bell curve,
right?" said Daddy, indicating somewhere in the top 10\%.  "Or at least that's
where I was used to being, I think."  

"Yeah, I guess," I said.  I didn't like where this lecture was going.

"Well, at Cornell, you might have to get used to being around here on the curve,
or even here.  Everyone here comes from the top end of the curve.  You have to
learn to be okay with being down here on the curve sometimes."

This is cliched stuff, I thought.  I felt a little angry at my dad and stressed.
I'm not like you, Daddy, I thought.  You don't understand.  I'm not even really
\textit{on} the bell curve, in some way.  I don't care about the bell curve.
I'm sorry you did.

"Sure," I said.  "I get it.  I've heard this stuff before, I'm used to it.  Big
fish and small pond stuff."

We talked for a pretty long time.  My father seem satisfied, but I was anxious.

***

I went through all of the orientation stuff.  I had no interest in spending time
with my assigned group of Communications majors, so I drifted between groups. 

At some point, after my parents left and the activities were finished, we were
set free.  I felt a sense of freedom that reminded me of the first "open" level
of an open world video game.  I suddenly realized that no one was expecting me
home that night.

***

I found Finn, a friend from my high school, and started following him to
Collegetown.  A lot of other people were following Finn too.  It was apparent to
me that Finn didn't really know where he was going, but on the way to
Collegetown he managed to chat with a girl who was going to a party and just
sort of co-opted that information.  

Finn used me as a way to create interesting conversation for the group.
"Steadman's crazy.  Tell them about the time you hitchhiked to Montana.  I drove
this fucking kid to the highway, and he just started hitchhiking there."  "Wait
whhat?" someone would say.  I would just sort of nod and smile, secretly pleased
but also a little uneasy.  Then I would answer typical questions about running
away from home and hitchhiking.  Collegetown was full of students, many on the
lawns of houses.

The party was at a Jewish fraternity's annex.  It was crowded. Finn and I pushed
our way in.  Finn got beers for the remaining people in our original group.  We
stood by the beer pong game, Finn grinning.  Occassionaly we would have a
shouting conversation with a girl.  Invariably she would also be a freshman.  I
would feel optimistic about each girl we talked to.   "These are going to be a
great four years, man," said Finn.  "We're going to kill it here."

***

I sat in the entrance to Goldwyn-Smith hall, with my copy of Ulysses, my
notebook, and the book of annotations that explained Ulysses.  The sun was
setting over the arts quad and the stage set up for the end-of-orientation-week
DJ.  Students were arriving and I continued reading Ulysses, but enjoyed
looking at the other students and listening to the PA music.

I met George.  Guitar
Morgan and Okensheilds 

B-
Morgan and mother and fate
Ulysses

I stayed up late on the Internet
Reading lots

Writing essays with George
Hipster Runoff stuff

First snow
Mythology
Medical Leave

\section{Seroquel}
My family goes to Connecticut for Christmas, where my grandparents live.  My
mother and my grandmother (a middle school pincipal) made the decision that I
would visit a special psychologist in Connecticut.

I could feel that I was making my family miserable at my grandparent's house.
At home in Buffalo, I felt okay with my situation being on leave of absence.
After an initial reluctance to give up on the semester, I felt an intense sense
of relief once I realized I wasn't going back to school.  I lay in my bed with my
phone, listening to classical music.  It was truly winter, after Thanksgiving,
and I had a wired heating pad.  During Christmas I wanted to continue this
lifestyle.  I had made a list of 100 books I wanted to read in the new year,
and I wanted to spend eight hours a day reading.  I had my bookstand and a copy
of Harold Bloom's ``The Western Canon'', and I tried to set it up in the dining
room away from everyone.  My mom came in and told me sternly that this wasn't
okay.  I argued.  This happend for days.  I was making my family miserable.  The
thought of playing with my toddler cousins or other high-spirited activities was
painful.  My mother and even father accused me of being unempathetic and
selfish.  Very soon I was miserable as well.

We went to visit the psychologist on a workday between Christmas and New
Years.  She worked out of her own home. 

My parents went in and talked to her for at least an hour.  I had my notebook
with me, and drew the door to her office.   ...

When I returned to Buffalo, I began to row 5 kilometers most days.  The first
time I tried to row five kilometers, I stopped at about three kilometers in, but
then started up again and finished.  I wrote the total time in my Moleskeine
planner.  I also used the Moleskeine to write down the PHP/JavaScript/HTML
tutorials I completed and the books I planned to read.  This was basically how I
passed the winter of 2010-2011.  I had to withdraw from the literature classes I
took at 

Bookstore job.  

\section{Learning to Program}
Hipster Runoff personals.

\section{Robert Horning}

\section{The University at Buffalo}
While I was very involved with the Internet in the spring of 2012, I also
started to have more a ``real life''.  I saw a flyer advertising the men's
rowing team, looking for people over six feet tall.   I wanted to
meet people and go to parties at UB. 

I went to the information session advertised on the poster.  I was impressed by
 broad chest of one of the rowers at the session.  I decided I
wanted to look like that.  I put my name down.

In the spring of 2012, I was taking programming, accounting, economics, and
math classes.  I cried bitterly after my first accounting test.  All of my
other classes were very easy and I got near-perfect grades.

In February I showed up to a few land workouts, where we ran stadiums or did
some erging.  I think I missed about half the practices, just as I missed about
half of the cross country, track, or wrestling practices in high school.
Something changed, though, in the beginning of March.

I drove myself to the boathouse (more of a boat quonset hut) early in the
morning.  I was late, so they didn't put me in a boat, but told me to go for a
run.  I had been up too late on the Internet most nights the past week.  I ran
behind one of the other rowers, barely keeping up.  I described the pain to
myself in words: I felt my shoulder muscles heat up and become loose, my chest
become a golden cage, a golden twinging cage.  Sweat dried onto my face and
burned the corners of my eyes.  When we finished, I felt good.  I told the coach
that I could go on the training trip during spring break.  

Then I got in my car and wrote a tumblr entry describing how I wanted to go to
Sports College: I needed a coach to tell me when to eat, when to sleep, when to
work.  I needed to learn to control my fatigue and pain.  I could take care of
my own mind, I thought.

\section{Writing Advice}
In the winter before my 20th birthday, it seemed like my dreams were starting to
come true.  I had just returned from working in China, and I was going to school
at the University at Buffalo so that I could return to Cornell in the fall.  One
weekend in February, I flew to New York to go to a reading by Marie Calloway,
Tao Lin, Spencer Madsen, and Megan Boyle.  It was my first time in New York
since I'd flown there to meet Marie when we were Internet boyfriend and
girlfriend.  

After arriving, I took the subway to Wall Street.  I was interested in Finance
as a career and Occupy Wall Street.  Zucotti Park was desolate, surrounded by
portable police watchtowers. A handful of people were protesting, or just there.
I admired a statue of a banker eating his lunch, staring into his slim
briefcase.

I wandered away from Wall Street, trying to find somewhere to charge my
Blackberry  so that I could keep trying to get in contact with Marie. I was
worried that I would be isolated and alone for the whole long weekend.  I was
becoming slightly more confident, and I didn't want to damage this confidence by
feeling ugly, disillusioned, and left out at the reading.

I spent the rest of the long, dark afternoon and evening wandering around
Williamsburg.  My duffle bag was chafing my legs.  Eventually I sort of gave up
on Marie messaging me back, and got a hotel room at the Brooklyn Sheraton.  I
felt calmed, and took a picture of the hotel room and put it on facebook.  I
wrote an email to the Yale cofounders of a location-based chat service that I
thought was interesting, and felt thrilled and excited when I recieved a
response back: yes, they'd be happy to set up a call.  With Marie off my mind, I
got ready for bed.  

I checked facebook one last time.  There were a bunch of messages from Marie.
She was upset and drunk and wondering where I was.  I felt excited, relieved: I
messaged her that'd I'd come to her if she gave me an address, and went down to
the hotel lobby to get a cab.  It was almost midnight.

The cab stopped at the address, around Union Square.  Marie was supposed to meet
me outside.  I got out of the car and saw her.  For a moment, it was like she
was pretending not to see me, or expecting me to come to her.  She was wearing
very short shorts and looked unstable.  I waved her towards the cab.  

She got in and put her head on my shoulder.

``Thank God that was so awful, I needed to get away,'' she said.  I felt sort of
validated: I didn't really trust the Internet women friends she was staying
with.  I told the cab driver to take us to the address of the first bar I found
on yelp in Brooklyn, and wondered how he interpreted Marie and I: did he think
she was a sex worker?  

Marie started crying, and I smoothed her hair.  She cried until we got to
Brooklyn.

We went into an empty bar, and Marie was suddenly more cheerful.  She talked
about the people she had met at the readings, and told me that I was special.
After a couple of sweet whiskey drinks we left the bar, and went to find 
food.  While crossing the street, Marie asked me ``Are we going to have sex,
Patrick?'' and I said ``I don't know...probably?''

Marie got chicken wings or something and we got a cab back to the Sheraton.  The
driver couldn't find it and I felt calm and detached, trying to give directions.  

When we got back to the hotel, Marie took a shower, and got onto the bed in the
doggy style position.  ``Fuck me,'' she said.  I took off my khakis, button
down, and sweater, and kneeled on the bed with a condom.  I felt ambiguously
turned on due to the alcohol and comfort of the hotel room, and put the condom
on.  When I saw the hair of her vagina and touched my body to her slightly wet
skin, trying to push my way into her, I was surprised to find that I wasn't
hard.  I had never even really considered this happening: the other times I'd
had sex with her, everything had happened without thought or effort, more or
less.  I tried to get myself hard for about thirty seconds, feeling more and
more abject, as Marie looked back at me.  Finally I gave up, hating myself for
giving up, and threw myself down on the bed next to her, hot tears in my eyes.

Marie seemingly ignored me for a while. Then she ran her hand through my hair.
I can't remember what she said, or if I said anything.  Before long we fell
asleep, together.

When we awoke I felt calmed and drained.  The shame of the night before would be
something I'd deal with later, I figured.  Brunch with literary people and the
reading were going to happen that day.  Marie woke up and we talked in bed,
mostly about literature.  She seemed to want to talk about my writing.  "You
should make a chapbook," she said.  I figured she was being sincere.  

``I don't think my writing is good, right now,'' I said.  I hoped that I would
soon not feel like this, that I would soon have a chapbook and I would have
something to my name.

``You need to stop worrying about making yourself look good in your writing,'' she
said.

\section{IMAX and tumblr} 


\section{Real Life} 

Throughout the middle half of college, the concept of ``real life'' was
important to me.  I felt that ages 18-20 had been a series of almost
uninterrupted crises, and that I hadn't had a chance to build or live in any
sense.  Instead of girlfriends, I had fucked people from Internet for a week and
never seen them again.  I had created nothing, only learned and moved and
wasted.  ``Real life'' became a self-conscious fantasy that involved
timelessness, progression towards a goal, sanity, happniness, and certain
situations of the physical world of life.  I thought of a certain overcast
Sunday in Taiwan when my host parents took me and a German exchange student to a
pool club on the top of a skyscraper.  The place was pleasant and decorated for
Halloween.  I did a few laps in the pool, and then talked about amorphous career
and international things with the German student.  He always feigned surprised
at many things I said and would say "sorry?".  I'm not sure where our host
mother was during all of this.  It was a very grey day and the club and building
were almost all white, except for the dark blue room reserved for the hot and
cold water baths.  I went into this room alone and was in the hot room when an
older Taiwanese man started making an effort to show me how to wash myself with
the pumice stone, which eventually escalated into him scrubbing my balls, which
obviously made me uncomfortable, but I put up with it.  Eventually the host
mother picked us up.  On the way home, we stopped for Italian food.  I came to
love the Taiwanese approximation of Italian food more than ``real'' Italian
food.

The fantasy of real life was strongest when I returned to Cornell in 2012.
I felt I was ready to get into a grind that would take me somewhere I couldn't
anticipate, a place of beauty and continuity.  

\section{Casual Sex}

There was a storm in the late afternoon the day before Marie left New York.
When the storm came all of the people in the office went to the windows to
watch.  I found this charming and romantic and gave up on work for the day, but
stayed late nevertheless, until after most other people had gone and the
office's air conditioning had been turned off.  

I thought back to the night before, when I'd had dinner with Marie.  It had been
very unsatisfying.  We met in Union Square and went to a Japanese restaurant,
for lack of something better to do, and talked about the usual stuff.  Marie
used my phone to try to arrange her last twenty four hours in the city.  At some
point it seemed like she tired of me and was ignoring me.  For some reason,
possibly due to the humidity that would eventually build into the storm of the
next day, I felt insane and used my chopsticks to drop my wet tofu onto the
plastic placemat she was staring at.  It was an angry gesture and it made Marie
angry.  She told me she felt disrespected and left a few minutes later.  I
wondered if it'd be the last time I saw her.

I realized that Marie had forgotten to log out of her facebook on my phone.  I
considered logging her out, but then decided against it.

I wanted to have sex with Marie again.  My feelings towards her were stale and
confused, but I felt more sane and powerful than I did in February, and I didn't
want her to ignore and forget me now.  Sitting in the empty office, I read
through her recent messages, making sure she didn't have plans, and then
messaged her: "ur mad at me but do u want to have sex."

I had a mental image of face-fucking her and doing exactly what I wanted.  I had
never quite done that before.  I thought back to ??? and how I had desired sex
on a sunday afternoon with plenty of time and without any context.  

I sent her another text: "it will be good".  Marie responded, asking where and
when.

---

An hour and a half later, Marie arrived at the lobby of the Radisson.  She
smiled and touched my arm, and then followed me into the hotel bar.  

Marie wanted a Bloody Mary and I ordered it for her, getting a water for myself.
I felt slightly self-conscious with Marie, she was wearing the GABM baseball cap
and didn't look too good.

She started up the conversation: her publisher was going to pay for her book
tour, and Glamour was going to do an article on her.  This was the stuff we were
talking about the night before.  

"I'm not surprised," I said.  I felt proud of her.

Marie's voice was barely audible in the loud Midtown bar.  She said something
about not putting on makeup.

"I can't really tell the difference," I said tiredly, thinking that her skin and
lips actually looked fine.

"Look at me," said Marie, after a second.

"What?" I said.

"Look at me."  She was smiling a bit.  She slapped me, with force.

I continued to look at her for a few seconds, then looked away.  No one at the
bar was looking at us.

"Don't do that again," I said quietly.

"Don't insult me again," Marie said.

"When did I insult you?"

"About my clothes, and my makeup..."

"I didn't mean to insult you," I said, after a moment.  I was thinking about how
maybe I was going to regret this whole thing.  Marie slipped off her barstool
and walked away.

"Where are you going?" I said unemotionally, but she was gone. 

The barman gave me some Chex Mix.  I used my phone to figure out how to cancel
the hotel room.  (It had free cancellation.)  I considered running after Marie,
but I realized I still had to pay the bill.  And, I'd be okay.  I was going to
take the train home, maybe exercise, go on the Internet, feel a little more sad
and stupid, but I'd wake up the next morning for work.

Marie sat back down a few minutes later.

"Hi," I said.  I felt better.

"Hi."

A minute passed.  I chose my words.

"When I said I couldn't tell the difference, I thought you had said that you
hadn't put on makeup...you look good."

She smiled.  I wondered if she was taking what I was saying at face value.

"Is the room like ready now?" asked Marie.

"Yeah, probably."

---

Marie lay on the bed looking at the room services menu while I looked out at the
city.  It was still light out.  

"Don't order yet," I said.  "I have to leave for my train."

She put down the menu.  I paced the room, then took off my clothes except for my
socks, and joined her on the bed.  Marie lay there.  I was silent for a moment,
then placed her hand on my penis, and she moved her head to give me a blowjob.
I thrust into her face, moved her head.

When I pulled away, she inhaled sharply, and I was surpised.  I felt sort of
exhilarated and kissed her, but her mouth tasted sour and I didn't kiss her
again.  I slowly undid the buttons of her shirt and pulled it off, and then
pulled off her shorts.  I paused to look at her pubic area fully, and then
inserted into her inexpertly.  

She gave me head on her knees.  Her eyes were closed and I forced them open with
my fingers for a moment.  She struggled her arms for a moment and then stopped.  

At one point, shortly before I came, I watched her stomach contract I thrust,
not wanting to come, I realized that this experience was definitely worth it.

I came over her neck and collarbones, it felt good.  I got her tissue paper from
the bathroom.

---

I took a short shower.  It felt good to close my eyes and feel my hair become
wet.  The bathroom was decent and it had gone ok.

When I got out, I consulted with Marie and ordered room service, then lay down
on the bed.  Marie lay her head on my shoulder.  This didn't bother me at all.
We watched the Olympics coverage for about twenty minutes.  After room service
came and I ate my burger, I got completely dressed.  

I pressed my face into the bed for a few minutes, then got up.

"If you ever need anything," I said, "hit me up."  Then I left.

---

The next morning on the train, I read through Marie's messages.  It was a
tedious process on my blackberry and the phone became very hot. 

The first thing that became apparent was that two British guys came over
after I left, and they had a threesome.  The threesome had been somewhat
disturbing.  As a footnote, she noted to her friends that she had slept with me
and that it hadn't been good, I had done lame forced dom stuff.

I didn't really care about the fact that the dom stuff had been lame and forced,
I had enjoyed it.  But I felt sad and empty seeing confirmation that I was a small part of
Marie's life, that there was always more exciting more intense and literary
stuff happening to her.

I continued searching for my name, and didn't find much.  There were some
messages after the party, where Marie was talking to Evan, someone who followed
me on Tumblr.

"I was disappointed with Patrick, he was boring," said Evan.

"Mew," said Marie.

I felt angry.  Evan was boring.  I was not boring.  It would show in the long
run. 
            
\section{Hyatt 48 Lex}

\section{The Bell Curve Part II}


\section{Un-Quitting}
The combination of the yelling, pain and sight of others failing around me would
make me expect that someone with authority would end this.


\section{Reticule}
During my first fall on the Cornell rowing team, I had a constant sense that
all of my teammates were calling me ``faggot'' or ``pussy'' behind my back.  I
imagined that if I showed my ``true self'' their friendliness would turn into
open hate, like when you kill allied soldiers in a video game, and they start
attacking you and your crosshairs go from green to red.  


\section{Hadoop}
When I arrived at Cornell for my sophomore year, I was determined to set myself
up for a prestigious internship.  In early September I put on my blazer and went
to the career fair.  I spent most of the day in the hot indoor track building
talking to employers. 

I found that I enjoyed talking to the tech companies more than the banks.  When
talking to someone from a financial services company, it always seemed like both
the recruiter and I didn't really know what the work was.  With the tech
companies, I was able to stretch my superficial knowledge of programming and
have some interesting discussions with the recruiters.  Many of them seemed to
want to believe that they were doing something really special, technology-wise.
An overweight neckbeard representing an ad tech company made a lasting impression 
by asking if I knew what Hadoop was.  

``No, what is it?'' I asked.

``Oh, man.  The next big thing.  The current big thing,'' he said,
rocking back and forth.

``Like, what is it though?''

``It's Apache's distributed MapReduce framework.  It's how we can process
millions of ad slots a minute.  You should really look it up.''

I did Google it, and felt the familiar sense of wondering what was really going
on: were terms like Hadoop, and NoSQL, or NodeJS, just buzzwords, like the names
of bands in high school, or the different sectors at an investment bank?  Or did
they \textit{actually matter}?  Over a year later, I would smile when my Database Systems class
required us to implement the PageRank algorithm using the Hadoop framework.
Computer Science had turned out to be more than emptiness, for me.

And a year after \textit{that}, one of the men indirectly responsible for the
creation of Hadoop spoke in my Information Retrieval class.  I was fascinated,
because I had watched a video where he talked about the role of 9/11 in Google's
history.  I had developed a pet theory: that the attack on the Twin Towers had
led to the creation of the Map Reduce programming paradigm, and in turn the
Hadoop framework that enabled many of the ``big data'' applications of the later
2000s.

Amit Singhal, director of search quality at Google, was away at a conference on
September 11th, 2001. As the public searched for news about the attacks on the
Twin Towers and the Pentagon, Amit and his colleagues realized that Google was
dramatically failing to meet the nation's information need. Searches for ``World
Trade Center'' led to web pages detailing the architecture of the now-destroyed
buildings, or real estate listings.  This was due to the fact that Google was
only able to index the internet about once a month: the index used to fulfill
searches did not reflect the current, dramatically different reality.  Over a
conference call, Amit and the Google engineers decided on a hacky solution: they
simply added links to relevant news articles on Google's homepage.  This didn't
work: the massive amounts of traffic directed to these articles caused the news
network's servers to crash almost instantly.

Amit and Krishna Bararat, a search architecture engineer, were trapped at the
conference center in upstate New York until planes were allowed to fly again.
Over the next few days, they sketched out the architecture of what would become
Google News, a system that would index news websites constantly, ensuring that
Google would be able to provide information about events that had just happened.
Building this system would require rethinking Google's entire data pipeline.
New programming models for distributed systems would need to be perfected in
order to have enough computing power to simulataneously index thousands of news
websites.

Over the next few years, Google News was developed, but Amit and Google realized
that everything else also needed to be indexed in ``real time''.  In order to
index the whole internet, every day, the programming methods used to create
Google News would have to be formalized.  In addition, Google realized that the
rest of the internet had to catch up with Google.  To address both of these
concerns, in 2003 Krishna released ``the MapReduce paper'', which detailed the
abstractions used by Google to think about their complicated distributed
systems.  

Something about the nature of MapReduce always felt very current and zeitgeist
to me, perhaps because of the neckbeard at career fair, or perhaps because the
paradigm reminded me of the way ``real life'' felt: many many isolated units of
data being mapped, fragmented by a hash function, and then reduced to useful
key-value pairs.


\section{LA}


\section{Winter and Spring}


\section{IRAs}


\section{Priceline}
Shake Carlo.  


\section{Wall of Real Depression}
In August my relationship with my grandparents worsened, and I rowed less.  I
worked on programming until after 2AM most nights, and my movements downstairs
woke up my grandfather.  I would hear his footsteps on the stairs and dread what
he said: "Do you know what time it is, Patrick?  What on \textit{earth} are you
doing?".  His face was tired and angry. "Work," I'd reply, ashamed.

I was, in fact, working, and I was enjoying it.  I didn't enjoy much of anything
else.  Every week Pop invited me to do various things with him.  Did I want to
go stand-up paddleboarding?  Did I want to go waterskiing with my cousins-once-
removed?  Did I want to get dinner after work?  

I had little or no interest in doing these things, and it hurt me.  I was
starting to wonder where my life had gone.  It seemed like my new thing was
a single desire to become ``a real software engineer'', so that I didn't have to
deal with so much bullshit.  

The sense of hope I had felt about Connecticut and adult life was starting to
unravel.  I was least happy during my morning commute.  My air conditioner was
failing and would occassionally emit hot humid air.  Every morning I drove past
a group of Mexican immigrants waiting for contractor pickup trucks.  My
grandfather also picked up Mexicans.  He often talked about how when he drove up
to the bridge that they waited under, too many would try to pile into the back
of his truck, because they thought he was a contractor.  I got angry each time
my grandfather talked about how he repsected the Mexicans because they at least
worked, unlike the other residents of South Norwalk.  One time when we were
driving down the to yacht club to go stand-up paddleboarding, I responded that
it probably made more economic sense for American citizens to collect welfare
than to do hazardous, low-paying, uninsured work, so maybe he should respect the
residents of South Norwalk for being smart, like bankers. 

I was starting to hate Connecticut.  I was most happy at the office, programming
and listening to EDM.  I didn't want the summer to end.

President Obama was visiting upstate New York in the driving rain and the I-90
was blocked completely.  I texted William and asked if I could sleep at the
Knoll, and he said sure.  I was excited to get back to Cornell.

There was a party at the Knoll the night I arrived, or maybe the night after.  I
drove Jim and V into downtown Ithaca to get kegs.  Seeing my teammates felt
awkward but good.  I'm not sure why I felt awkward.  It was difficult to go back
to the level of intimacy we had felt as a boat, or something.  Also my head was
still in New York, thinking about programming, to some degree.    

The party started.  I felt expectant.  I gradually got drunk and put my energy
into bro-ing out with my teammates.  Soon the house was totally packed.  But
nothing \textit{happened}, beyond being greeted happily by my teammates.  I
didn't talk to any girls.  As the night progressed I felt more and more
irrelevant and I told myself that the simple and good solution was to row
harder, get better.  Then things would happen.

***

I was having trouble pushing myself.  During 2013, even in the most miserable
workout the perfect EDM song might come on, and I would feel cold adrenaline go
down my neck and feel unexpectedly good and want to push more.  Something about
the reality of the situation made me embrace it, even if meant going to the
bathroom between pieces and seeing sweat roll off my legs my scared red face in
the mirror.  

***
First time I got high
Driving home
Rowing robotically the next day

\section{Bushwick}


\section{Annabelle}


\section{Moat}


\section{Facebook}


\section{Graphics}


\section{Rowing Machine}
The rowing machine was the only sacred object in my apartment.  It was situated
so that I could see the Manhattan skyline while I rowed, and in the late fall
my heart was moved by the slow pulsing of the warning lights across the river.

My parents were confused by the young adult I became in college, and when I
walked on to the rowing team, they had clung to my identity as a rower as one of
the few positive facts they knew about me.  As a result, almost all of my
Christmas gifts during those years were rowing related.  These gifts hung on the
wall next to the rowing machine: a hand drawn poster from the 1920's
commemorating Ivy League crews, parts of antique rowing shells, and,
``embarassingly'', pictures of me racing at Eastern Sprints and the IRAs.  When
people made fun of my shrine, I reminded myself that these items were proof that
my parents loved me more than any of them.


\section{Abuse of Trust}
The methods used to accomplish most cracking-style hacks are almost embarassing
to me. 

Abuse of trust is one of the most effective ways to gain access to valuable
data.  Part of Facebook's culture is that Facebook ostensibly places a high
level of trust in its employees.  I think most people are unwilling to steal
data from their employers due to organizational loyalty, intense punishments
when caught, and the difficulty of finding a buyer for that information.  On a
personal level, something else seems to stop people from violating each other's
personal data.

What does it mean, that most people, if left alone in a room with one of their
friend's diaries, would respect that friend's privacy and leave it alone?  To
me, this is strange.  The information in that diary could potentially really
benefit them in the long run, perhaps deepen the relationship with that friend,
or make you realize that you should stop being friends with that person.  The
friend is unlikely to be hurt by your having read the diary.  Does this mean
that people would rather ~not~ know the truth about the relationships in their
lives, or believe that not knowing the truth is ideal?  Or, does this mean that
many people are fundamentally uninterested in the mental lives of their friends?
These explanations all seem plausible to me.  Or, there may just be very
effective taboos in place that prevent any well-socialized individuals from
reading private information.

\section{Old Messages}
While reading through the old messages of friends and lovers, I sometimes ended
up reading conversations from my own past.  I usually felt the familiar dread
and pain that came  with seeing myself as others saw me.  None of the messages
were old enough to allow for the possibility that I'd changed since sending
them.  A person with more willpower might have used these messages as motivation
to change himself.  A more humble person might have realized that she should be
kinder to others.  I felt pain but realized that I had to ignore it.

\section{Buffalo 2015}

I spent my last winter break in Buffalo, feeling mildly euphoric most of the
time.  I went to work with Daddy.  He had decided to buy a fully electric
Volkswagen Golf, and I went with him to pick it up.  We worked strange hours,
driving to the factory around noon and leaving at eight in the evening, when it
was freezing cold and dark.  During the drive, we discussed the project, and
then often continued talking in the kitchen when we got home.  Sometimes it
seemed like we were both a little embarassed by the amount of passion we
had for the project. 

My mother was angry that Daddy spent so much time at work.  One time she
came down and interrupted us.  "Your father's workplace is so dysfunctional,"
she said.  "It's given him a stomach ulcer."  

I think my father and I were motivated by the the vision of a Buffalo factory
that was not dysfunctional. 

My family went out for dinner a lot, to new restaurants in revitalized areas of
Buffalo.  They were nice restaurants.  Walking in, I was often silently amazed
that the brassy, dark, spacious glow of New York restaurants could be replicated
in Buffalo.  These restaurants had a selection of good beers and I would get two
and be quite buzzed.

One time, my family was waiting for a table at the bar.  My mother was talking
about her drink, part St. Germaine and part white wine.  I hoped she was
overhearing the handsome older men talking about real estate investments in
Colorado.  One man owned a warehouse that he was now leasing to medicinal
marajuana growers.  I imagined my friends from high school meeting for drinks in
this restaurant, by the edge of the old Erie canal, discussing the wealth we had
built over our lives, wealth that extended beyond Buffalo but was still somehow
still rooted in Buffalo.  I thought of sixty years from now.  The future
seemed...nice.

Many nights I had nightmares about rowing, where I missed practice and shouted
at Coach Kennett.  I had another strange dream about Ocean Palace.  It was
disturbing.  Somehow we became close, and she was at my place after going out.
I was touching her, about to have sex.  Then she stopped me and said "stop, I'll
never have sex with you, you disgust me, so many aspects of your body and
personality gross me out."  Then I experienced a telepathic slideshow of the
parts of my body that she found disgusting.  I silently backed away, full of
pain, knowing it would be hard to foget this moment, and feeling betrayed. But
somehow I knew it was deserved.  Later on in the dream I ended up having a
threesome with Ocean and her friend.  I woke up and thought of the dream I had
in the fall about a beautiful relationship with Ocean, and felt a sense of
trepidation towards the coming semester.

Other things to write about in this section: IMG1772, Zoe from tinder, New York
trip

\section{Amazon}



\section{Marissa}
It was surprising that within a month of moving to New York, I was in a
relationship, sleeping with someone that really lit up my life.

Marissa was my first adult relationship, I think.  It was more than a little
surreal: us, on Christmas eve, dancing in front of our small Christmas tree,
staring into each other's eyes, only half drunk.  The amount of comfort and
pleasure we managed to enjoy during any given weekend never ceased to surprise
me.  We had perfectly achieved our ``goals'', and only dimly remembered our
dreams.

Marissa was a banker, but she made being a banker look easy.  She didn't really
buy into the whole ``insane work hours'' thing.  She was effective and composed,
and had a moderate, genuine interest in the financial services industry.  Also,
she was attractive.  Blonde.  She looked like a female banker. 

The only thing that hinted at her past was the fact that she wore too much
perfume.  I think I have a thing for women --- (I always make a point to call
the ``girls'' of our generation ``women'') --- I think I have a thing for women
who wear too much perfume.

New York social life.  I was sick of the fact that all of my friends besides
William didn't seem to care for me much at all, or were very flaky.  But I also
realized the paradox in the fact that I still wanted to be popular and have many
interesting impressive friends.  I felt that this postgrad time in New York was
my last chance to realize this ideal.  I had failed in high school and college.
In certain moments I saw I was just as calculating and cruel as the
acquaintances that disappointed me.  I told myself that I wanted a social life
that was brighter and more whole, but I also realized that this might just be a
cowardly excuse to avoid the social realities that came with my appearance,
income, lack of relevance or fame, and personality.


\section{End of Rowing}
Losing to Columbia.
Conversation re: loss to Japanese team at Henley.

\section{Showcase}


\section{The End}
Alumni regatta

"Come back to Donlon with us, we have some presents for you."
The end would probably feel like this, but leaving bigger victories and defeats.

---

\section{Notes/Miscellaneous}
Wow, a lot of this, up to the middle of the book, is about Marie.  Interactions
with her serve as the sort of framework for the first half of the book.  Then it
goes to ???  basically a structureless time.  And the third part is whatever is
happening now.

Telling yourself that the only real thing is your work will make you depressed
and undermine your work.  

I've got to have a "game plan" for getting this done.

When you start to fall in love
and your stomach turns to syrup
and you don't fall in love often
so you can't identify with this

Why am I so basic and boring?

But really writing is just hard like erging was hard.  When I do write it is
okay, is moving you in the correct direction.  But it is painful and hard, and
hard to get started.

\section{Justification}
I'm done with college now, but I don't think I really understand the past five
years.  I have about a month before I move to New York and begin adult life.
I've decided to spend this time reading my notebooks, thinking about past
events, and trying to understand them.  I'm putting all of the writing on the
Internet, and hope it'll be relevant to someone out there.  People have never
seemed particularly 'into' my artistic or literary work, but I want to believe
that if I stay focused and honest that the writing will be beautiful and
useful.  If not, I will begin my working life knowing that I tried writing, and
that writing is a way for me to communicate with myself and the people in my
life, but is not a viable career option.

This writing is kind of impossible.  I feel very conscious that the time I have
now for writing, these bus rides and train rides across the state of New York,
are my chance to write.  If I let even one slip it is a problem.  Some walls you
can't get through, right?  Is there any point to getting through this wall?  The
only writing that seems to come naturally is this sort of journaling --- it's
what I've practiced the last five years.  But it is useless when it comes to
``real writing''.  The truth is that I have little understanding of what
experiences in my life were important and literary, and I also don't have an
understanding of how to present those experiences in a way that would be
interesting to others.  There is noise and distraction and life will slip by as
it always has in the past.  Any other approach seems to lead to insanity and
suffering.  Insanity and suffering to no end, due to lack of talent, willpower,
relevance, etc.  Okay, done.  The only hope is that I can be like the shitty
rowers who just keep trying and trying and a way that you'd never expect them to
succeed but they kind of have their plan, and eventually it works.  Is that me?
I don't know.  But it kind of \textit{needs} to be me.

The fantasy that you will just work really hard and then it'll be perfect will
not happen.  Your goal should be to get this to a state where it can be
critiqued.

Are you really going to do this?  It's like sitting there before a 2k.  Remember
your true last 2k, the one you did after you got kicked out of the boat, with
L.  Going out at a pace you knew you wouldn't hold.  It would be too
painful.  At 500?  600 to go, knowing it would be too painful.  "It's good you
stopped, and didn't have a shitty piece," said L.  And remember your second last
2k, after you got sick during spring break.  That one was a thing of beauty in a
way.  
\end{document}
