\documentclass[12pt]{article}

\usepackage[margin=1in]{geometry}
\setlength{\parindent}{1cm}
\linespread{1.3}
\usepackage{indentfirst}
\title{College Memoir}
\author{Patrick Steadman}
\begin{document}
\maketitle

\section{Writing Advice}
In the winter before my 20th birthday, things were going better for me.  I had
just returned from working in China, and I was going to school at the University
at Buffalo so that I could return to Cornell in the fall.  One weekend in
February, I flew to New York to go to a reading by Marie Calloway, Tao Lin,
Spencer Madsen, and Megan Boyle.  It was my first time in New York since I'd
flown there to meet Marie for the first time, when we were Internet boyfriend and
girlfriend.  

After arriving, I took the subway to Wall Street.  I was interested in Finance
as a career, but I was also interested in Occupy Wall Street and Zucotti Park.
Zucotti Park was filled with police, and there were portable watch towers and
surveillance instruments surrounding its perimeter.  A handful of people were
protesting, or just there.  It was cold but clear. I admired a statue of a
classic-looking banker eating his lunch, staring into his slim briefcase.

I wandered away from Wall Street, trying to find somewhere to charge my phone so
that I could keep trying to get in contact with Marie.  We weren't on terribly
good terms then, for whatever reason.  I was worried that I would be isolated
and alone for the whole long weekend.  I was becoming slightly more confident,
and I didn't want to damage this confidence by feeling ugly, disillusioned, and
left out at the reading.

I spent the rest of the long, dark afternoon and evening wandering around
Williamsburg.  My duffle bag was chafing my legs.  Eventually I sort
of gave up on Marie messaging me back, and got a hotel room at the Brooklyn
Sheraton.  I felt calmed, and took a picture of the hotel room and put it on
facebook.  I wrote an email to the Yale cofounders of a location-based chat
service that I thought was interesting, and felt thrilled and excited when I
recieved a response: yes, they'd be happy to set up a call.  With Marie off my
mind, I got ready for bed.  

I checked facebook one last time.  Marie had finally responded to my messages.
She was upset and drunk and wondering where I was.  I felt excited, relieved: I
messaged her that'd I'd come to her if she gave me an address, and went down to
the hotel lobby to get a cab.  It was almost midnight.

The cab stopped at the address, around Union Square.  Marie was supposed to meet
me outside. I was confused.  I got out of the car and saw her.  For a moment, it
was like she was pretending not to see me, or expecting me to come to her.  She
was wearing very short shorts and looked unstable.  I waved her towards the cab.  

She got in and leaned into me.

``Thank God that was so awful, I needed to get away,'' she said.  I felt sort of
validated: I didn't really trust the Internet women friends she was staying
with.  I told the cab driver to take us to the address of the first bar I found
on yelp in Brooklyn, and wondered how he interpreted Marie and I: did he think
she was a sex worker?  

Marie started crying, and I smoothed her hair.  She cried until we got to
Brooklyn.

We went into an empty bar, and Marie was suddenly more cheerful.  She talked
about the people she had met at the readings, and told me that I was special.
After a couple of sweet drinks we left the bar, and went to find greasy food.
While crossing the street, Marie asked me ``Are we going to have sex, Patrick?''
and I said ``I don't know...probably?''

Marie got chicken wings or something and we got a cab back to the Sheraton.  The
driver couldn't find it and I felt calm and detached, trying to give directions.  

When we got back to the hotel, Marie took a shower, and got onto the bed in the
doggy style position.  ``Fuck me,'' she said.  I took off my khakis, button down,
and sweater, and kneeled on the bed with a condom.  I felt ambiguously turned on
due to the alcohol and comfort of the hotel room, and put the condom on.  When I
saw the hair of her vagina and touched my body to her slightly wet skin, trying
to push my way into her, I was surprised to find that I wasn't hard.  I had
never even really considered this happening: when we had had sex the last time,
everything had happened more or less without thought or effort.  I tried to get
myself hard for about thirty seconds, feeling more and more abject, as Marie
looked back at me.  Finally I gave up, hating myself for giving up, and threw
myself down on the bed next to her, hot tears in my eyes.

Marie seemingly ignored me for a while. Then she ran her hand through my hair.
I can't remember what she said, or I said.  Before long we fell asleep,
together.

When we awoke I felt calmed and drained.  The shame of the night before would be
something I'd deal with later, I figured.  Brunch with literary people and the
reading were going to happen that day.  Marie woke up and we talked in bed,
mostly about literature.  She seemed to want to talk about my writing.  "You
should make a chapbook," she said.  I figured she was being sincere.  

``You need to stop worrying about making yourself look good in your writing,'' she
said.

\section{Seroquel}
My family goes to Connecticut for Christmas, where my grandparents live.  Phone
conversations between my mother and my grandmother (a middle school principal)
resulted in the decision that I would visit a special psychologist in
Connecticut.

I could feel that I was making my family at my grandparent's house.  I was okay
with my situation, living at home.  I felt an intense sense of relief, lying in
my bed with my phone, listening to classical music.  It was truly winter, after
Thanksgiving, and I had a wired heating pad.  During Christmas I wanted to
continue this lifestyle.  I had make a list of 100 books I wanted to read in the
next year, and I wanted to spend eight hours a day reading.  I had my bookstand
and a copy of Harold Bloom's ``The Western Canon'', and I tried to set it up in
the dining room away from everyone.  My mom came in and told me sternly that
this wasn't okay.  I argued.  This happend for days.  I was making my family
miserable.  The thought of playing with my toddler cousins or other
high-spirited activities was painful.  My mother and even father accused me of
being unempathetic and selfish.  Very soon I was miserable as well.

We went to visit the psychologist on a workday between Christmas and New
Years.  She worked out of her own home. 

God this shit is pretty boring.

When I returned to Buffalo, I began to row 5 kilometers most days.  The first
time I tried to row five kilometers, I stopped at about three kilometers in, but
then started up again and finished.  I wrote the times in my Moleskeine planner.
I also used the Moleskeine to write down the PHP/JavaScript/HTML tutorials I did
and the books I planned to read.  This was basically how I passed the winter of
2010-2011.  I had to withdraw from the literature classes I took at 

Bookstore job.

\section{The University at Buffalo}
While I was very involved with the Internet in the spring of 2012, I also
started having more of a ``real life''.  I saw a flyer advertising the men's
rowing team, looking for people over six feet tall.  While I was having a
pretty good time posting on tumblr and getting my homework done, I wanted to
meet people and go to parties at UB. 

I went to the information session advertised on the poster.  I was impressed by
 broad chest of one of the rowers at the information session.  I decided I
wanted to look like that.  I put my name down.

I sort of knew how to row, because my dad bought a rowing machine when I was a
kid.  In the spring of 2012, I was taking programming, accounting, economics, and
math classes.  I cried bitterly after my first accounting test.  All of my
other classes were very easy and I got near-perfect grades.

In February I showed up to a few land workouts, where we ran stadiums or did
some erging.  I think I missed about half the practices, just as I missed about
half of the cross country, track, or wrestling practices in high school.
Something changed, though, in the beginning of March.

I drove myself to the boathouse (more of a boat hut) early in the morning.  I
was late, so they didn't put me in a boat, but told me to go for a run.  I had
been up all night, up all the last week on the computer.  I ran behind one of
the other rowers, barely keeping up.  I described the pain to myself in words: I
felt my shoulder muscles heat up and become loose, my chest become a golden
cage, a golden twinging cage.  Sweat dried onto my face and into my eyes.  When
we finished, I felt good.  I told the coach that I could go on the training trip
during spring break.  Then I got in my car and wrote a tumblr entry describing
how I wanted to go to Sports College: I needed a coach to tell me when to eat,
when to sleep, when to work.  I needed to learn to control my fatigue and pain.
I could take care of my own mind, I thought.

\section{First Story}
I am so sick of thinking and writing about the first few months of school.  I
could write about my hopes, disappointments, time in the library, experiences
going to parties, or meeting my friend George, who I realize will be like family
to me when I move to New York. 




\section{Real Life}
Throughout the middle half of college, the concept of ``real life'' was
important to me.  I felt that the ages of 18-20 had been a series of almost
uninterrupted crises, and that I hadn't had a chance to build or live in any
sense.  Instead of girlfriends, I had fucked people from Internet for a week and
never seen them again.  I created nothing, only learned and moved and wasted.
``Real life'' became a self-conscious fantasy that I had, that involved infinite
timelessness, slow progression towards a unreachable goal, and certain physical
places in my mental landscape.

The fantasy of real life was strongest when I returned to Cornell in 2012.
I felt I was ready to get into a grind that would take me somewhere I couldn't
anticipate, a place of beauty and continuity.  

\section{Casual Sex}

\section{Captain of the Crew Team}

\section{Un-Quitting}
Bitterly lonely when I went back.

\section{Reticule}
During my first fall on the Cornell rowing team, I had a constant sense that
all of my teammates were calling me ``faggot'' or ``pussy'' behind my back.  I
imagined that if I showed my ``true self'' their friendliness would turn into
open hate, like when you kill allied soldiers in a video game, and they start
attacking you and your crosshairs go from green to red.  

\section{Hadoop}
When I arrived at Cornell for my sophomore year, I was determined to set myself
up for a prestigious internship.  In early September I put on my blazer and went
to the career fair.  I spent most of the day in the hot indoor track building
talking to employers. 

I found that I enjoyed talking to the tech companies more than the banks.  When
talking to someone from a financial services company, it always seemed like both
the recruiter and I didn't really know what the work was.  With the tech
companies, I was able to stretch my superficial knowledge of programming and
have some interesting discussions with the recruiters.  Many of them seemed to
want to believe that they were doing something really special, technology-wise.
An overweight neckbeard representing an ad tech company made a lasting impression 
by asking if I knew what Hadoop was.  

``No, what is it?'' I asked.

``Oh, man.  The next big thing.  The current big thing,'' he said,
rocking back and forth.

``Like, what is it though?''

``It's Apache's distributed MapReduce framework.  It's how we can process
millions of ad slots a minute.  You should really look it up.''

I did Google it, and felt the familiar sense of wondering what was really going
on: were terms like Hadoop, and NoSQL, or NodeJS, just buzzwords, like the names
of bands in high school, or the different sectors at an investment bank?  Or did
they \textit{actually matter}?  Over a year later, I would smile when my Database Systems class
required us to implement the PageRank algorithm using the Hadoop framework.
Computer Science had turned out to be more than emptiness, for me.

And a year after \textit{that}, one of the men indirectly responsible for the
creation of Hadoop spoke in my Information Retrieval class.  I was fascinated,
because I had watched a video where he talked about the role of 9/11 in Google's
history.  I had developed a pet theory: that the attack on the Twin Towers had
led to the creation of the Map Reduce programming paradigm, and in turn the
Hadoop framework that enabled many of the ``big data'' applications of the later
2000s.

Amit Singhal, director of search quality at Google, was away at a conference on
September 11th, 2001. As the public searched for news about the attacks on the
Twin Towers and the Pentagon, Amit and his colleagues realized that Google was
dramatically failing to meet the nation's information need. Searches for ``World
Trade Center'' led to web pages detailing the architecture of the now-destroyed
buildings, or real estate listings.  This was due to the fact that Google was
only able to index the internet about once a month: the index used to fulfill
searches did not reflect the current, dramatically different reality.  Over a
conference call, Amit and the Google engineers decided on a hacky solution: they
simply added links to relevant news articles on Google's homepage.  This didn't
work: the massive amounts of traffic directed to these articles caused the news
network's servers to crash almost instantly.

Amit and Krishna Bararat, a search architecture engineer, were trapped at the
conference center in upstate New York until planes were allowed to fly again.
Over the next few days, they sketched out the architecture of what would become
Google News, a system that would index news websites constantly, ensuring that
Google would be able to provide information about events that had just happened.
Building this system would require rethinking Google's entire data pipeline.
New programming models for distributed systems would need to be perfected in
order to have enough computing power to simulataneously index thousands of news
websites.

Over the next few years, Google News was developed, but Amit and Google realizes
that just about everything needed to be indexed in ``real time''.  In order to
index the whole internet, every day, the programming methods used to create
Google News would have to be formalized.  In addition, Google realized that the
rest of the internet had to catch up with Google.  To address both of these
concerns, in 2003 Krishna released ``the MapReduce paper'', which detailed the
abstractions used by Google to think about their complicated distributed
systems.  

Something about the nature of MapReduce always felt very current and zeitgeisty
to me, perhaps because of the adtech neckbeard at career fair, or perhaps
because the paradigm reminded me of the way ``real life'' felt: many many isolated
units of data being mapped, fragmented by a hash function, and then reduced to
useful key-value pairs.

\section{Bushwick}
My fantasy of "real life" was realized in the winter of 2013.  The fall of 2013
was the most challenging semester of my three continuous years at Cornell.  I
was planning to drive back to Buffalo before the start of the semester, but
President Obama was visiting upstate New York in the driving rain and the I-90
was blocked completely.  I texted Will and asked if I could sleep and park at the
Knoll, and he said sure.  I was excited to go back to Cornell.

There was a party at the Knoll the night I arrived, or maybe the night after.  I
drove Jim and V into downtown Ithaca to get kegs.  Seeing my teammates felt
awkward but good.  I'm not sure why I felt awkward.  It was difficult to go back
to the level of intimacy we had felt as a boat, or something.  Also my head was
still in New York, thinking about programming, to some degree.  I  



\section{Rowing Machine}
The rowing machine was the only sacred object in my apartment.  It was situated
so that I could see the Manhattan skyline while I rowed, and in the late fall
my heart was moved by the slow pulsing of the warning lights across the river.

My parents were confused by the young adult I became in college, and when I
walked on to the rowing team, they had clung to my identity as a rower as one of
the few positive facts they knew about me.  As a result, almost all of my
Christmas gifts during those years were rowing related.  These gifts hung on the
wall next to the rowing machine: a hand drawn poster from the 1920's
commemorating Ivy League crews, parts of antique rowing shells, and,
``embarassingly'', pictures of me racing at Eastern Sprints and the IRAs.  When
people made fun of my shrine, I reminded myself that these items were proof that
my parents loved me more than any of them.


\section{Abuse of Trust}
The methods used to accomplish most cracking-style hacks are almost embarassing
to me. 

Abuse of trust is one of the most effective ways to gain access to valuable
data.  Part of Facebook's culture is that Facebook ostensibly places a high
level of trust in its employees.  I think most people are unwilling to steal
data from their employers due to organizational loyalty, intense punishments
when caught, and the difficulty of finding a buyer for that information.  On a
personal level, something else seems to stop people from violating each other's
personal data.

What does it mean, that most people, if left alone in a room with one of their
friend's diaries, would respect that friend's privacy and leave it alone?  To
me, this is strange.  The information in that diary could potentially really
benefit them in the long run, perhaps deepen the relationship with that friend,
or make you realize that you should stop being friends with that person.  The
friend is unlikely to be hurt by your having read the diary.  Does this mean
that people would rather ~not~ know the truth about the relationships in their
lives, or believe that not knowing the truth is ideal?  Or, does this mean that
many people are fundamentally uninterested in the mental lives of their friends?
These explanations all seem plausible to me.  Or, there may just be very
effective taboos in place that prevent any well-socialized individuals from
reading private information.

\section{Old Messages}
While reading through the old messages of friends and lovers, I sometimes ended
up reading conversations from my own past.  I usually felt the familiar dread
and pain that came  with seeing myself as others saw me.  None of the messages
were old enough to allow for the possibility that I'd changed since sending
them.  A person with more willpower might have used these messages as motivation
to change himself.  A more humble person might have realized that she should be
kinder to others.

\section{Marissa}
It was surprising that within a month of moving to New York, I was in a
relationship, sleeping with someone that really lit up my life.

Marissa was my first adult relationship, I think.  It was more than a little
surreal: us, on Christmas eve, dancing in front of our small Christmas tree,
staring into each other's eyes, only half drunk.  The amount of comfort and
pleasure we managed to enjoy during any given weekend never ceased to surprise
me.  We had perfectly achieved our ``goals'', and only dimly remembered our
dreams.

Marissa was a banker, but she made being a banker look easy.  She didn't really
buy into the whole ``insane work hours'' thing.  She was effective and composed,
and had a moderate, genuine interest in the financial services industry.  Also,
she was attractive.  Blonde.  She looked like a female banker. 

The only thing that hinted at her past was the fact that she wore too much
perfume.  I think I have a thing for women --- (I always make a point to call
the ``girls'' of our generation ``women'') --- I think I have a thing for women
who wear too much perfume.


\section{Notes/Miscellaneous}
Wow, a lot of this, up to the middle of the book, is about Marie.  Interactions
with her serve as the sort of framework for the first half of the book.  Then it
goes to ???  basically a structureless time.  And the third part is whatever is
happening now.

\section{Justification}
I'm done with college now, but I don't think I really understand the past five
years.  I have about a month before I move to New York and begin adult life.
I've decided to spend this time reading my notebooks, thinking about past
events, and trying to understand them.  I'm putting all of the writing on the
Internet, and hope it'll be relevant to someone out there.  People have never
seemed particularly 'into' my artistic or literary work, but I want to believe
that if I stay focused and honest that the writing will be beautiful and
useful.  If not, I will begin my working life knowing that I tried writing, and
that writing is a way for me to communicate with myself and the people in my
life, but is not a viable career option.

This writing is kind of impossible.  I feel very conscious that the time I have
now for writing, these bus rides and train rides across the state of New York,
are my chance to write.  If I let even one slip it is a problem.  Some walls you
can't get through, right?  Is there any point to getting through this wall?  The
only writing that seems to come naturally is this sort of journaling --- it's
what I've practiced the last five years.  But it is useless when it comes to
``real writing''.  The truth is that I have little understanding of what
experiences in my life were important and literary, and I also don't have an
understanding of how to present those experiences in a way that would be
interesting to others.  There is noise and distraction and life will slip by as
it always has in the past.  Any other approach seems to lead to insanity and
suffering.  Insanity and suffering to no end, due to lack of talent, willpower,
relevance, etc.  Okay, done.  The only hope is that I can be like the shitty
rowers who just keep trying and trying and a way that you'd never expect them to
succeed but they kind of have their plan, and eventually it works.  Is that me?
I don't know.  But it kind of \textit{needs} to be me.

The fantasy that you will just work really hard and then it'll be perfect will
not happen.  Your goal should be to get this to a state where it can be
critiqued.

Are you really going to do this?  It's like sitting there before a 2k.  Remember
your true last 2k, the one you did after you got kicked out of the boat, with
L.  Going out at a pace you knew you wouldn't hold.  It would be too
painful.  At 500?  600 to go, knowing it would be too painful.  "It's good you
stopped, and didn't have a shitty piece," said L.  And remember your second last
2k, after you got sick during spring break.  That one was a thing of beauty in a
way.  
\end{document}
