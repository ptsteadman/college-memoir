\documentclass[12pt]{article}

\usepackage[margin=1in]{geometry}
\setlength{\parindent}{1cm}
\linespread{1.3}
\usepackage{indentfirst}
\title{College Memoir}
\author{Patrick Steadman}
\begin{document}
\maketitle

\section{Justification}
I'm done with college now, but I don't think I really understand the past five
years.  I have about a month before I move to New York and begin adult life.
I've decided to spend this time reading my notebooks, thinking about past
events, and trying to understand them.  I'm putting all of the writing on the
Internet, and hope it'll be relevant to someone out there.  People have never
seemed particularly 'into' my artistic or literary work, but I want to believe
that if I stay focused and honest that the writing will be beautiful and
useful.  If not, I will begin my working life knowing that I tried writing, and
that writing is a way for me to communicate with myself and the people in my
life, but is not a viable career option.

\section{Writing Advice}
In the winter before my 20th birthday, things were going better for me.  I had
just returned from working in China, and I was going to school at the University
at Buffalo so that I could return to Cornell in the fall.  One weekend in
February, I flew to New York to go to a reading by Marie Calloway, Tao Lin,
Spencer Madsen, and Megan Boyle.  It was my first time in New York since I'd
flown there to meet her for the first time, when we were Internet boyfriend and
girlfriend.  

After arriving, I took the subway to Wall Street.  I was interested in Finance
as a career, but I was also interested in Occupy Wall Street and Zucotti Park.
Zucotti Park was filled with police, and there were portable watch towers and
instrumentation surrounding its perimeter.  A handful of people were protesting,
or just there.  I admired a statue of a classic-looking banker eating his lunch
while staring into his slim, angular briefcase.  It was fairly cold and clear,
without snow or slush.

I wandered away from Wall Street, trying to find somewhere to charge my phone so
that I could keep trying to get in contact with Marie.  We weren't on terribly
good terms then, for whatever reason.  I was worried that I would be isolated
and alone until my flight back at the end of the long weekend.  I was becoming
slightly more confident, and I didn't want to damage it by feeling ugly,
disillusioned, and left out at the reading.

I spent the rest of the long, dark afternoon and evening wandering around
Williamsburg.  The duffle bag I had was chafing my legs.  Eventually I sort
of gave up on Marie messaging me back, and got a hotel room at the Brooklyn
Sheraton.  I felt calmed, and took a picture of the hotel room and put it on
facebook.  I got on my laptop and wrote an email to the Yale cofounders of a
location-based chat service that I thought was interesting, and felt thrilled
and excited when I recieved a response: yes, they'd be happy to set up a call.
With Marie off my mind, I got ready for bed.  

I checked facebook one last time.  Marie had finally responded to my messages.
She was upset and drunk and wondering where I was.  I felt excited, relieved: I
messaged her that'd I'd come to her if she gave me an address, and went down to
the hotel lobby to get a cab.  It was almost midnight.

The cab stopped at the address, around Union Square.  Marie was supposed to meet
me outside. I was confused.  I got out of the car and saw her.  For a moment, it
was like she was pretending not to see me, or expecting me to come to her.  She
was wearing very short shorts and looked unstable.  I waved her towards the cab.  

She got in and leaned into me.

``Thank God that was so awful, I needed to get away,'' she said.  I felt sort of
validated: I didn't really trust the Internet women friends she was staying
with.  I told the cab driver to take us to the address of the first bar I found
on yelp in Brooklyn, and wondered how he interpreted Marie and I: did he think
she was a sex worker?  

Marie started crying, and I smoothed her hair.  She cried until we got to
Brooklyn.

We went into an empty bar, and Marie was suddenly more cheerful.  She talked
about the people she had met at the readings, and told me that I was special.
After a couple of sweet drinks we left the bar, and went to find greasy food.
While crossing the street, Marie asked me ``Are we going to have sex, Patrick?''
and I said ``I don't know...probably?''

Marie got chicken wings or something and we got a cab back to the Sheraton.  The
driver couldn't find it and I felt calm and detached, trying to give directions.  

When we got back to the hotel, Marie took a shower, and got onto the bed in the
doggy style position.  ``Fuck me,'' she said.  I took off my khakis, button down,
and sweater, and kneeled on the bed with a condom.  I felt ambiguously turned on
due to the alcohol and comfort of the hotel room, and put the condom on.  When I
saw the hair of her vagina and touched my body to her slightly wet skin, trying
to push my way into her, I was surprised to find that I wasn't hard.  I had
never even really considered this happening: when we had had sex the last time,
everything had happened more or less without thought or effort.  I tried to get
myself hard for about thirty seconds, feeling more and more abject, as Marie
looked back at me.  Finally I gave up, hating myself for giving up, and threw
myself down on the bed next to her, hot tears in my eyes.

Marie seemingly ignored me for a while. Then she ran her hand through my hair.
I can't remember what she said, or I said.  Before long we fell asleep,
together.

When we awoke I felt calmed and drained.  The shame of the night before would be
something I'd deal with later, I figured.  Brunch with literary people and the
reading were going to happen that day.  Marie woke up and we talked in bed,
mostly about literature.  She seemed to want to talk about my writing.  "You
should make a chapbook," she said.  I figured she was being sincere.  

``You need to stop worrying about making yourself look good in your writing,'' she
said.

\section{The University at Buffalo}
While I was very involved with the Internet in the spring of 2012, I also
started having more of a ``real life''.  I joined the University at Buffalo's
rowing team/club around March.  I saw a flyer advertised the men's rowing team,
looking for people over six feet tall.  While I was having a pretty good time
posting diaries on tumblr and getting my hom ework done, I wanted to make some
friends to go to parties with at UB, and the  rowing team seemed like a way to do
it.  

I went to the information session advertised on the poster.  I was impressed by
 broad chest of one of the rowers at the information session: I decided I
wanted to look like that.  I put my name down.

I sort of knew how to row, because my dad bought a rowing machine when I was a
kid.  A few days after I started my ``medical leave'', I began to row 5
kilometers more or less every day.  The first time I tried to row five
kilometers, I stopped at about three kilometers in, but then started up again
and finished.  I wrote the times in my Moleskeine planner.  I also used the
Moleskeine to write down the PHP/JavaScript/HTML tutorials I did and the books
I planned to read.  This was basically how I passed the winter of 2010-2011
(and failed my literature classes).

In the spring of 2012, I was taking programming, accounting, economics, and
math classes.  I cried bitterly after my first accounting test.  All of my
other classes were very easy and I got near-perfect grades.

In February I only partially participated in the rowing team.  I showed up to a
few land workouts, where we ran stadiums or did some erging.  I think I
missed about half the practices, just as I missed about half of the cross
country, track, or wrestling practices in high school.  Something changed,
though, in the beginning of March.

I had missed a bunch of practices, but I drove myself to the boathouse (more of
a boat hut) early in the morning.  I was late, so they didn't put me in a boat,
but told me to go for a run.  I had been up all night, up all the last week on
the computer.  I ran behind one of the other rowers, barely keeping up.  I
described the pain to myself in words: I felt my shoulder muscles heat up and
become loose, my chest become a golden cage, twinging.  Sweat dried onto my face
and into my eyes.  When we finished, I felt good.  I told the coach that I could
go on the training trip during spring break.  Then I got in my car and wrote a
tumblr entry describing how I wanted to go to Sports College: I needed a coach
to tell me when to eat, when to sleep, when to work.  I needed to learn to
control my fatigue and pain.  I could take care of my own mind, I thought.





\section{Un-Quitting}
Bitterly lonely when I went back.

\section{Casual Sex}

\section{Captain of the Crew Team}

\section{Reticule}
During my first fall on the Cornell rowing team, I had a constant sense that
all of my teammates were calling me ``faggot'' or ``pussy'' behind my back.  I
imagined that if I showed my ``true self'' their friendliness would turn into
open hate, like when you kill allied soldiers in a video game, and they start
attacking you and your crosshairs go from green to red.  


\section{First Story}
I am so sick of thinking and writing about the first few months of school.  I
could write about my hopes, disappointments, time in the library, experiences
going to parties, or meeting my friend George, who I realize will be like family
to me when I move to New York.  Most of what happened my first semester can
probably be inferred by someone who has gone to a similar school.  I failed; I
don't like losing, I don't want to dwell on the details.  I tried to write
papers and failed.  I tried to make friends and failed.  I learned but I failed
to make anything of it.  I didn't ask for help.

I went home to Buffalo, to live with my parents.  I felt so relieved.  I could
stay in my room and read and go on the Internet.  I was sad, but relieved.  I
was upset that I was causing my parents pain

\section{Old Messages}
While reading through the old messages of friends and lovers, I
sometimes ended up reading conversations from my own past.  I usually
felt the familiar dread and pain that came  with seeing myself as others
saw me.  None of the messages were old enough to allow for the
possibility that I'd changed since sending them.  A person with more
willpower might have used these messages as motivation to
change himself.  A more humble person might have realized that she
should be kinder to others.

\section{Marissa}
It was surprising that within a month of moving to New York, I was
in a relationship, sleeping with someone that really lit up my life.

Marissa was my first adult relationship, I think.  It was more than a
little surreal: us, on Christmas eve, dancing in front of our small
Christmas tree, staring into each other's eyes, only half drunk.  The
amount of comfort and pleasure we managed to enjoy during any given weekend
never ceased to surprise me.  We had perfectly
achieved our ``goals'', and only dimly remembered our dreams.

Marissa was a banker, but she made being a banker look easy.  She didn't really
buy into the whole ``insane work hours'' thing.  She was effective and composed,
and had a moderate, genuine interest in the financial services industry.  Also,
she was attractive.  Blonde.  She looked like a female banker. 

The only thing that hinted at her past was the fact that she wore too
much perfume.  I think I have a thing for women --- (I always make a point
to call the ``girls'' of our generation ``women'') --- I think I have a thing
for women who wear too much perfume.

\section{Rowing Machine}
The rowing machine was the only sacred object in my apartment.  It
was situated so that I could see the Manhattan skyline while I
rowed, and in the late fall and winter my heart was moved by the slow
pulsing of the warning lights across the river.

My parents were confused by the young adult I became in college, and
when I walked on to the rowing team, they had clung to my identity
as a rower as one of the few positive facts they knew about me.  As
a result, almost all of my Christmas gifts during those years were
rowing related.  These gifts hung on the wall next to the rowing
machine: a hand drawn poster from the 1920's commemorating Ivy
League crews, parts of antique rowing shells, and, ``embarassingly'',
pictures of me racing at Eastern Sprints and the IRAs.  When people
made fun of my shrine, I reminded myself that these items were proof
that my parents loved me more than any of them.

\section{Hadoop}
When I arrived at Cornell for my sophomore year, I was determined to set
myself up for a prestigious internship.  In early September I put on my
blazer and went to the career fair.  I spent most of the day in the
hot indoor track building, talking to employers. 

I found that I enjoyed talking to the tech companies more than the
banks.  It was always very difficult to discern what exactly I was
talking about when I was talking to the financial services companies.
With the tech companies, I was able to stretch my superficial knowledge
of programming in order to have interesting discussions with the
recruiters.  Many of them seemed to want to believe that they were doing
something really special, technology-wise.  A fat neckbeard representing
an ad tech company made a lasting impression on me by asking if I knew
what Hadoop was.  

``No, what is it?'' I asked.

``Oh, man.  The next big thing.  The current big thing,'' he said,
rocking back and forth.

``Like, what is it though?''

``It's Apache's distributed MapReduce framework.  It's how we can process
millions of ad slots a minute.  You should really look it up.''

I did google it, and felt the familiar sense of wondering what was
really going on: was this, and NoSQL, or NodeJS, just significant
buzzwords, used like the names of bands in high school, or the different
sectors at a bank?   Over a year later, I would smile when my
Database Systems class required us to implement the PageRank algorithm
using the Hadoop framework.  Computer Science had turned out to be more
than emptiness, for me.

And a year after ~that~, one of the men indirectly responsible for the
creation of Hadoop spoke in my Information Retrieval class.  I was
fascinated, because I had already watched a video of him talk about the
role of 9/11 in Google's history, and had developed a pet theory: that
the attack on the Twin Towers had led to the creation of the Map Reduce
programming paradigm, and in turn the Hadoop framework that enabled many
of the ``big data'' applications of the later 2000s.

Amit Singhal, director of search quality at Google, was away at a
conference on September 11th, 2001. As the public searched for news
about the attacks on the Twin Towers and the Pentagon, Amit and his
colleagues realized that Google was dramatically failing to meet the nation's
information need. Searches for ``World Trade Center'' led to web pages
detailing the architecture of the now-destroyed buildings, or real
estate listings.  This was due to the fact that Google was only able to
index the internet about once a month: the index used to fulfill
searches did not reflect the current, dramatically different reality.
Over a conference call, Amit and the Google engineers decided on a hacky
solution: they simply added links to relevant news articles on Google's
homepage.  This didn't work: the massive amounts of traffic directed to
these articles caused the news network's servers to crash almost
instantly.

Amit and Krishna Bararat, a search architecture engineer, were trapped
at the conference center in upstate New York until planes were allowed
to fly again.  Over the next few days, they sketched out the
architecture of what would become Google News, a system that would index
news websites constantly, ensuring that Google would be able to provide
information about events that had just happened.  Building this system
would require rethinking Google's entire data pipeline.  New programming
models for distributed systems would need to be perfected in order to
have enough computing power to simulataneously index thousands of news
websites.

Over the next few years, Google News was developed, but Amit and Google
realizes that just about everything needed to be indexed in ``real time''.
In order to index the whole internet, every day, the programming methods
used to create Google News would have to be formalized.  In addition,
Google realized that the rest of the internet had to catch up with
Google.  To address both of these concerns, in 2003 Krishna released
``the MapReduce paper'', which detailed the abstractions used by Google to
think about their complicated distributed systems.  

Something about the nature of MapReduce always felt very current and
zeitgeisty to me, perhaps because of the adtech neckbeard at career
fair, or perhaps because the MapReduce paradigm of millions of isolated
units of data being accumulated, fragmented by a inscrutable hash
function, and reduced to useful key-value pairs reminded me of the way
real life was starting to work.

\end{document}
