\section{}

The seaside fortress was owned by the slaves.  It was given to them at the end
of the rebellion, to ensure they would not suffer so terribly again.  Only the
leaders of the rebellion were allowed refuge in the fortress; the rest of the
slaves on the island would remain slaves. 

I was born in this fortress, but my only memory of it is a memory of playing on
a balcony and smelling the ocean.  My parents brought me on their honeymoon, and
we never returned to the fortress.  Strangely, they decided to visit the islands
of Hawaii.  During the honeymoon, my parents decided to emigrate to the United
States.  My parents are both educated, and their application for refugee status
was accepted, and our family eventually moved to upstate New York.

Good things happened in my childhood.  In my early 20's, I spent some time
reading the 4chan board, /r9k.  I was chilled when someone on the board said it
was a community for people "who had never had anything good happen in their
lives".  

Nothing good had ever happened in their lives.

That wasn't me.  In 6th grade, I was accepted into a highly selective Math
program.  The application said it was for ``the top of the top percentile''.  I
took the application test in 5th grade, and was not accepted.  I sobbed.  The
next year, I studied using SAT prep books, and was accepted.  They called me to
the principal's office to receive the news.  It was the happiest moment of my
life.  I knew that being in this program would give me an advantage and shelter
me from the full competition of the brutal world.

Nothing really good happened during High School, but I wasn't too worried about
it.  I felt very special.  I did some special things.  I dated someone from the
math program.

At some point during the summer before I started college, my mother went to
Target and bought me things for my dorm room.  She put the shopping bags into a
large laundry basket and left it outside my door.  The morning before we were
supposed to drive to Ithaca, she noticed that I hadn't packed the basket into
the car.  I tried to explain why I didn't want to bring all the stuff, and she
got increasingly upset.

The night before, I had set my Cornell netID password to ``10Nianhan'', a
Chinese idiom that translated to ``Ten Years by the Cold Window'', a period of
transformative, ascetic study.  I thought having posters and an ironing board
would distract me, or give people the wrong idea about my personality.  I didn't
really know how to communicate how I felt to my mom.

I was tired from spending the night reading Craigslist personals from the Ithaca
area.  I found someone looking for a ``literary correspondence'', and I paced
around the kitchen feeling excited and manic.  I took a picture of myself with
my childhood stuffed animal and sent it to her.  She also wanted respondents to
put a writer in the email's subject.

I chose James Agee.  One time, I wrote in the email, Agee and his wife were
having a fight at his office.  His wife stormed out of the room, leaving Agee
alone.  A minute later she ran back into the office and found the window open.
She screamed and pulled out tufts of her hair, leaning out of the window to look
for Agee's body on the street below.  A few moments later, Agee stepped out from
behind the door where he had been hiding.

I eventually agreed to bring some of the stuff from the laundry basket, because
I didn't want to fuck up the whole day.  I finished packing, took a shower, and
put on a red polo shirt from American Eagle.  My parents drove me to Ithaca.  It
was a beautiful, clear, cool day.  

\section{}

Moving in, I saw a pretty Indian girl with long hair laughing in the sun.  I
felt a stab of anxiety, wondering if my dreams would be realized my over the
next four years.  It was time, I told myself.  I couldn't let things stay the
same.

My parents and sisters helped me carry stuff up to my room, on the fourth floor
of one of the brick dorms that had been built in the seventies or eighties.
Both of my roommates had already moved in, but weren't present.  Over email, I
had volunteered to take the top bunk.

\section{}

I went to get registered at Barton Hall, and then bought textbooks at the
Cornell Store.  I bought all of the optional books for my Ulysses class, and my
dad smiled as I talked excitedly about the class.  He insisted that I didn't pay
for the first semester's books.

My mom and sisters stayed the the Cornell Store picking out t-shirts while my
dad and I walked to the dining hall for dinner.  It was obvious that he was
going to lecture me.  

I knew that my dad had struggled at Cornell, to some degree.  He had taken a few
semesters off to work as the chef for his fraternity, and then met my mom at
some point.  They were both on the sailing team.

We got chocolate milk from the steel milk udders, and sat down at a table.
Daddy took a napkin and drew a bell curve on it.

``You know what this is, right?'' he asked, seeming a little embarrassed.

``A bell curve?'' I said.  I was amused.

``Yeah!  A bell curve.  I've been thinking a lot, what I should say to you.  I
guess college is pretty long and exhausting, and you've got to be ready for that.''  

``What does a bell curve have to do with it?''

``Well, I guess you're used to being around \textit{here} on the bell curve,
right?'' said Daddy, indicating somewhere in the top 10\%.  ``Or at least that's
where I was used to being, I think.''  

``Yeah, I guess,'' I said.  I didn't like where this lecture was going.

``Well, at Cornell, you might have to get used to being around here on the
curve, or even here.  Everyone at Cornell comes from the top end of the curve.
You have to learn to be okay with being down here on the curve sometimes.''

This is cliched stuff, I thought.  I felt a little angry and stressed.  I'm not
like you, Daddy, I thought.  You don't understand.  I'm not even really
\textit{on} the bell curve, in some way.  I don't care about the bell curve.
I'm sorry you did.

``Sure,'' I said.  ``I get it.  I've heard this stuff before.''

``As long as you go to class, you should be okay.  You have to go to class.  You
just have to.''

``Look, remember, I calculated the cost per hour of college during that boring
orientation thing earlier?  \$7.70 an hour, even when I'm asleep.  I'm sure the
cost of class hours is, like, insane.  I'm going to make the best use of my time
and go to class. Don't worry.''

``And make sure you go to the orientation stuff.  Mommy and I were talking about
this.  We're worried that you're not going to go to these events that are set up
for you.  Just go to them.''

``Okay, I will.''

I felt annoyed.  The summer sunset filled the dining hall with an intense orange
and red light.  I shielded my face as I made my way to the soft serve machine to
make a Sunday.

\section{}

I went through all of the orientation stuff.  I had no interest in spending
time with the group of Communications majors I was assigned to, so I drifted
between groups. 

Very suddenly around 10pm at night, we were set free.  We were all sitting in
groups on a large, dark lawn somewhere between the dorms on North Campus.  No
one was expecting me at home, I realized.  I could go where I wanted, and
explore as long as I wanted.  It reminded me of an large and detailed level in a
video game becoming open and available.

\section{}

I found Finn, a friend from high school, and started following him to
Collegetown.  A lot of other people were following Finn too.  It seemed to me
that Finn didn't really know where he was going, but he managed to chat with a
girl who was going to a party and just sort of co-opted that information.  

Finn used me as a way to create interesting conversation for the group.
``Steadman's crazy.  Tell them about the time you hitchhiked to Montana.  I
drove this fucking kid to the highway, and he just started hitchhiking there.''

``Wait whhat?'' someone said.  I nodded and smiled, making a face at Finn.  I
was secretly pleased to get some attention, but also a little uneasy.  I
answered some typical questions about running away from home and hitchhiking.
This happened a few times, and eventually we all started talking about what we
were going to study.  I was quiet. 

We passed through campus, occassionally seeing and hearing other students
walking in the same direction.  We crossed a stone bridge over a dark gorge.
Collegetown was full of loud students, drinking on the lawns of houses or
standing in the entrances to cafes and bars.

Finn's party was at a Jewish fraternity's annex.  It was crowded.  Finn and I
pushed our way in, and Finn got beers for the remaining people in our group.  We
stood by the beer pong game.  Finn was grinning.

Occassionaly we would talk with a girl, shouting, who always turned out to be
another freshmen.  I would feel optimistic about each girl we talked to, for a
minute or two, but then would worry that I had nothing to say and that the girl
didn't seem very interesting.  But Finn seemed happy and excited.  ``These are
going to be a great four years, man,'' said Finn.  ``We're going to kill it
here.''

\section{}

When I woke the next morning, I was sort of sad that I had walked home from the
party alone.  I checked my email, and saw that the Craigslist person had
responded: \\

\textit{
So I looked you up on the cornell website. are you a freshman? unfortunately, i
am much too old for you. go hit on some girls in your dorm or something. you've
been here for, what, less than a week?}

\textit{some sage advice from your new friend,}

\textit{alice munro}\\

I felt moderately annoyed, but it was another beautiful day, and I went to the
dining hall for breakfast, carrying a book.  

\section{}

I sat in the entrance to Goldwin-Smith hall with my copy of Ulysses, my
notebook, and annotations for Ulysses.  The sun was setting over of the arts
quad and the stage set up for the end-of-orientation week DJ. I continued
reading,  but I enjoyed the presence of the other students and the music
playing over the PA system.

Every line of Ulysses seemed to contain references to information that would
enable to become the sort of human being I wanted to be.  

I decided to join the crowd on the quad, and put my books into my backpack.  As
I walked down the marble stairs from Goldwin-Smith, I was stopped by two girls.
It was immediately obvious that something was up.

``Hey, can you take us to a bathroom,'' said a short tan Asian girl, probably
Korean.  I looked at her and her friend.  The Asian girl was sort of overweight.

I opened my mouth but for a second nothing came out; I had been in the library
all day.

``Sure, yeah,'' I said, making myself smile.  The other girl's hair was dyed in
a white streak down the center, making her look like a skunk.  I thought she
was cute.

I led the pair through the crowd.  The Asian girl walked next to me, bumping
her hips into mine. ``So you guys are freshmen?'' I said.  

``Noo,'' said the Asian girl.  ``We go to Ithaca college.  We're sophomores.''

``Oh, cool,`` I said.   I took them into the art library, and we found a
bathroom in the basement.  They went in, and I could clearly hear their
conversation.

``Oh my god, you always go for the nerdy ones!'' said skunk girl.

``He's cutee though,'' said the Asian. 

I waited in the hallway, which was completely white, like most art building
hallways were.

When they came out of the bathroom, I followed them outside, and we worked our
way into the crowd.  The DJs were on the stage.  They had something to do with
Mario and video game music.  The group that did ``Like a G6'' played next.  The
skunk girl had a plastic water bottle of vodka, half full.  She shared it with
the Asian girl but not with me.

The Asian girl stood in front of me and then started grinding her ass into my
thighs.  I thought back to stuff I'd read online about how the best degree at
Ithaca College was a Mrs. from Cornell.  It felt fucked up.  She yelled things
into my ear but I didn't really understand anything she said.  ``Do you want to
hook up with me?'' she yelled.  

I felt the cold air that sometimes is felt in a hot crowd like this.  I
remembered the first time I had gone to a live concert, when I was 17, and the
sight of Spoon taking the stage and everyone screaming caused the sweat to feel
cold on my back.

I thought for a second.

``Uh not really,'' I said.  ``What?'' she yelled.  ``No,'' I said louder.

``Then why are you so \textit{hard},'' she said into my ear, running her hand
over my dick.  I felt sort of drunk even though I hadn't had anything to drink.
I looked at the skunk girl, dancing with her eyes closed.  

``Like a G6'' came on.  I danced with the Asian girl.  She was quiet.

Eventually I got to the point when I knew nothing would be gained by staying
with the two girls.  I said goodbye, laughing, and walked back to my dorm,
feeling bemused and happy. 

Over the next five years I often thought back on this event and wondered why I
didn't try to have sex with the Asian girl.  

\section{}

At some point during the first few week of classes, I met George.  Paola set up
a meeting at her suite for me to meet some of her friends from Clarence High
School.  Paola and I had met in 7th grade when we were accepted into a
Buffalo-wide ``Gifted Math Program''.  A few days into the program, the other
guys and I decided that Paola and one other girl were the only two hot girls in
the program.  Four years later, during Junior year of high school, Paola and I
dated for about a month.  It was painfully awkward, and we broke up shortly
after the first time we kissed.  Paola often told me that I would get along with
her friend George, but I never had a chance to meet him. 

I went over to Paola's suite and uneasily chatted with her and her
roommates, waiting for George to show up.  Paola had prepared bowls of snacks.

When George showed up, we talked about music and writers.  Despite how reserved
George was, and how in many ways he seemed to be a ``typical Asian'', I quickly
felt that our fates were aligned and we became friends.  We made plans to play
music together on the weekend.

\section{}

I was disappointed with the people in my dorm.  I tried to hang around in the
common room, hoping I'd meet someone interesting, but most of the guys were
either Engineers, really into athletics, or painfully normal.  The Engineers
were interested in solving Rubik's cubes or talking about Pokemon.  The athletic
guys had decided to join the Judo club together.  One time I tried doing an ``ab
ripper X'' video with them but I couldn't get through it, to my surprise.  It
seemed the the girls were not very interested in me.

Orientation week passed.  I messaged George and we met on the balcony outside
the common room.  We started to do this every few evenings.  He would usually
listen to me play guitar, and then we would talk about music or listen to it on
my laptop.  I wasn't very good at guitar or singing.  I could only learn one or
two songs at a time, and had a bad sense of rhythm.  George had studied piano as
a kid, but when I tried to get him to play at the piano on the top floor of the
dorm he would silently attempt to play a piece of classical music for about
fifteen minutes, and then suddenly give up.  

\section{}

Before classes started, I took it upon myself to re-read the Odyssey before I
started reading Ulysses.  It seemed like my only chance at producing an original
understanding of the text was via rigorously exploring of all of the text's
dependencies, all of the things that it referenced or was built upon. 

My experience with the required reading book for incoming freshman had unsettled
me.  The assigned book was ``Do Androids Dream of Electric Sheep?''  There was
an essay contest and a reading group.  I put many hours into my essay, waking up
early to finish it.  I  had to run through the rain to turn it in, late to the
group discussion.  I felt that I had real feeling for the book and the movie
``Bladerunner'', I loved the aesthetic of 80s urban cyberpunk and the idea of a
city where it always rained, like Taipei, where I had spent my senior year of
high school.  The rich girl that I had been obsessed with in Taipei had told me
that I looked sort of like Harrison Ford, and I identified with Rick Deckard's
technology-oriented, unemotional Noblesse Oblige. 

Listening to my peers talk about the book, I felt more and more confused and
defeated.  In the back of my head, I knew that the essay wasn't very good, and I
was tired.  I tried to say insightful, interesting things about the book, as
concisely as possible, but it seemed pointless.

The night before my first day of classes, I posted an excerpt from the Odyssey
on Facebook: ``I will leap out of bed, sling a sharp sword over my shoulder,
strap a stout pair of boots onto my lissom feet, and go forth from my chamber
like a young god to face 8am calc.''

\section{}

The Ulysses class met in one of the classic seminar rooms in Goldwin-Smith.  I
was the only freshman in the class.  Many of the other students were attractive
or seemed grown-up.  The upperclassmen girls had golden jewlery and beautiful
sweaters.  I was annoyed by a handsome Irish student, but interested in a
crazy-looking woman with a hunchback.  The classroom had a single large window
overlooking the arts quad.  The light in the room was bracing and fresh, very
alive due to the moving shadows of leaves and branches.  The arts quad was empty
and covered in dew, it was 8:40 AM.

I was pleased with the way the professor talked.  He was short and Jewish.  He
talked for most of the class about how the embellished typography of the
frontispece and postscript of the first edition of the text suggested
interesting things about the father-son relationship between Stephen and
Leopold, suggestions that were not available to most modern readers.

After class, I wandered around the crowded plaza in front of Willard Straight.
Smiling students came up to me and asked me to join their clubs.  I had been
considering joining the sailing team or auditioning for an ``A Cappella'' group.
It seemed necessary, to ``get involved'', so that I'd know a lot of people and
make the friend group that I'd always wanted in middle and high school.  

Shortly after starting 5th grade, I became pretty obsessed with trying to become
part of the popular group.  One of the happiest moments of middle school
occurred in the first week of classes, when I overheard a guy saying to another
guy that I was ``pretty chill''. 

After getting sashimi for lunch, I went to my Advanced-Intermediate Chinese
course.   I'd only gotten into the course because I'd spoken to the professor
one morning at the dining hall, after overhearing her conversing with some
former students in Mandarin.  After talking with me, she gave me the code to
sign up for the course online.  Working up the courage to talk to her had been
difficult, and I was proud of myself.  It felt like a good sign. Based on the
first class, it seemed like my Chinese ability was significantly better than the
other students in the class.

\section{}

I liked to talk to George about my schemes for college.  It seemed like George
had a quiet interest in my more weird or risky ideas.  For example, I wanted to
build a ``secret'' room somewhere on Cornell's campus.  This was motivated by my
interest in ``urban exploring'', something I had become interested thanks to
StumbleUpon and the breathless Internet of the late '00s.  I dreamed of
exploring Cornell's tunnels and rooftops and finding place to build a secret
room where people could meet, like I had done in the basement of my high school.
I imagined it would be a grimy version of the ``secret'' rooms that exclusive
societies like Quill and Dagger had. 

George helped me look for a suitable place to build this room.  George watched
my back while I opened hatches that looked promising.  At night during the
middle of September, the Cornell campus proper was empty and green.  Most students
were walking through the dark streets above the gorges, on their way to
fraternity houses.

I was impressed by George's bravery and calm.  Most of the hatches led into
dark, tiny spaces filled with loud machinery.  I promised him that we'd
eventually find a good place; that it was like hitchhiking: you always got a
ride at the very moment you least expected it.

\section{}

I was studying in an unused grad student carell in the basement of the Kroch
rare book library.  I needed a windowless, fluorescent and silent environment to
focus.  I wanted to do school right, finally, in college.  In high school, my
academics had always been a big mess.  I did things at the last minute, and
barely studied or did homework properly.  I knew that this would eventually
become problematic, and I knew that college was my chance to make a change and
get it right. 


I had a daily planner, given to me by most Taiwanese host mother.  It was a
reward for buying ten boxes of donuts at Mister Donut.  I loved the planner, it
had pictures of cartoon loins with donuts as manes. I felt I had learned a lot
about study discipline from my host family and classmates.  Halfway through my
year in Taipei, I finally decided to stop spending so much time on the Internet
and get serious about learning Mandarin and making friends with my classmates.
I'd become very disgusted with myself, falling asleep in class and then waking
up with my face stuck to the desk, hours later.  I felt a deep, petrifying sense
of shame each time my host father noticed that I forgot to turn off my air
conditioning and yelled at me in a language that I barely understood.  I started
studying Chinese obsessively, and tried to talk to my classmates, even if I had
to drink to deal with my embarassment.  At 5pm when regular classes ended, I
started following them to their cram school classes.  I loved my new routine and
the sense of exhausted happiness I felt, talking with my classmates in the
\textit{noir}-ish noodle restaurants near the cram school street. 

My most important class besides the Ulysses class was Calculus.  I made myself
work through the preamble chapters of the textbook, which reviewed the more
difficult concepts of algebra.  In general, my Math skills were weak.  The early
years of the ``gifted math program'' had focused on set theory, formal logic,
and number theory at the exclusion of factoring and other more foundational
topics.  When the gifted math program eventually got around to these subjects, I
was in the middle of my most distracted and rebellious stage of high school.
I'd managed to get a 740 on the Math section of the SAT thanks to review books
and using Reading section time to work on math problems that I could just barely
see through the thin newsprint of the test booklet. 


I wrote out all the pre-Calculus chapters I'd need to review, and then the
assigned readings for the first two weeks of classes.  The exercises were hard,
and I tried to suppress the thought that no matter how many exercises I did, I
still wouldn't be able to make the conceptual leaps necessary to solve the more
difficult problems.  It was hard to not fall asleep in the warm, dry air of the
rare books library.

\section{}

On Friday, Seth texted me, asking if I wanted to get lunch with him at
Okensheilds.  Seth was a ``nerdy'' friend from high school, who was planning to
be a doctor.  I was surprised when I found Seth sitting with a bunch of
good-looking, articulate people.

They had met during orientation, playing ultimate frisbee.  I was instantly
attracted to Morgan, who was tall, brunette, and classically good looking in a
somewhat masculine way.  She was from Virginia. 

George and I were both frustrated with the social lives afforded by our dorms
and frat parties.  Saturday night was misty, and neither of us was interested in
going urban exploring.  I still felt restless, though, and asked George if he
wanted to check out Tellurdide House.  I explain what Telluride House was,
showing him its website: it was an ```intellectual community'' that provided
free housing.  I told George it was most famous for hosting a group of
neo-conservative thinkers like Allan Bloom, Francis Fukuyama, and Paul Wolfowitz
back in the early 1960's.  The present-day branding was much more progressive.

We met up later and walked to the house.  It seemed very quiet and empty, next
to the DKE and DU fraternity houses.  It took a fair amount of courage to simply
knock on the door as planned.  When someone answered, I told them that I'd done
the Telluride summer program, and was interested in living in the house.  The
summer program was free and quite prestigious, with a 5\% acceptance rate.

George and I watched ``The Silence of the Lambs'' with some of the other house
members, and I talked to a mutual friend I recognized from facebook.  They told
me to apply to the house in October.  George and I left, and started walking
back to North Campus.  On the way back, two fraternity brothers called us over
to a tent they had set up for shelter from the rain, and got us to join their
mailing list for parties.  Overall, it was a sort of underwhelming night, but I
was proud of myself for doing what I needed to do.


The next week, I made a point to eat lunch at the same time and place, so I
could see her again.  Okensheild's was an old, faux-Gothic meal swipe dining
hall at the center of campus, that most people seemed to avoid.   I was happy to
see Morgan was there, with Seth, and David.

Talking with Morgan was still good.  I talked with her about the stuff I was
researching for my Ulysses class, and I was surprised that she was interested in
some of the weirder aspects of religion, like the heretics of Catholicism and
Gematria.  She also liked music: she made we watch a Nicki Manaj video, which I
was immediately obsessed with.  ``She's the best female rapper since Lil' Kim,''
said Morgan.

\section{}

I decided to stay in Ithaca for the Labor Day long weekend.  I went for a jog on
the gravel path around Beebee lake.  Strange birds stood in the water, and I
stopped to look at a group of turtles.  I walked back to my dorm, feeling a
sense of well-being, depsite the confusion and stress of the first two weeks of
classes.  Mommy called me on my silver Motorola RAZR flip phone.  It was my
dad's old cell phone, I didn't use it much.  I told her things were going pretty
well.  My mom was shopping at the grocery store, and seemed relieved and happy.
I made it back to my room, and lay on my bed.  Ithaca was starting to feel like
home, I decided.  

\section{}

The next week of school went fast.  It rained.  On Saturday afternoon, I tried
to convince George to go out with me.  A bunch of people from my floor were
trying to go to the S.A.E. highlighter party, and I figured George and I could
go on our own.

``Well, it'll either be intersting and worth going to and we'll meet people,'' I
said to George, ``or it will be awful and embarassing, and we'll feel superior
and learn a lesson about how things, like, don't work.''  I wanted him to be
enthusiastic about going out, and trust that I'd figure out a way to make it
fun.  He eventually agreed to go to the party.  

George had a new Android phone with Google Maps, and looked up the address.
When we arrived, there was a long line leading up to the house.  The house was
set far back from the single-lane road, and provided the only light in the area.
I saw my floormates somewhere ahead of us in line.   It hurt my pride to have to
wait like everyone else, and watch drunk people making friends or talking.
George and I talked about his rhetoric class and the strange the period right
after 9/11, until we got close to the front the line.
 
The fraternity brothers at entrance to the house yelled at everyone.  George and
I were silent.  The brothers kept letting people in ahead of us.  One guy yelled
``get...the fuck...back'' over and over again, and his friend mutterred
``fucking faggots'' under his breath.  I loathed the overly serious faces of the
other students waiting in line with me, pushing towards the door.

I wondered why we weren't getting let in.  I didn't have a group of girls with
me, but that didn't really seem to be the problem.  I compared myself to the
fraternity brothers.  One of the guys was almost overweight.  They weren't
better-looking than me, really, but they were louder, and wearing tank tops.  I
knew that I didn't really look very impressive.  I was wearing a white polo
shirt, what I considered ``camouflage''.  I walked with a bit of a limp, because
my ingrown toenail was becoming a problem again.  It smelled bad and I had to
clean crusts of blood and pus out from between the toenail and flesh almost
every day.

Finally, someone from my floor yelled my name from inside the house, and the
brothers waved George and I in.  I followed my floor mates up the crowded stairs
into a blacklit room.  I did a few shots with the group, smiling.  After a
while, George and I sat on a couch, watching the people from my floor joking
around with the brother who lived in the room.  It was too loud to hear what
they were saying.  I felt tired.

\section{}

The next day, George and I met up to study in the Graduate English Lounge, which
I had told him about.  I had given up on the underground Kroch library.  The
English Grad Lounge  had a curated collection of ``dry'' literary books and a
view of the clocktower.  George worked on his readings for his rhetoric class,
while I worked on Calculus and Ulysses.  I didn't think the readings for my
Communications class were really worth doing, at this point.

As it got dark, we gradually stopped working on school work and started showing
each other things on the Internet.  I showed him trailers for some of my
favorite Chinese movies.  George told me that he couldn't read Chinese
characters, but wrote his Chinese name out for me on a scrap of paper.  George
showed me some of the blogs where he read about music, and we talked about how
having ``good music taste'' was now just a matter of reading the right blogs.  

Before we walked back to North Campus, I showed him Hipster Runoff, my favorite
website.  I made him read some of the ``classic'' articles, like ``U, me, and
every relevant concert we attend'' and ``My job/career does not align with my
true personal brand''.  I watched his face.  I felt that these articles touched
on an aspect of the present day that no one else seemed to be addressing.  I
felt that Hipster Runoff was one of the few things that gave me a really unique
perspective on things.  Because of Hipster Runoff, I didn't feel as ``caught
up'' in culture as I once had.   George seemed to ``get it''.   We continued
talking about Hipster Runoff on the walk home, and I felt happy.

excerpt?
\section{}

My first essay was due in class on September 20th, Monday.  It was a take home
exam for my Communications class.  I often fell asleep in Communications,
because it was right after Calculus, early in the morning.  The professor's
self-satisfied tone bothered me, and the content was obvious.  I started buying
myself a Starbucks frappachino and an oatmeal raisin cookie to enjoy before I
involuntarily fell asleep.

On Friday, I realized I had the weekend to finish the essay.  I wouldn't be able
to go out. I went to bed early.

I woke around 11am on Saturday morning, and felt a sense of luxury at the
thought of all of the textbook reading I had to do.  I walked uphill towards the
new Appel dining hall, located in front of a large soccer field at the edge of
campus.

Breakfast was my favorite meal.  I enjoyed having a glass of Mountain Dew with
my cereal, and eating warm muffins and pastries.  As I was waiting in line for
french fries, Finn grabbed my shoulder and said hi.  It felt weird talking to
Finn suddenly.

``Steadmaan, how's it going?''

``Alright,'' I said.

``Come upstairs and eat with us.'' 

I joined Finn and a large group of people he knew.  They had gone out together
the night before.  Most of the people were the very forgettable type of
Cornellian I had started to expect: a sort of heavy, dirty blonde, somewhat
athletic male or female from the tri-state area, wearing a t-shirt from their
former high school or athletic team.

I was instantly interested in Mira, though.  Finn was obviously interested in
her too, and joked with her like an old friend.  She was wearing a black dress
and a leather jacket.  Finn told me that Mira was an architect student, from
Romania.  I asked her what she thought of brutalist architecture and Cornell.
Ian continued making jokes involving sex, physics, or literature. 

In these situations, I always felt that the type of person I was fated to be
with would understand my reserve and silence as a sign of character.  I imagined
Mira going back to her dorm room and thinking of my calmly amused face, my
reaction to Finn's humor.  

Finn invited me to go out with him and Mira that evening.  As an afterthought,
he invited everyone else at the table, who had been talking amongst themselves
the whole time.  I told Finn that I had to work on my essay, but I'd text him.

\section{}

I started reading the Communications textbook from the begninning.  I'd bought
the eBook version because it was cheaper, even though my dad had insisted that I
not worry about the price and get the paper version.  I looked at the syllabus
and made a list of all the chapters I had to read.  

The book started out with an overview of pyschology, sociology and philosophy.
I started taking notes, skeptically.  The stuff in class was all very obvious,
disappointingly obvious.  Obviously the Internet was changing how people
intereacted.  Obviously people perceived their friends differently now that
everyone was on social networks.  The stuff in the textbook was more
interesting, because the textbook referenced the names and theories of past
thinkers, names and ideas that might be useful to me.  I started writing them
down. 

I'm like clay that has been sitting in the sun, I thought.  I was 18.  I had a
bit of a dry, pliable shell due to certain life experiences.  I thought back to
running away from home when I was 16, and my crush had rejecting me in middle
school, saying she wouldn't date me if I was ``the last man on earth''.  But my
core was pliable, I told myself, and reading the right things would shape it.
The hard crust  would allow me to hold shape for a while, and do things like
accomplish my goals or finish essays.

I started fitting two lines of text into each ruled line of my
notebook.  I didn't want the Communications notes to take up all the space that
I needed for Ulysses.  

I was excited when I read about the theory of the ``Fundamental Attribution
Error'': the tendency of people to attribute failure to personality or character
instead of ``external factors'' in life. 

I remembered commuting home from school when I was an exchange student in
Taipei.  One day, I noticed someone else from my high school.  She was staring
at a Patrick Star keychain, made from colored plastic circles melted together
with an iron.  She stared at the keychain for almost twenty minutes.  She was a
little shorter and chubbier than the average Taiwanese high school student, and
was wearing the same purple uniform that I was wearing.

I wondered if she was experiencing intense, painful understanding of her
situation as a citizen of the second world nation of Taiwan, in between the
glossy opportunity of the first world and the hope and misery of the third
world.  I imagined that she might have a noble personality, but simply couldn't
do anything about it, and therefore had to attach significance to keychains of
Patrick Star, the stupid starfish.  I thought about the the things that I
thought were important, my own personal history, and wondered if they might be
equally void.

As the sun set, I had finished two chapters, and taken a few pages of notes.  I
texted Finn, and walked back to my dorm to shower.  I felt optimistic, thinking
over the stuff I had read.  Maybe Communications would be an okay major.  I felt
like I was finally doing college right, and that going out with Finn again might
give me the chance to hook up with a girl.

\section{}

We went to a party at my dad's fraternity, Rockledge.  

``Didn't your dad build these tables and stuff?'' yelled Finn, referring to the
wooden booth we were sitting in, drinking beer.  It was loud and dark.  Mira sat by
Ian's side.  

``Yeah.  I think he spent like a whole semester on them.''

I was sort of drunk and danced with a few girls, one who was attractive.  I felt
hopeful, but each time the song would end and one of us would drift away for
another drink, or smile and step away, pushing through the crowded basement.  

What is the force that keeps people apart, I wondered, thinking back to my
readings.  Socio-psychological vs. socio-cultural factors. 

Finn grabbed me, and pulled me upstairs.  We sat on the balcony.  I felt dazed
and calmed by the sound of water falling hundreds of feet below in the gorge and
rich young voices yelling to each other in the dark.  Finn and I stood up, and
leaned over the wooden railing, not meeting each others eyes as we talked.

Later, Finn led us through west campus to another party.  Mira was with us
again.  As we walked I talked with her about Romania.  During a manic weird time
in Taiwan I had watched a video of them being shot against a wall.  I remembered
that there was a lot of yellow in that video.  

``Ceausescu,'' said Mira.  ``They recently dug up his body to do DNA tests.''  

``That's interesting,'' I said.  ``It's weird that we were born just a few years
after that.''  

``Yeah, my family is like obsessed with it.  I came here when I was five, so I
don't know.''

Mira was wearing the same black dress and leather jacket.  I wondered what she
thought of me, it was hard to tell.

At the party, I reluctantly agreed to play beer pong against Finn and Mira.  My
partner was yet another fraternity brother with a doughy face and body.  Beer
pong made me anxious, even though I was drunk.  Mira was getting louder and
louder, and screamed every time she made a shot, hugging Finn.

Eventually we left, making our way back to North Campus, a long walk over
bridges and along steep paths with wooden steps.  We were going to meet
some people at Finn's dorm. 

I noticed that Finn was behaving sort of strangely.  He was always a forceful
drunk, but now he seemed unusually angry.  He was talking about physics and
engineering, which always annoyed me when he was drunk.  Mira was arguing with
him.  He said that architecture was inferior to engineering, and repeatedly
referred to a building that had collapsed because of extra features added by
architects who hadn't done the necessary static analysis of the structure.

``You architects could learn these things, but you \textit{don't}.  Architects
are too lazy to learn \textit{math},'' said Finn.  It seemed like Mira was
about to cry, or become very angry.

``That's not true!  We do learn how to calculate load and stuff!''

``When.  What class?  You don't take any physics classes.''  Finn was walking
quickly, and Mira struggled to keep up in her high heels.

``We still learn how, though.  There's a lot of math.''

``There's more to designing buildings than calculating loads, Finn,'' I said.
''And don't computer simulations can handle most of the structural calculations,
right?''

``That's fucking lazy.  Mira, what would tell the families of the people that
die if your building collapsed?  Are you going to trust a computer program?'' 

``Ugh,'' said Mira, and fell back from Finn.  Finn seemed to be breathing
heavily.  I wondered if he was going to fall apart during college.  

When we got back to Finn's dorm, someone was drunkenly playing the piano, songs
they'd obviously learned as part of their high school music program.  People
were running around.  Mira left, and I left soon after.  

\section{}

The next day I kept taking notes.

You don't \textit{have} to \textit{want} to write, I wrote.

I fantasized about getting a little drunk with someone and then going to the
library together, picking up random books and reading them in the empty graduate
lounge.

``Surrounded by deniers, he must deny them,'' I quoted.

In the evening, I decided that I'd done enough reading.  I needed to go
somewhere to focus, so I went to the laundry room of my dorm.  I started writing
in my notebook about what I'd need to do to write the essay.  I looked at the
essay prompts.

I'd need to be able to bullshit, but in an authentic way, like the articles I
was reading on JSTOR.  To bullshit effectively, productively, and easily  (I
wrote), I'd need a weighty, diverse, mass of fermenting mental matter in the sea
of my mind.  I'd need facts, theories, jokes, friends, stories, personal issues,
obsessions (past and present), magic words, fancy sentences, quotes, frameworks,
trivia, lies, references, etc.  I would also need a prose style, a persona, that
would be easy to slip into, that wouldn't alientate a reader or listnerer.  And
I'd have to be able to structure it all together...

After getting Mountain Dew and Skittles from the vending machine, and trying to
start the essay by writing in my notebook, I gave up and went to bed.  I would
have to turn it in late.

\section{}

The night after the Communications essay was due, I left the dining hall and
started walking towards the woods, carrying my guitar.  I knew it wasn't a very
good use of my time, and that I should just be working on the essay, but I felt
that there was something in me that had to come out.  I walked down the
boardwalk to the river that fed into Beebee lake.  The boardwalk was well-lit,
and had BlueLight stations every hundred or so feet.  The leaves were starting to
fall, and the pines stood out clearly on the hillside.

I sat on a stone bridge and tried to make a up song, singing to myself.  When I
got tired of that, I started thinking, trying to figure things out.  What should
I focus on, I asked myself.  What was I going to study in college?  Obviously
things weren't working out so well right now.  I had to focus.

I tried to rule things out, based on my personality.  I wanted to be creative.
I wanted to be able to show my work, and know it was good.  What if I failed?
What would failure mean in different careers? 

I fought the feeling that ultimately failure was my fate.  That was defeatist,
what my dad would call ``a self-fulfilling prophecy''.  I wouldn't accept it,
and therefore I wouldn't fail.  

But if I studied business and became a businessman, maybe a startup
entreprenuer, failure would mean a powerpoint deck of my shitty business ideas.
That's all I'd have.  If I was a writer or artist, at least I'd have my art.
What else was there?  Engineering, like my dad? 

I didn't want to just be the powerpoint guy, I wanted a skill, I wanted to feel
powerful.  At least I have Chinese, I told myself.  The next day, I went deep
into the stacks of the Kroch library checked out a stack of contemporary Chinese
novels, and some Tang Dynasty poems.

\section{}

I sat at a small desk in Olin library, listening to a piece of music my high
school friend Brian had sent me.  Brian was notable for having been arrested in
10th grade for ``hacking'' the school's database, which included social security
numbers.  We became friends when I was sent to the Harkness Center for five days
of in school suspension, the consequence of installing port scanning software on
the school's network (a mistake).  Brian was stuck there for a year.  We both
agreed it was a wonderful place.  Brian also managed to get into the Gifted Math
Program, because the Harkness center didn't have Calculus.  He joined my carpool,
which raised eyebrows.

Brian and I liked the same kind of music, and we started playing guitar
together.  It made me really happy to go to the swamp behind our school with
some of Brian's fucked up friends and play guitar for them.  In the fall of
2008, Animal Collective concvinced us that some type of pop music would soon
destroy indie music,  and that not many people would be still be into lo-fi folk
and laptop music, like us.  Brian emailed me while I was in Taiwan, sending an
encrypted archive with text and midi files about his girlfriend and drugs.

I was happy to receive a new email from him, and listened to the music.  It
included a clip of me talking, recorded in my backyard.  I normally hated my own
voice, but I thought the music was beautiful.  I felt the sudden realization that
September was almost over.   Cornell was the new normal for me,
and would be for the next four years.  I had crossed some sort of Rubicon: I had
learned the meaning of ``the Rubicon'' from a reference in Ulysses. 

I listened to the Green Day song ``Wake Me Up When September Ends'' repeatedly,
watching the video for the first time.  Suburban High School life, which had
been mostly annoying and boring, now seemed like a deeply fraught emotional
landscape.  Brian's lo-fi music made me feel manic, despite my confusion. It was
happening: my friends were making \textit{good art}.  I would make good art too.

It was very painful, trying to write the essay.  I listened to Crystal Castles,
to try to get pumped up, but it just made me more antsy.   I chose the essay
prompt that involved the difference between public and private personas.
Persona came from the Greek word for ``mask'', which broke down into ``per'':
for, and ``sona'': ``sound''.  Greek actors didn't have electricity, so they
used masks that amplified their voices.  I would write an essay about how we
used personas in order to make ourselves more effective in life, and get what we
want.

At 2AM, Olin library closed, and I went over to Uris library, which was open all
night.  Despair was starting to set in, but told myself that I'd stay up all
night, and get the essay done.  I've got to fuck myself up, to be a special
person, I thought.  It's not going to be easy.  If I want to be great, I've got
to do great things.  It won't just magically happen.  

I sat in the ``fishbowl'' area of Uris, an area that my father had pointed out
to me once.  I got up and explored the building until I found the vending
machines, up near the clock tower, and bought peanut MnMs.  Around 2:30AM a loud
alarm bell rang, causing my stomach to leap into my throat.  A smelly man and
woman in grey suits came around, checking that everyone had Cornell IDs.  Then
it was silent.  I slowly made progress on the essay, spending most of my time on
Wikipedia.  By 4:40AM, I was finally mostly done, and I fell asleep in an
armchair.  I had to finish that essay, and then write two more.  We had to
address three prompts in total.  But hopefully the essay was very good.  I
wondered how many points I'd lose for it being late.

The sound of the belltower playing the Alma Mater woke me up.  It was around
7:30AM.  I went outside, bliking the crust out of my eyes.  It was cold and
people were walking to class.   I thought it was really beautiful. 

I went to the Olin Library cafe and bought a Buffalo Chicken wrap and another
Mountain Dew.  I sat down among the other studnets, and article on Thought
Catalog by Bebe Zeva, who I was amazed to realize was actually younger than me.
It's happening, I thought again.  I've got to write.  It's happening.

\section{}

The Communications essay made me realize that I was ``struggling'' in college.
But I felt hopeful. I was learning properly now.  After turning in the essay, I
let myself have a  day or two to recover.  I \textit{had} finished it, after
all.  

I lay awake, late at night, loving the atmosphere of the window open a crack and
the heater, with the sound of rain.  I felt alone, on my top bunk. 

The next morning I slept in and then went to have lunch at Okensheilds.  I took
a nap in the reading room, which was always warm and seemingly low on oxygen.
When I woke up, I was staring at a copy of Richard Nixon's ``Five Crises''.
``Life for everyone is a series of crises,'' it began.  I already liked Nixon.
I related to the phrases he used, ``derilict in one's duty'', ``a marred life'',
``taking one's job but not oneself seriously.''  I wrote down quotes into my
notebook.

I went to the one class I hadn't missed, and then back to my dorm room.  It was
still raining.  Facebook somehow led me to Omegle.  I talked to a few people.
One was another young man, who seemed a lot like me but less advanced.  We
exchanged gChat usernames and talked a few weeks later, late at night.  I also
talked to a girl who eventually told me that she had cancer.  I added her on
skype.  

Cancer reminded me of a Lorrie Moore story I had read long ago about a woman who
got cancer and then decided to kill herself.  I told the Omegle girl that I had
a story she might be interested in, and went to the library to find a copy of
the book.  I checked it out, and started the walk back to my dorm.  It was
already late at night, and I felt manic.  On the bridge over the gorge, I saw an
Asian woman using a flashlight to look at the green steel beams of the bridge.
I realized she was looking at clumps of spiders and taking notes on them.  This
was Cornell.

I found the story and typed it up on tumblr for the girl with cancer.  I
considered the fact that she might not actually have cancer, but decided that if
she didn't actually have cancer, she was still really fucked up for pretending
to have cancer on Omegle.  Her skype user photo was very homely, and looked like
someone with cancer from the midwest, so I was inclined to believe her.  I fell
asleep thinking about being one of the many people who had jumped off the green
bridge over the gorge.  Three people had done it last semester, forcing the
school to put up chain-link fences on the bridges.  I wouldn't want to do it
unless I had something to leave behind, like my laptop with a manuscript saved
to the desktop.  If I had that, I could do it.

\section{}

I got into the habit of doing homework in the laundry room.  I liked the big
plastic table, the white noise, and the warmth.  I did a lot of calculus
homework there, and I also remember it as the place where I read most of Tao
Lin's ``Richard Yates''.  

I decided to actually read ``Richard Yates'', which had just been published, due
to conversations that arose as George and I tried to write our Telluride House
application essays.  During the week leading up to the application deadline,
George and I spent every evening in the Graduate English Lounge, writing and
talking. 

I wasn't making very much progress on my essays.  I wrote one that described the
sounds of the beginning of a class in middle or elementary school: the unzipping
of binders, etc.  I showed it to George and he sort of smiled and asked if it
was supposed to be like James Joyce.  I felt discouraged. 

George didn't know what to write about.  I was procrastinating by reading
Hipster Runoff, and asked if George had heard of Tao Lin.  ``Of course,'' said
George, surprising me.  ``Maybe you should write your essay about Tao Lin,'' I
said.  ``But I haven't read any of his books,'' said George.  ``So?  That's
perfect.''

While George worked on his essays, I read everything that I could find online
about Tao Lin, or written by Tao Lin.  I had become aware of Tao Lin in Taipei.
My second host family was only a nervous old woman in the misty, hilly northern
part of the city.  I was very isolated during winter break, and my mental health
rapidly deteriorated as I tried to write my college application essays.  It
would be the start of a new decade, hopefully my decade.  I'd be 28 at the end
of the 2010's, I realized.  I started using my planner.

I became obsessed with Hipster Runoff, feeling that it somehow cut closer to the
heart of my life than anything else.  I first felt contempt for the excerpts of
``Shoplifting from American Apparel'' that I read, but I realized that some of his
writing accurately captured some of the strange feelings I was having in Taipei,
feelings that I had thought were unique.

I told that George that my essays were ``almost finished'', not wanting to
discourage him.  At midnight, I sent in an application that consisted only of
blank essays.  I told myself that I would finish them all the next day and
pretend that I had forgetten to attach them.  George seemed exhausted, and not
particularly happy with his essays.  Jokingly, I ordered a copy of ``Richard
Yates'' from Amazon.

\section{}

Sometimes I got dinner with Morgan and David at Appel on North Campus.  It was
disconcerting spending time Morgan in the evening, surrounded by freshman.

David often came to dinner after rowing practice with some of his teammates.  He
had joined the lightweight rowing team.  I was sort of annoyed by how Morgan
went along with the jokes of David and his teammates.  The jokes were pretty
funny, usually repetively referencing pop culture in some way, but my heart
wasn't into them.  

I vaguely wondered what it was like, doing something like rowing.   I knew that
the rowing machine was very difficult: my dad had bought a rowing machine when I
was in 7th grade, and had told me that I'd be allowed to use his Laser sailboat
when I could row 5000 meters faster than my mom.  I remembered turning on the
radio and rowing for a long time, until I broke 10,000 meters.  After doing this
for a few days, my arms developed hard muscle for the first time in my life,
which had legitimately impressed people at school.  The erg was next to my
parent's little bookshelf, and when I got bored of rowing I'd sit down and read
``The Joy of Boys'' or other parenting books, and think about how weird it was
that my parents thought of me this way, like every other boy, turning into a
young man, with sexual feelings and a personality.  

\section{}

I took my first Calculus prelim.  I was surprised at how hard it was.  I mostly
understood the material, but I was having more and more trouble paying attention
in lecture.  I got a score in the high 70s, which was the mean for the test.  I
felt moderately angry, and promised myself that I'd study harder next time.

\section{}

I had to go home for Fall Break, because my uncle was getting married.  Paola's
mom volunteered to drive George, Paola and I back to Buffalo.  She took us out
for lunch at the Statler, the on-campus hotel partially managed by students.
``My mom still really likes you,'' said Paola as we waited in the lobby.  Paola
was wearing a panel dress with fall patterns, and I felt relaxed and confident
about the rest of the semester.

The midterm essay for the Ulysses class was due right after break ended, and I
felt like I'd definitely be able to finish it in the next five days.  The
wedding was in the hills of Virginia near Camp David, and I would have lots of
time in the car to read and write.  

The Communications essay had taught me some hard lessons.  More importantly, I
thought, I actually cared about the Ulysses class.   I never missed it, and the
professor seemed to really like what I said in class.  I was interested in Keri,
the strange hunchbacked woman.  One time, we were sitting next to each other,
and she saw my notebook.  She got very excited.  I explained how it was
color-coded based on the type of information I was writing down: black for my
own thoughts, green for quotes, blue for facts and things to look up, and red
for important things.  She cooed over it for a while, and I started to feel
strangely attacted to her.  She wore camoflague most days, and seemed very out
of place in the beautiful old classroom.

Paola's mother greeted us happily and we led her up to the sunlit dining room.
I got a turkey sandwich with apple and mustard, and napped most of the ride back
to Buffalo.  

Later that night, my mom drove Paola, George and I to the movie theater.  We all
wanted to see ``The Social Network''.  I felt strange and calm, getting out of
my mom's car at the strip mall where the theater was located.

We all agreed it was a really good movie.  I was moved and inspired the scenes
of Harvard at night, with students walking quickly to and from buildings in a
type of light that I couldn't quite put into words.  It seemed to capture the
feeling of significance and youth that I deeply wanted in my own life.  I also
thought that the rowing race scene was very beautiful.  Mark Zuckerberg had been
a Sophomore in when he had founded facebook.  I felt stressed.  I had to meet
the right people now, and make things happen. I needed the leverage that he had,
the ability to program or create.  Could writing even be like that?  Writing was
ephemeral, it was hard to get people to respect or admire writing.

It was hard to focus on my essay, during the drive down to Virginia.  I thought
of all the car rides in my childhood where I had promised myself that I'd read,
write or think about something specific, but only stared out the window and
thought about myself. 


We stayed at a motel.  My parents could tell I was getting more and more
stressed about the essay, and gave me time alone to work.   It was weird
spending time with relatives that I'd never met before, my dad's side of the
family.  Many were very similar to us, very awkward.  I could see myself in
them.  I felt very aware of my body at the lakeside wedding ceremony.  I spent
as much time as possible with my book, telling people that college was busy and
hard work.  The next morning, we went to a brunch in the woods near Camp David,
and then drove back to New York.

\section{}

My parents dropped me off in Ithaca.  I went straight to the library to start
writing.

I had my thesis: that James Joyce had been heavily inspired by the Italian
heretic Giordano Bruno, burnt at the stake by the Inquisition.  I would show
that Bruno's thought had informed many of the more humanistic elements of
Ulysses's structure, especially the last chapter from Molly Bloom's perspective.
Bruno believed in an infinite, non-hierachical universe, where virtuous planets
orbited other suns.  It was important to be human, Joyce believed: the
incomprehensible world was a intractable problem for sterile young men like
Stephen Deadalus, but a consummated joy for a man like Leopold Bloom.  

I structured my paper around qutoes from Bruno, which seemed to perfectly
describe the narrative arc of the novel and my own time at college.  The
infinite nature of the universe, the thousands of students at Ivy League schools
across the Eastern Seaboard: it was impossible to try to be above all that.
``Every sensible world is imperfect,'' wrote Bruno, ``wherefore evil and good,
matter and form, light and darkness, sadness and joy unite, in coincidence of
the contraries where things are everywhere in change and motion.''

I was living in the coincidence of the contraries.  I would have to deal with
the finite nature of my time and being.  I wrote down the things I needed to do
for my parents: 1) get good grades, 2) don't die/suicide, 3) don't hurt other
people, 4) support self, 5) get married, 6) talk to people, 7) eat sleep well.

I would be calm, tolerant, but not indifferent.  It's going to hurt, I told
myself.  That was the consequence of fucking around all weekend.

\section{}

I stayed up most of the night trying to write the paper, and managed to get a
few rough paragraphs done.  Around 3am, I subconsciously gave up, but spent an
hour or two reading on the internet before going to sleep in disgust. 

I slept through Calculus but made myself go to Ulysses, wearing a green polo
shirt.  I smiled at myself in the mirror.   When everyone was turning in their
papers at the end of class, I slipped out the door.  

I felt strange the rest of the day.  I talked brightly with Morgan and David at
lunch, and worked on a crossword with them.  Everything that happened felt
critically important.  I had two full days to finish the essay.  I would
research Giordano Bruno further.  I felt I could write a paper that made
original contributions to the study of Ulysses. That would make up for the
lateness.  And, I didn't care that much about grades anyways, I told myself.
I'd prefer good grades: I wanted to be the type of person who got good grades,
just like I wanted to be popular and had a pretty girlfriend.  There was no use
in denying that.  But I figured those things would work themselves out.

\section{} 

Paola texted me, wondering what I was up to.  I was working in the laundry room.
I invited her over.  I wondered if she still felt attracted to me. 

It was one of the first dark, cold nights.  Paola seemed happy to see me.  We
talked about our schoolwork.  It seemed like college was starting to change
Paola.   One of the things that had made me want to break up with Paola was the
fact that she was almost constantly twee and bubbly, but her words now seemed to
be heavier, as she spoke about how difficult her statistical methods class was.
She was considering changing her major, or studying abroad.  I said that
Calculus was hard too. 

``I'm really tired,'' she said, and put her head down on the folding table.

``This is a tricky problem set,'' I said, looking down at the book.  I suddenly
felt very self-conscious.   Then I realized that Paola was making crying sounds.
I listened until I felt sure that it was from Paola, and not some other noise
from the washing machine.  I wondered if she was going to tell me what was going
on, or if she even knew that I could hear her.  

I imagined patting her back, hugging her, or saying something, but I couldn't
actually make myself do these things, because of some sort of strange
embarassment.  I felt ridiculous and weak, pretending that I was oblivious.

She lifted her head from her arms, and straighten herself.  

``I think I'm going to go back to my dorm,'' she said in flat voice.  I looked
over in her direction, not meeting her eyes.   

``Oh yeah, good luck.'' I said.

``Good luck,'' she said on her way out the door.


\section{}

I started to spend almost every evening in the Graduate English Lounge with
George, working on my essay.  I knew I had a problem with procrastinating on the
Internet, so I left my laptop at home.  This worked for a day or two, but I
started to use the library search terminals, old Dell Inspirons that I had to
stand up to use.  

I realized I wasn't taking enough advantage of the fact that I wasn't living
with my parents anymore.  I ordered a set of lockpicks from a PayPal shop, and
showed the excitedly to George.  I thought that the lockpicks would reignite our
stalled ``secret room'' plan.  I practiced using the lockpicks on my own dorm
room door until I could open it easily, and showed my new skill to the
floormates.  It seems like they were starting to respect me more, for my music
and other weird skills.

I also ordered a MIDI keyboard and pad.  I was listening Crystal Castles a lot,
because one of the people in the Ulysses class emailed everyone a link to one of
their songs, ``Air War''.  The song included a remixed a recording of an Irish
woman reading from the ``Sirens'' chapter of Ulysses.  I was realizing that I
loved pop music, even music like Nicki Manaj.  Very late at night I would sit at
my desk with my headphones and listen to ``Bottoms Up'' over and over again,
probably loud enough so that my sleeping roommates might of heard the sound
leaking out of my default apple earbuds.  

I was also reading personals on Craigslist again.  I saw one that was obviously
posted by the ``Alice Munro'' person who said that I should hit on girls in my
dorm.  I sent another email to her, using the gmail account I had created to buy
the lockpicks with, instead of my Cornell email.

When I recieved a response, I showed the emails excitedly to George.  She
was 23 year old Cornell grad, living in Trumansburg, a small town about 25 miles
away from Ithaca.  I imagined the five years of time between me and her, all of
the experiences she must have had.  She told me to call her ``AM''.

George seem bemused, a little worried, maybe disgusted.  I felt thrilled.  We
were at Okensheild's dinner.  It was ``dinner for breakfast night'',   It
was breakfast for dinner night, and the sun was now setting around dinnertime
instead of at 9pm, like in the childhood summers.  The sun filled Okensheilds,
and I had to shield my eyes as I made my way back to where George was sitting,
carrying pancakes, eggs, and sausage. 

\section{}

Finally Morgan and I were studying together, somewhat regularly.   Morgan and I
studied on the top floor of Uris library, where the sound of the belltower was
noticably louder, and the sound of the wind often made me sad.  I fantaized
about her joining George and I, and three three of us becoming a dependable,
growing intellectual group. 

On a particularly dark October day, Morgan fell asleep, her chin resting on her
collarbone.  I stared at her, wondering if what I felt for her was real.  She
was very beautiful, in a way.  

\section{}

AM and I emailed back and forth a few times.  She sent me a picture of herself.
I spent almost an hour at my desk, late at night, trying to take a good picture
of myself.  I eventually sent her a picture of myself making a pouty face, with
a pen behind my ear.  I was initially surprised by AM's picture.  She had short
hair.  I told her that she looked kind of like Virginia Woolf.  

\section{}

I was not going to class, because I needed to finish the essay.  I went to the
the Appel dining hall with a book every morning, after most people had already
left.  I felt a sort of morning sickness that was helped by the book and
listening to catchy, pretty songs over and over.  

On Thursday morning I was interrupted when a man in a ROTC uniform came up to my
small table.  ``Is anyone sitting here?'' he asked me.  The skin around his
mouth was scaly, tight and slightly red.  He wore glasses with small lenses, and
made eye contact with me.  He pointed at the seat across from me.  

``No,'' I said.  He sat down, making the noise that people make when they've
been on their feet all day.  After a moment, he stood up and said ``I'll be
right back''.  

I thought that he was brave, forcing himself to socialize with people.  I had
stopped doing that.  I should probably make myself do stuff like this, I
thought, even though it was so annoying. 

When he came back, I watched him eat.  His eyes looked like little blobs of
shit.

On Friday, I decided that if I didn't finish the essay over the weekend, I'd
talk to the professor.

\section{}

I had an Oral exam in Chinese.  I had been up until 4am reading about Aristotle
and my face felt stiff.  The teacher asked why I was missing class, and I smiled
and apologized.

We started going through the pre-defined dialogue, but I found myself having
more and more trouble.  I felt far away, dissociated.  We were talking about
paper cranes, and I was forgetting words.  I was transfixed by the desire to
excuse myself and leave.  The teacher gave my a pitying smile, and told me to
take care of myself.

George and I were getting more interested in what I called ``laptop music''.  I
wanted to use my laptop as my primary instrument.  I wasn't having much luck
with making songs on my classical guitar.  I wasn't going to be able to make an
impact online doing that, anyways.  

What finally convinced me to download Ableton was finding the song ``Is This
Really What You Want?'', made by Tao Lin and Carles.  In an effort to focus
better, I'd been going to the computer lab at Robert Purcell Community center
after finishing dinner, instead of back to my dorm.  But instead of working on
Calculus or Ulysses I generally just use the computers there to look up more
information about Tao Lin and other internet things.  I felt that I probably
knew more about Tao Lin that just about anyone else: I researched his father's
arrest, and read all the interviews and blog posts that I could find.  I was
thrilled when I found ``Jesus Christ (The Indie Band)'', which was just Tao Lin
and Carles talking over a synth-pop background.  They only had one song.  

I showed it to George, and we eventually agreed that it was a great song. It was
also proof that  Tao Lin and Carles were distinct human beings.  I pirated the
Ableton music production software and started learning how to use it.  

\section{}

The next week, I unexpectedly received an big chain of emails.  It took me a
while to figure out what was happening.

\textit{Professors,}

\textit{My name is not Patrick, and the email below is not directed to me.  It seems
that you are trying to contact a Patrick Steadman.  I do not know who that is,
but I have occasionally received emails intended for him.  My name is Peter
Steadman, and my email address is psteadman@gmail.com.  I'm guessing that
Patrick has sometimes made a mistake when giving out his email address.  I
imagine it must be something quite close to mine.  Please note that he has not
received this message, and if you could let him know that I have received
another email intended for him, I would appreciate it.  I have received
several others, including one or two from his mother, so I don't imagine that
he has given an erroneous address on purpose.
}


\textit{Well, this explains some of the non-responsiveness, I suppose.  I just
forwarded my message to Patrick's Cornell e-mail address:  pts44@cornell.edu.}


The assistant professor of the Ulysses class had forwarded me the email, asking
me to ``please let us know how you are and where you are with this''. 

They were wondering where my essay was, obviously.  I hadn't been receiving
their emails because they were sending them to the wrong email address.

A few minutes later I received an email from someone named Pamela.  I was in my
boxers, starting to sweat.

\textit{
He reports that you still have not turned in your paper, despite a meeting last
Monday where you told him you would complete it and turn it in.  It seems as
though you are struggling with something that you are having trouble
overcoming, and I am urging you to set an appointment with Lisa Ryan or
Catherine Thompson in my office.  We can help you. }

\textit{
As our concern for you rises, our method of trying to reach you will escalate 
we need you to respond so we do not have to take such measures.}


I took these emails as a wake-up call.  I had let things get out of hand, but I
could still probably turn it around.  I took a shower and shaved. 

I'd study for the Calculus prelim on Thursday.  I hadn't been very good about
going to class because it was early in the morning, but it'd be good to focus
on math.  There was also an essay due in the Communications class.  I'd work on
that after the Calculus prelim, and turn it in a bit late again.  The Ulysses
essay would have to be put on hold for a while.  It'd be okay.

\section{}

I stared at the Calculus prelim.  I didn't know how to do anything.  I hadn't
studied, really. I had forced myself to go to the exam anyways, which was hard
enough.  But still, I didn't know how to do it.  Tears came to my eyes.  The
test room was a large, warm apitheater with red carpeting.  I looked through the
test booklet for about a half hour, attempting some of the problems on scrap
paper.  Then I turned in the empty test. 

It was a beautiful night, and I walked home with other testtakers, but feeling
removed from them.  I felt that I might really be in trouble now, grade wise.
Time had passed so fast, last week, and now I was now deep in the semester.  I
dreaded the thought of my parents.  I couldn't let the semester keep going like
this, I had to turn it around.  This would be my true test.

\section{}

On Halloween, Sunday night, George and I went out together.  We were looking
forward to the Risley art dorm party, which would hopefully be
\textit{different}.

Some of my floormates were also planning on going.  A group of girls were
standing in the entrance to my suite, talking to my Asian roommate.  The girls
were wearing skintight black dresses and cat ears.  My roommate was drunk,
sitting at his desktop computer.  He told the girls that he was too drunk to go
out with them.  I borrowed the robe from his Jedi costume.

I felt excited.

George and I walked towards Risley.  A group of guys yelled ``faggot'' at me,
probably because of my robe and mannerisms.  I turned around and screamed
wordlessly at them, unable to see their faces.  George laughed.  I felt
anxiously happy and hopeful.

There was a long line for the party.  George and I waited in it silently.  Most
other people were wearing pretty elaborate costumes.

When we got inside, it felt more like an ``activity night'' than a party.  There
was a volunteer massage therapy station and an empty dancefloor.  It was not
worth the effort to get drinks.  

It was kind of fun to wander around the building, looking at the drawings and
paintings on the walls.  The art kids seemed to all know each other and rushed
in groups through the halls, laughing and talking in a way that seemed to cause
George pain.  We eventually found a quiet place in the moonlit courtyard, and
talked about things. 

George and I eventually left, and managed to get into another party.

\section{}

On Monday, I woke up feeling very anxious.  I had an email from my Chinese
professor, in English for the first time, that simply said ``Patrick, you've
been absent for a long time, please set up a time when we can talk.''

I hadn't responded to the emails from the professors or the advisor.  I had
planned to take care of the Calculus prelim and make a good start on the essay
before I talked to them.  That had not happened.

I buried my head in the pillow.  It was 4pm in the afternoon, and sunny, almost
summery. 

I'm doing the Raskolnikov method of problem sovling, I thought: sleeping and
hoping the problem is solved in my sleep.

\section{}

I don't know if I can return to my family like this, I thought.  But I felt like
I didn't belong at college.  I belonged in my parents basement, reading. 

I hadn't expected this problem to happen.  I was a totally ordinary person with
nothing special to offer the world.  I was torn between routine and the need to
get shit done.    

Sometime the week before, when I was at the library after midnight, I started
drawing my hand on a sheet of printer paper.  I was trying to make it a routine.
The drawings seemed to be improving; they were good, at least.  I drew on the
back of handouts and printer paper.  I imagined compiling them into a book, and
sending them to my high school crush Ally, as a way to prove that a person could
improve.  

Ally had rejected me in 6th grade but we eventually became friends in 9th grade
art class.  She had sent me a message on Facebook, telling me that she was going
through a rough time.

It took over an hour to make one drawing.  I made a list of the things I needed
to do each day.

I was in a mess.  But I was starting to feel a euphoric sense of the largeness
of the world, inspired by all of the Giordano Bruno reading.  I drew pictures of
Bruno, the heretic. 

One of the benefits of my failure to be a standard ``successful college
student''  was that there were so many other places I could go.  I talked to my
old friends from high school on facebook.  Facebook had just released a chat
feature.  It would change things, I thought.

\section{}

On Tuesday morning, I got an email from the TA of my Calculus class, saying we
needed to talk about my prelim.  I also had an email from Mommy, saying she had
to drive to Syracuse for a meeting, and could have dinner with me on the way
home.  

``Do you think people can change?'' I asked George, making him look up from his
book.  We were in the Graduate English Lounge.  The email from my mother had
made me feel better, and I got dressed and headed to the library, and decided to
meet with my Chinese professor in the afternoon.

George thought about it for a bit.

``I don't know.  Probably not.''

``I'm really afraid of that.  I hope people can change.''

``Well, I think situations can change them.''

``But they can't like consciously change themselves, you mean?''

``Yeah.''

I kept thinking about this for the rest of the day, especially when I was
walking to meet Teng Laoshi.  It was a muggy, foggy day, and the afternoon still
felt like morning.  Teng Laoshi, like everyone else, seemed sympathetic.  She
agreed that it was still possible for me to pass the class, just barely, but
that I should talk to my advisor and try to drop it.  I left her and went to the
library.

I was reading a pdf of ``100 Years of Solitude''.  It seemed to reference many of the
things I had researched while reading Ulysses.  I felt like I was becoming aware
of the nature of literature, of many specific things that made up humanity,
somehow beyond words but hinted at by words, only carried along by the feeling
created by good sentences, or the feeling one got at night, walking from place
to place.  I decided to finish reading the book in one sitting.

\section{}

The next morning, when I returned to my dorm from the library, the cringey male
RA was waiting outside of my door.  A week or so ago, before the prelim, he had
told me to email back the advisor asking about me, and had also mentioned that
since I was an artist, maybe I could make a mural for the common room?

``Hey Patrick.  Uh, just so you know, some people were looking for you.  You
should definitely respond to them, as soon as you can.  Alright?''

He gave me a business card with Pamela's name and phone number on it, and I went
into my room.

I opened up my laptop and checked my email.  I immediately knew I was dealing
with a crisis: 

\textit{If you do not respond to this email within 24 hours, we will ask
the Cornell Police to do a ``well visit'' to your class and/or dorm room.}

I quickly responded with one line:

``I'm fine.  'Well visits' sound embarrassing/traumatic.'' 

I also responded to a few other emails.  Forlornly, I looked at an email from the
Communications TA, asking if I was going to turn in a paper.  A few days
later he had replied informing me that the paper would be an F even if I turned it
in, at this point.  I responded to the math TA, telling him it had been a bad
few weeks, but that I thought I could pass the class.

I called Pamela's office and set up an appointment for the
next day. 

I spent a lot of the evening talking to people on gChat.  I was talking to my
friend Jennifer again, who was at Yale.

\section{}

I woke up early and made myself look normal.  I felt very strange and detached.
I knew the meeting would go well.  One of the few things I was born with, I
believed, was the ability to make adults trust me and think I was the ``right''
sort of person.  I always remembered adults, especially older adults, praising
me.

I smiled and shook the hand of the advisor, and followed her into her office.
She talked about how her and some of my professors were worried about me, but
felt that I had a lot of great potential.  I explained that a lot of my troubles
were stemming from trouble writing my essays, and not really liking what I was
studying in the College of Agriculture and Life Sciences.  I lied a little bit,
telling her that I'd been going to my other classes, and that I would stop
working on the English class.  I also showed her the emails that my professors
had sent to the wrong email address, explaining that it was part of my
unresponsiveness, but that I was sorry for the problems it had caused.

By the end of the meeting, it seemed like the advisor was concerned, but
confident in my ability to turn things around.  I left the office feeling
better, and felt that I had a chance to fix things.  I just needed to pass my
classes, I told myself.  This semester wasn't going well, and wouldn't look good
on my transcript, but I'd face the consequences, and past most of my classes,
and get on track.

\section{}

My hair was too long, and I didn't want it to look like a ``jew fro''.  Instead of
going to get my hair cut, I ordered hair clippers from Amazon.  I cut my own
hair in the dorm bathroom, the hot metal of the clippers burning the nape of my
neck.  

A little after midnight, I was walking home from the library.  I saw a guy who
seemed like he was crying.  He looked pathetic, I was disgusted.  When I got
back to my room I realized that I was disgusted because I was that way too, that
ugly and alone, but I didn't allow it to show. 

I knew I needed to get my shit together.  I needed to manage my life.  Things
were a big mess, I recognized.  It was too unstable.  This was a big change in
my thinking.  I needed a clearly-stated philosopy of life, that was
well-articulated, but devoid of sophistry.

I needed to fully believe certain truths.  I wrote that I needed to believe that
there was nothing inherently special about being me, Patrick Steadman.  Nothing
was ``meant to be'' in my life, in any objective way.  It would be dangerous,
false, and inefficient to believe otherwise.  It'd make me hate other people.

I needed to believe that I was probably not a ``genius'', whatever a ``genius''
was.  I needed to believe and accept that other people were better than me, at
things.  Even if I was a genius I wouldn't be a genius at every moment.  My true
self couldn't be expressed at every moment.  

I needed to recognize that what I was doing, wasn't working.  But I also had to
realize that I didn't have to ``remake'' myself in some epiphanic way.  Some
things \textit{were} working.  And remaking myself was impossible, I had to
remember that, even though it was a seductive idea. 

The one thing I \textit{hadn't} tried was diligent, patient, disillusioned
effort.

I thought of the pitying, longing, insecure feeling I had when I saw people
studying the sciences or math.  I felt hopeful about this new approach, and went
to sleep. 

\section{}

On Thursday night, my mother pulled up to Robert Purcell Community center in the
Volkswagen Jetta.  It was one of the first cold, wet evenings, and north campus
seemed quiet and subdued.  I got into the car.  She commented about my hair,
which was sort of lopsided because I cut it myself, but she seemed cheerful.
She seemed like she had absorbed positive energy from her day of meetings in
Syracuse.  

We drove down the the Commons, and walked around in the rain a bit before we
chose the Mexican place.  The Commons smelled of spiced apple cider, cooked on
the stove.  My parents had taken me to visit the Ithaca Science Center when I
was a kid, which was the center of a scale model of the solar system that
extended out through the long plaza of the Commons.  Even as a kid, Ithaca
seemed to have the maximum amount of ``culture'' that a small town could have,
mixed with a sort of run-down upstate counterculturalism.

Soon we were talking about school stuff.  I had talked to her about my essay
problems over the phone a few weeks earlier. 

``I talked to an advisor and stuff,'' I said.   '' I'm handling this.''

I also told her I hadn't managed to finish my Telluride house essays, and that I
was sorry.  

``Next year,'' she said, and talked about some of the times she had had problems
in college.  It seemed that she didn't really ``get'' what I was dealing with,
but I wasn't upset.  It was my problem.

When she dropped me off back at Jameson hall, it was as dark as the middle of
the night and raining.  She noticed I was limping as I walked out of the car,
the water from puddles splashing over my sandals.

``Is your ingrown toenail still a problem?'' she asked, smiling.

``Yeah,'' I said, embarassed.  It had been getting worse again.  

``You should go to Gannett and get it looked at.  Maybe you could talk to
someone there about your essay writing problems too.''

``Okay, it'll be fine though.  Don't worry.''

I just need to pass, I reiterated to myself.

\section{}

I was spending a lot of time reading and writing in the dining hall.  I think I
was inspired by Thomas Pynchon, who was known to get a big plate of spaghetti
and a coke and spend hours studying.  If I stayed at the dining hall most of the
evening normally someone I knew would see me and sit down with me.  Spending the
whole evening in the dining hall would drain me though, and I would leave as the
work-study students began cleaning.  One evening I heard people singing in one
of the rooms of RPCC.  The SA had organized a karaoke night; they were playing
videos on youtube for people to sing over.  I thought of my host family in
Taiwan, and the simplicity I felt living there, singing Taiwanese songs on the
karaoke machine in their living room.  

I said that I wanted to sing, and found the Teresa Teng song I wanted.  There
were about five people lounging on the couches in the room, most of them
overweight.  I sang the song, enjoying the sound of my voice, and the ease that
the Chinese lyrics came to me and the feeling I felt for each character.  After
the song no one made eye contact or said anything to me, and I walked away.  For
the first time at Cornell I felt sad and defeated.  I wondered what this new
feeling meant.  My thoughts went to my mother, my father, and Morgan.  Maybe I
was realizing that I was nothing special, and that was the lesson of college.  I
was fated to be with Morgan, who reminded me of my mother, and was perfect.
Soon enough I would kiss Morgan and she would be my first girlfriend and calm my
heart, and I would be able to write my essays like a normal student.  It was my
fate.  I went back to my room and fell asleep without taking my contacts out,
tears at the edge of my eyes.

\section{}

Multiple times in the same week I lost the four-colored bic pen that I needed in
order to take notes.  I went to the the Cornell store to buy a new one, and
generally spent at least an hour there.  The first book I bought there was ``The
Literature Student's Survival Guide'', which contained summaries of the Bible
and other canonical texts.  While reading Ulysses late at night in the library,
only managing to finish a page every fifteen minutes or so until I fell asleep,
I suddenly awoke and realized I wasn't really ready for the book yet.  I needed
to learn more, and discipline my character before I could write about Ulysses.  

I learned where many of the words I was using in my essay came from: the word
Shibboleth, used by Gileadites (balm of Gilead people) to identify and kill
Ephraimites in their midst, who couldn't pronounce with word \textit{shibboleth}.
It was a word I recognized from the URL querystrings that I saw when signing out
of the various academic portals.  A.M. had mentioned that I should take
advantage of my access to JSTOR: I had seen SHIBBOLETHs being passed around when
connecting to JSTOR.  I looked up a bunch of articles about Bruno and Joyce on
JSTOR, and was amazed by the new wealth of knowledge.

I made myself stop reading all the other books I had borrowed from the library.
I had checked out a large stack of old books about Anglo-Saxon mysticism, books
that I would have found fascinating as a child, when I lay awake in bed at night
hoping desperately that magic was a real thing, hoping so hard that tears came
into my eyes.  I remembered that this type of behavior had continued into middle
school.  Now I felt like magic was still possible, that certain experiences I
had crafted for others must have seemed magical: when I had created a hidden
shrine in the basement of our high school and led some friends and acquaintances
down there to ``discover it''; when I had spent weeks planning for a party my
freshman year that ended up being a good party where things happened; and
hopefully I would be able to create an experience like that for George and I,
finding the right party.  But maybe the lesson I was learning was that
``magical'' experiences could never be magical for the person creating them.
George regarded the big stack of books with skepticism as I brought it back to
my dorm.  In early November I had received an email notifying me that I owed a
\$80 fine to the library, because the books were overdue.

I didn't have much to show for all my agonizing over the Ulysses essay.  What
did I have to show for the last week, I asked myself.  It was a lacuna, lacuna
was a word I had learned from Ulysses. I had screwed up my semester over the
essay.  I decided to make a final push on the Ulysses.  It would be the easiest
way to solve my problems: if I could just show my advisors and parents a
finished essay, or even semi-finished essay (I had to be realistic), they would
understand where my efforts had been focused.  Transferring to a different
program within Cornell was probably necessary, anyways: hopefully one of the
state schools, since they were much cheaper.

I went to the library with only my copy of Ulysses and my notebook, and started
working my way through the last few hundred pages of my book.  From research on
JSTOR I knew that my thesis was heavily dependent on the last few chapters of
the book, where everything miraculously came together.

My notebook had been becoming more and more organized as the semester passed.
Now I wrote down every word I didn't know in blue, and added a definition, so
that I could easily find and review them.  Joyce referenced so many beautiful
things, the things I wanted my vocabularly and life to be filled with: rich
looking green and yellow drinks, feminine calm, and very specific types of
plants like jonquils.  I went to the computer and looked up a picture of a
jonquil, and then drew it in my notebook.  I wanted all the things the ancients
had.  I googled ``Frankencense and Myrrh'', to see if I could buy it.


\section{}

I finished reading Ulysses on a Saturday morning in the dining hall.   I spent
almost all of the nights during that week in the library, until after the 2AM
when the security people came around to check student IDs.  I had made slow
progress on the book, but the final chapter seemed like the perfect
consummation, the imaginations of two perfectly real characters treading the
same memories, affirming love.  It fit perfectly with my essay, and my thoughts:
I wanted to live a consummated life, not just a life of books.  I would write
the essay and fix the semester.

To reward myself, I met up with Paola (she had texted me).  I went to the Cornell
store with her, and bought one of the books I had sampled online: ``Foucault's
Pendulum''.  It had strange, silky pages.  Paola and I went to sit outside on
the quad, something I had never really done.  It was windy, and almost too cold.


\section{}

Morgan, Seth, David and I always talked about going out, but it never really
happened.  Sometimes Morgan and David would talk jokingly about things that
happened on the weekend, but those nights out seemed to be with their
floormates.  We didn't actually all go out together until a chilly night in the
middle of November.  

I had high hopes for the night.  I figured something would happen between Morgan
and I.  I had no idea how to negotiate the physical space between her and I, and
saying something to her seemed impossible in her presence.  I figured alcohol
would make things happen.

We ended up going to a few fraternities around the new West Campus dorms.  I
watched Morgan play beer pong with Seth.  I had a few shots.  Morgan, Seth,
David and I danced.  Seth was being goofy, and Morgan laughed.  

I was starting to feel more tired than drunk.  A few times, Morgan and I talked
about our schoolwork, but when Morgan was drunk, talking to her about Ulysses
and her child development stuff felt hollow.  The converstations didn't last
long.

I was falling behind the group, which had swelled to include some other loud
guys, as we headed to an afterparty in Collegetown.  I had no connections at
these places so I was just along for the ride.  I realized suddenly that it was
November, and I didn't have the social connections that other people were
developing, despite my best efforts.  I stalked behind Morgan, listening to her
talk to a group of loud drunk guys.  I felt more and more drunk and unhappy, and
felt angry and disappointed with Morgan and also myself.

I dropped away behind a wall, and then ran away, and stopped again, feeling
ashamed.   I was inside the arches of the World War II memorial.  I waited,
breathing hard.  Suddenly Morgan appeared.  

``What are you doing, Patrick?'' she said, seeming annoyed, but a little amused.
I stared at her in mute shame, knowing that I didn't have words for what I was
feeling.  ``I think you should go home,'' she said.  ``Can you get home?'' she
said, in an overly concerned voice.

``Yeah,'' I said.  I set off on the direction of our dorms, but then turned
towards the suspension bridge.  I felt a sort of thrill at my unhappiness and
disappointment, watching my feet move quickly in the orange streetlight.  It
felt like something was broken and I was being pushed on by a force.  I walked
over the bridge, thinking about climbing over the fence.  I told myself I was
fine and that I was free of Morgan now, and that tomorrow I'd write.

\section{}

On Monday morning I went to the library early.  On the way there, a girl I had
noticed before rode by me on a bike, her hair shining in the morning light.  I
felt moved.  I had seen her reading a book in the dining hall, once.  I had
tried to see the title.  It seemed to be about gender issues.

At the library I wrote a long poem-like thing about her in the margins of my
notebook.  Don't flinch away from a beauty you can't avoid, I told myself.  Life
was beautiful, just like most books said.  I would be saved by desire for sex and
happiness.  I didn't want to have a miserable shitty life.  I wanted beauty and
happiness.  I would work hard to get them.  I didn't want to have a miserable
life.  I was reading ``The Garden of Eden'' by Hemingway, because AM had recommend it to
me.  It was really beautiful.  I wrote down sentences from it in green pen,
which described the the ``fun'' that Hemingway and his young wife had had, by
the blue ocean in their rough sweaters.  More than anything I wanted beauty like
that.  Tears came to my eyes.  I was sick of this.

I forced myself to stare at the blinking cursor until I thought of a sentence to
write.  I wouldn't let myself look away from the cursor for a single second.
This seemed to work.  I wrote and rewrote the first paragraph of the essay,
uniterruptedly, until it made no sense when I woke up the next morning.  In
dismay I went back to sleep, and woke up at 4pm, disoriented.  

\section{}

I wanted to discipline myself.  For my parents, for her.  In my notebook I wrote
that I would spend seven hours sleeping, one hour on the internet, one hour
drawing, one hour playing my guitar, one hour for eating and hygeine, and the
rest studying.  I would get myself out of the hole.  But some things were simply
too hard or painful for me to apply discipline to.  I could only think about
writing the Ulysses essay for about five minutes at a time before I would find a
way to stop and do something else.

\section{}

I lay facedown on the upholstered cushions on a bench in Uris library.  I had
just awoken from a short nap, after making no progress on the Ulysses essay.  It
was late, and my core felt cold.  I kept my eyes closed, thinking of strange
things from the past: the gerbil warrior king I had imagined as a childhood.  I
tried to rememeber the name I had given him: GerbilAwfulous.  My own life felt
strange to me.  I remembered how, at last Christmas, my parents had told me that
I was terrified by The Nutcracker as a child.  I thought about AM and Jennifer,
the skype girl with cancer, all of the people I talked to online.  Eventually it
became too painful to continue lying there, and I sat up and went on my
computer.  Jennifer was online and we skyped for a few hours.  She was being
strangely combative, talking about her experience at Yale.  She was talking
about a Danish guy she had met, and how attractive and ... brooding he was.  He
was a Junior, and wrote for the literary magazine.  I asked her if I wasn't
attractive in the same way.  She laughed.  Jennifer was still angry at me for
unempathetic things I had done while talking to her on the Internet in Taiwan.
``You can't be attractive, \textit{in that way}, if you know what I mean?''
``Why?'' I asked repeatedly, wanting to know. ``It's hard to explain in words,''
she said.  ``You're too...\textit{hairy} to be that way, you know?''

I had done everything but my duty that day.  At worst I would end up like a very
overcultured Chinese gentleman, if I continued like this.  I thought of my host
father in Taipei, with his beautiful house built into the hill, full of shelves
of books and DVDs.  

\section{}

``I feel like I keep putting off the beginning of a 'long climb'.  Or I start
and just quit, which is much worse,'' I wrote in an email to AM.  I said that I
loved ``The Garden of Eden''.  It was November 7th. She responded that she was
laying in bed after driving for four hours, and was tired but awake.  I never
responded to her email, and we never talked again.

At the library, I started reading a pdf of ``The Secret History'', about a
student who is seduced by a group a wealthy classics students at a small liberal
arts college in Vermont.  I wanted to write seductively.  I stayed up the whole
night reading.  The only time I stopped was to get Snickers and Mountain Dew and
go stand in the glass coupula that overlooked Libe slope, and the city of
Ithaca.  On the gently sloping blackness of the hill across the lake I saw the
bright line made by a long country road, where individual cars were visible in
the morning.  It was like looking down from an airplane, a scene that contained
too much information, and I tried to make sense of the shifting geometry and
light but only felt a deep thrill of emotion.  I felt I was destined for
something.  

\section{}

A few days later, I hadn't really made progress on the essay.  What did I have
to show for the week? I woke up in the middle of the night with a intense desire
to shave my head, or half of it.  I lay awake in bed, sweating.  Somehow, I went
back to sleep.

\section{}

I would fall into a shallow sleep in dining halls or the library.  In
half-dreams I remembered the cold one felt when one swam just a meter or two
down into a lake or the ocean.  I remembered how it felt to grab another boy in
the water, or for someone to jump into the water on top of you.  In middle
school I had joined an Existentialism discussion board on MySpace.  I remembered
reading MySpace stories written by one member of the board that described
intentionally drowning someone weak at summer camp when he was a boy.  The story
described how the struggling vicim had suddenly released his bowels.  I wondered
if I had any deep dark secrets like that, secrets that I could never tell
anyone, even my wife someday.


\section{}

I started reading the copy of ``Foucault's Pendulum'' again.  The book had even
more references than Ulysses, almost: the references were almost made in a very
tounge-in-cheek way, which supported my growing feeling that literature was a
sort of massive, half-serious network of very old buzzwords and concepts.

Sometimes I realized my real life situation was bad.  Using the library
computers, I talked with old friends from middle school on facebook, who told me
that they believed in me.  

``Foucault's Pendulum'' led me to read a lot of Wikipedia articles while
listening to classical music.  I read about ``Pecca Fortiter'', the idea that
men should sin \textit{greatly} and without fear because God's grace was
infinite.  I read about the Church of the Holy Sepulchre, where a ladder had
stood against a wall since the 1800's because moving it would upset the status
quo between the Jews, Christians and Muslims.  I read about Euler's Identity:

\begin{equation}
  e^{i\pi} + 1 = 0
\end{equation}
      
Gauss is reported to have commented that if this formula was not immediately
obvious, the reader would never be a first-class mathematician, I read.  I spent
hours trying to understand it.  I would have to work hard.

Giordano Bruno had been intrigued by the Ars Memoria, a system of mnemonics
based on the Sepherot that allowed practicioners to systematize and remember
their knowledge.  I found old texts on the subject and began writing out a table
of the letters of the alphabet and the core concepts in my life that they
responded to.

My research into religion led me to read the Inferno, where I read about the 
\textit{relapsi}, those who realize the errors of their way, but continue.  They
were said to be doubly hated among sinners: hated by God, and hated by
themselves.


\section{}

I was spending a lot more time with the people from my floor.  It seemed like
they enjoyed hanging out with me now, amused by the things I was interested in.
I would stay up many nights with the floormates, talking.  I showed a prefrosh
the massive collection of art pngs I had torrented.  My computer got viruses,
and was booted from the network, so I had to go to the computer lab, where I
obsessively read articles about Tao Lin, or talked to a 23 year old I had met on
craigslist.  

``You know, in the beginning of the semester, I thought you were a scrub,'' said
Jonny, one of the athletic guys on the floor.  ``But you're like the opposite of
a scrub,'' he said.  

``What's a scrub?''  I asked, intuiting the meaning of the word instantly.

Jonny seemed embarassed.  ``You know, \textit{that} guy.  Because you like
didn't talk to any of us.''  

A bunch of us were sitting around, making ramen.  One of the Asian students was
cracking eggs into the boiling water.  It was a cold morning.  I had stayed up
the whole night before, and had finally finished ``Foucault's Pendulum''.  A
significant amount of snow had fallen in the dark.  I felt a sense of stillness
and doom, knowing that Thanksgiving break was next week.  I listened to
exclusively classical music now, a lot of Schubert and Ravel.  Morgan had given
me her gChat username and we occassionally talked late at night, mostly about
the stuff we were reading, and pop music.  She always signed off before me,
around 2AM.

I felt like I was impressing my floormates.  One of the girls told me she was an
English major.  I was surprised, she didn't look like an English major at all.
She had a soft face that reminded me of a seal or otter and talked like everyone
else from Westchester.  I showed her Mumford and Sons and played some of their
songs on guitar, and she came to my room a few days later and told me that she
was obsessed with them.  She stood in the doorway for a while talking to me,
causing me a vague sense of pain and unease.

On the phone at the library I told Jennifer about the English major girl, and
made fun of her for liking ``Mumford and Sons''.  I told her that there was no
one at Cornell for me, at least that I could find.  Jennifer said that there
were lots of guys that were attractive at Yale, at least from a distance, and I
felt a sense of unease.  I sent her a picture of the English major girl.
Jennifer told me that she had already left for Thanksgiving break, and was
staying at a friends house in New England, and had pimples.  Jennifer was
annoyed at me for always being gross about things.  I told her it was the
reality of the situation.  She was hiding in the bathroom to avoid being around
her friend, who had a boyfriend.  I knew them all from the summer program.  I
remembered one night near the end of the summer program Jennifer had stayed up
all night talking.  We lay on separate couches, talking about our dreams and
hopes, and our understandings of the world.  Before we stopped talking and fell
asleep I suddenly felt very aroused, more turned on than I probably had at any
point in my life.  I was confused: at the time I did not find Jennifer all that
attractive, or didn't think she was what I needed in a girlfriend.  

On the phone Jennifer told me that she was doing so many things because she
didn't want to be unlovable, and I didn't understand.  I told her that I
absolutely identified with wanting to feel lovable.  Jennifer wanted to be
objectively good and write in pretty ways.

The next day I woke up in the afternoon and ordered a book stand of Amazon.  I
wanted to deal with my neck and upper back pain, and be able to read more
without feeling tired.  My mother had commented that I was twitching my neck, it
was a sort of tic. 

I don't think I consciously thought much about my academic situation.  I talked
to my friend, a guy who had also dated Paola, about Deep Springs.  I had told
him to apply to the school, a free two year school where you worked on a Cattle
ranch and discussed philosophy.  It was very selective, and fed directly to
Harvard, Yale, and UofC.  I started telling people on my floor that I was
planning to transfer to Deep Springs.  I had until the new year to write the
application essays.

I bought Bulfinch's mythology, and read it, waiting for fall break.  I also
started downloading TV shows that I had read about online.  I downloaded ``Mad
Men'', and found most of the pilot episode pretty dumb and shallow.  But I felt
myself being moved by the last scene, where Don Draper held his wife and both of
them looked at their sleeping daughter, in a Connecticut bedroom that looked
like an oil painting.

\section{}

My mother drove Paola, Finn and I back to Buffalo.  She was in a good mood and
took us to eat at Moosewood.  After, we walked around the Brentwood Mall, which
isn't really a mall, it's the basement of an old institutional building.  It
smelled like fall in Ithaca, a hazy smell of upstate New York and water in
basements. 

I bought a beautiful embossed version of ``The Aeneid'' at the rare and used book
store. 

In the car, I could feel the stress of the past months affecting how I talked
to Mommy. I  wanted her to stop talking to Finn about his physics major.  I
felt that he'd definitely flake out of it soon.  The way Finn talked about his
physics major, it seemed very obvious that he wanted to impress people with it. 

My plan would work out better in the long run.

As we drove through the amber fields overlooking Cayuga Lake Mommy stopped
talking to Finn and Paola and quietly asked me how things were going.

``I think I really hate Cornell,'' I said.

``Your classes still aren't going well, huh?'' Mommy said. 

``They're going badly but it's not that.  It's the people.  I'm really bored and
so...disappointed.''

I could tell what I was saying was hurting her.  But I was sick of her
upbeatness.

``Patrick, where else would you go?  I think you're being unfair.  It'll get
better.  I remember how stressful it was.  I still have bad dreams sometimes
where I have a prelim and a bunch of homework I haven't turned in.''

``I don't think you understand.  I don't know if I can go back there.''

I said the communications classes were such bullshit and that I wished I was
studying something in the Arts and Sciences College.  

``You can't just blame us for your problems,'' Mommy hissed at me, even though Finn
and Paola were probably asleep in the back.

We drove by an abandoned house that I noticed every time we made the drive from
Buffalo to Ithaca, a dark farmhouse close to the road.  I wanted to
explore it someday.

When we got home I put the Aeneid on my book stand.  The Aeneid was good and
important to read.  Of arms and men.  I could hear Mommy and Daddy arguing
downstairs as they made dinner.   The golden light of childhood autumns filled
my room.

I was called down for dinner a few times until I heard anger.  I sat down at the
table feeling shaky and tired, and drank the milk.  It felt chalky in my mouth.
Avery made a face when she saw my hair.

We started arguing around the end of dinnner, and my sisters left.  We ended up
in our ``dining room''.

``You have to figure out what you can do to recover your classes, Patrick.  Talk
to your professors,'' said Daddy.

``I can't, I can't.'' 

``What do you mean 'you can't'?  You don't really have a choice.  College is an
endurance thing, you just have to get through this semester.''

``You can't take this for granted, Patrick.  Grandma and Grandpa have been saving
for a long time so you can go to Cornell and not the University at Buffalo.  We
thought you'd be happy, once you got to a place like Cornell, with challenging
classes and people,'' said Mommy, her voice self-pitying.

``Can we please not talk about money?  It doesn't help, right now.''

I paced around the dining room table.

``You guys don't understand.  I think I'm going to fail all of my classes, or at
least all but one or two, no matter what I do,'' I said, finally.

``What?''

``Great,'' said Mommy.

``There are things you can do, Patrick,'' said Daddy after a few seconds.

``No, there's not.  I've already tried.  It's too late.  I just couldn't write my essays.  I'm
sorry.''

``Then forget you essays!  Focus on the other classes.  Have you been going to
them?''

I felt the pain coming.

``Not since, October, maybe?  I was really focused on trying to write that essay
and was at the library all night.  I'm sorry.  It's a writing problem.  I think
I'm getting closer to figuring it out.''
 
``Your essays aren't the issue here, Patrick.  You need to drop that class and
figure out how to pass the other classes.''

``I'm going to fail them all, except for maybe one.  There's no chance.  I'm
almost sure.''

I could see tears coming to the edges of Mommy's eyes.

``Oh great,'' she said.  ``You really screwed up.''

``Yeah, basically.''

My father was silent for a period of time.

``You need to talk to your advisor,'' said Daddy.  ``And figure out how to drop the
courses.''

``It's past the drop deadline.  I'm just going to fail those classes.''

``No, that's not acceptable.''

``You're going to need to get a job,'' said Mommy.

``What?  No, I'm going back to try again.  Even if I fail my classes this semester.''

``No, you're not,'' said Daddy.  ``Not right now.  We can't let you go back like this.''

``I can't \textit{not} go back, okay?''

``I know...'' said Daddy, ``that you're afraid that if you leave, you won't ever go
back.  That makes sense.''

``I don't even want to go back, I'm so miserable there.  I need to figure out
this essay thing, though.''

``It's not essays I think, Patrick.  You need to understand that you won't always be the
best in your class.''

``I know that, stop saying that,'' I said.  ``That's not my problem.''

``I think you're not turning your essays in because you're afraid of not being
the best,'' said Mommy.

``I can't even really get them started.  I don't know why.  I've been trying
\textit{so hard}, I hope you understand that.''

``You're going to have to pay for this semester you wasted,'' Mommy said.

``Please stop talking about money!  It's not helping...me write my essays.
Cornell is so freaking stressful.''

``Tomorrow we need to call your advisor and figure out how to drop those
classes.''

``First, I have never talked to my advisor, and second, it's way past the end of
the drop period.''

``Why didn't you talk to your advisor?  That's just stupid, Patrick,'' said Mommy.

``Because he's in Communications, and I don't want to be a Communications major!
You have no idea how dumb my Communications class is.  It's like the opposite of
Legally Blonde.  I'm the only person with a black Windows computer surrounded by
a bunch of sorority girls and bros with MacBooks.''

They laughed.

``Look, I know that might not be the best match for you, but that doesn't mean
you don't need to talk to your advisor or go to class.  You're being stubborn
and can't blame us,'' said Mommy.

``We can call the CALs office tomorrow and figure out what to do.''

``He should call up La Scala's and see if he can get his job back,'' said Mommy to
Daddy.

I scoffed, and made my way up to my room.  I could hear my parents arguing
downstairs.

\section{}

The next day my parents called the College of Argiculture and Life Sciences.  I
stood at the head of the stairs in my underwear, listening to them talk on the
phone.  I wasn't sure why they hadn't made me make the call, but I was vaguely
grateful.  I was so sick of it all.  I shivered in bed at night, watching Mad
Men.  My sister had moved into my childhood bedroom and so all my stuff was
crammed into a small room with large windows that let in the cold.  
