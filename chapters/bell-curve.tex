\section{}
At some point during the summer before I started school, my mother went to
Target and bought me things for my dorm room.  She put the shopping bags into a
large laundry basket and left it outside my door.  The morning before we were
supposed to drive to Ithaca, she noticed that I hadn't packed the basket into
the car.  I tried to explain why I didn't want to bring all the stuff, and she
got increasingly upset.

The night before, I had set my Cornell netID password to ``10Nianhan'', a
Chinese idiom that translated to ``Ten Years by the Cold Window'', a period of
transformative, ascetic study.  I thought having posters and an ironing board
would distract me, or give people the wrong idea about my personality.  I didn't
really know how to communicate how I felt to my mom.

I was tired from spending the night reading Craigslist personals from the Ithaca
area.  I found someone looking for a ``literary correspondence'', and I paced
around the kitchen feeling excited and manic.  I took a picture of myself with
my childhood stuffed animal and sent it to her.  She also wanted respondents to
put a writer in the email's subject.

I chose James Agee.  One time, I wrote in the email, Agee and his wife were
having a fight at his office.  His wife stormed out of the room, leaving Agee
alone.  A minute later she ran back into the office and found the window open.
She screamed and pulled out tufts of her hair, leaning out of the window to look
for Agee's body on the street below.  A few moments later, Agee stepped out from
behind the door where he had been hiding.

I eventually agreed to bring some of the stuff from the laundry basket, because
I didn't want to fuck up the whole day.  I finished packing, took a shower, and
put on a red polo shirt from American Eagle.  My parents drove me to Ithaca.  It
was a beautiful, clear, cool day.  

\section{}

Moving in, I saw a pretty Indian girl with long hair.  I felt a stab of anxiety,
wondering if my dreams would be realized my over the next four years.  It was
time, I told myself.  I couldn't let things stay the same.

My parents and sisters helped me carry stuff up to my room, on the fourth floor
of one of the brick dorms that had been built in the seventies or eighties.
Both of my roommates had already moved in, but weren't present.  Over email, I
had volunteered to take the top bunk.

\section{}

I went to get registered at Barton Hall, and then bought textbooks at the Cornell
Store.  I bought all of the optional books for my Ulysses class, and my dad
smiled as I talked about my excitement for the class.  We talked about who would
pay for textbooks, he insisted that I didn't pay for the first semester's books.

My mom and sisters stayed the the Cornell Store picking out t-shirts while my
dad and I walked to the dining hall for dinner.  It was obvious that he was
going to lecture me.  

I knew that my dad had struggled at Cornell, to some degree.  He had taken a few
semesters off to work as the chef for his fraternity, and then met my mom at
some point.  They were both on the sailing team.

We got chocolate milk from the steel milk udders, and sat down at a table.
Daddy took a napkin and drew a bell curve on it.

``You know what this is, right?'' he asked, seeming a little embarrassed.

``A bell curve?'' I said.  I was amused.

``Yeah!  A bell curve.  I've been thinking a lot, what I should say to you.  I
guess college is pretty long and exhausting, and you've got to be ready for that.''  

``What does a bell curve have to do with it?''

``Well, I guess you're used to being around \textit{here} on the bell curve,
right?'' said Daddy, indicating somewhere in the top 10\%.  ``Or at least that's
where I was used to being, I think.''  

``Yeah, I guess,'' I said.  I didn't like where this lecture was going.

``Well, at Cornell, you might have to get used to being around here on the
curve, or even here.  Everyone at Cornell comes from the top end of the curve.
You have to learn to be okay with being down here on the curve sometimes.''

This is cliched stuff, I thought.  I felt a little angry and stressed.  I'm not
like you, Daddy, I thought.  You don't understand.  I'm not even really
\textit{on} the bell curve, in some way.  I don't care about the bell curve.
I'm sorry you did.

``Sure,'' I said.  ``I get it.  I've heard this stuff before.''

``As long as you go to class, you should be okay.  You have to go to class.  You
just have to.''

``Look, remember, I calculated the cost per hour of college during that boring
orientation thing earlier?  \$7.70 an hour, even when I'm asleep.  I'm sure the
cost of class hours is, like, insane.  I'm going to make the best use of my time
and go to class. Don't worry.''

``And make sure you go to the orientation stuff.  Mommy and I were talking about
this.  We're worried that you're not going to go to these events that are set up
for you.  Just go to them.''

``Okay, I will.''

I felt annoyed.  The summer sunset filled the dining hall with an intense orange
and red light.  I shielded my face as I made my way to the soft serve machine to
make a Sunday.

\section{}

I went through all of the orientation stuff.  I had no interest in spending
time with the group of Communications majors I was assigned to, so I drifted
between groups. 

Very suddenly around 10pm at night, we were set free.  We were all sitting in
groups on a large, dark lawn somewhere between the dorms on North Campus.  No
one was expecting me at home, I realized.  I could go where I wanted, and
explore as long as I wanted.  It reminded me of an large and detailed level in a
video game becoming open and available.

\section{}

I found Finn, a friend from high school, and started following him to
Collegetown.  A lot of other people were following Finn too.  It seemed to me
that Finn didn't really know where he was going, but he managed to chat with a
girl who was going to a party and just sort of co-opted that information.  

Finn used me as a way to create interesting conversation for the group.
``Steadman's crazy.  Tell them about the time you hitchhiked to Montana.  I
drove this fucking kid to the highway, and he just started hitchhiking there.''

``Wait whhat?'' someone said.  I nodded and smiled, making a face at Finn.  I
was secretly pleased to get some attention, but also a little uneasy.  I
answered some typical questions about running away from home and hitchhiking.
This happened a few times, and eventually we all started talking about what we
were going to study.  I was quiet. 

We passed through campus, occassionally seeing and hearing other students
walking in the same direction.  We crossed a stone bridge over a dark gorge.
Collegetown was full of loud students, drinking on the lawns of houses or
standing in the entrances to cafes and bars.

Finn's party was at a Jewish fraternity's annex.  It was crowded.  Finn and I
pushed our way in, and Finn got beers for the remaining people in our group.  We
stood by the beer pong game.  Finn was grinning.

Occassionaly we would talk with a girl, shouting, who always turned out to be
another freshmen.  I would feel optimistic about each girl we talked to, for a
minute or two, but then would worry that I had nothing to say and that the girl
didn't seem very interesting.  But Finn seemed happy and excited.  ``These are
going to be a great four years, man,'' said Finn.  ``We're going to kill it
here.''

\section{}

When I woke the next morning, I was sort of sad that I had walked home from the
party alone.  I checked my email, and saw that the Craigslist person had
responded: \\

\textit{
So I looked you up on the cornell website. are you a freshman? unfortunately, i
am much too old for you. go hit on some girls in your dorm or something. you've
been here for, what, less than a week?}

\textit{some sage advice from your new friend,}

\textit{alice munro}\\

I felt moderately annoyed, but it was another beautiful day, and I went to the
dining hall for breakfast, carrying a book.  

\section{}

I sat in the entrance to Goldwin-Smith hall with my copy of Ulysses, my
notebook, and annotations for Ulysses.  The sun was setting over of the arts
quad and the stage set up for the end-of-orientation week DJ. I continued
reading,  but I enjoyed the presence of the other students and the music
playing over the PA system.

Every line of Ulysses seemed to contain references to information that would
enable to become the sort of human being I wanted to be.  

I decided to join the crowd on the quad, and put my books into my backpack.  As
I walked down the marble stairs from Goldwin-Smith, I was stopped by two girls.
It was immediately obvious that something was up.

``Hey, can you take us to a bathroom,'' said a short tan Asian girl, probably
Korean.  I looked at her and her friend.  The Asian girl was sort of overweight.

I opened my mouth but for a second nothing came out; I had been in the library
all day.

``Sure, yeah,'' I said, making myself smile.  The other girl's hair was dyed in
a white streak down the center, making her look like a skunk.  I thought she
was cute.

I led the pair through the crowd.  The Asian girl walked next to me, bumping
her hips into mine. ``So you guys are freshmen?'' I said.  

``Noo,'' said the Asian girl.  ``We go to Ithaca college.  We're sophomores.''

``Oh, cool,`` I said.   I took them into the art library, and we found a
bathroom in the basement.  They went in, and I could clearly hear their
conversation.

``Oh my god, you always go for the nerdy ones!'' said skunk girl.

``He's cutee though,'' said the Asian. 

I waited in the hallway, which was completely white, like most art building
hallways were.

When they came out of the bathroom, I followed them outside, and we worked our
way into the crowd.  The DJs were on the stage.  They had something to do with
Mario and video game music.  The group that did ``Like a G6'' played next.  The
skunk girl had a plastic water bottle of vodka, half full.  She shared it with
the Asian girl but not with me.

The Asian girl stood in front of me and then started grinding her ass into me.
I thought back to stuff I'd read online about how the best degree at Ithaca
College was a Mrs. from Cornell.  It felt fucked up.  She yelled things into my
ear but I didn't really understand anything she said.  ``Do you want to hook up
with me?'' she yelled.  

I felt cold air that sometimes comes in a hot crowd like this, and felt the
sweat on my back.

I thought for a second.

``Uh not really,'' I said.  ``What?'' she yelled.  ``No,'' I said louder.

``Then why are you so \textit{hard},'' she said into my ear, running her hand
over my dick.  I felt like I was drunk even though I hadn't had anything to
drink.  I looked at the skunk girl, dancing with her eyes closed.

``Like a G6'' came on.  I danced with the Asian girl.  She was quiet.

Eventually I got to the point when I knew nothing would be gained by staying
with the two girls.  I said goodbye, laughing, and walked back to my dorm,
feeling bemused and happy. 

Over the next five years I often thought back on this event and wondered why I
didn't have sex with the Asian girl.  

\section{}

At some point during the first few week of classes, I met George.  Paola
introduced me to him.  Paola and I had been in the same Buffalo-wide ``Gifted
Math Program'', and had been my only girlfriend in high school.  When we
started dating she told me that I would get along with her friend George, but
we broke up before I had a chance to meet him. 

One evening, I went over to Paola's suite and uneasily chatted with her and her
roommates until George showed up.

Despite how reserved George was, and how in many ways he seemed to be a
``typical Asian'', I quickly felt that our fates were aligned and we became
friends.

\section{}

I was disappointed with the people in my dorm.  I tried to hang around in the
common room and meet someone interesting, but most of the guys were in the
Engineering school.  They were interested in solving Rubik's cubes or talking
about Pokemon.  Some of the other guys were interested in athletics and were in
the Judo club or went to the gym.  One time I tried doing an ``ab ripper X''
video with these guys but couldn't get through it.  It seemed the the girls
were not very interested in me.

I would message George and we would meet on the balcony outside the common room.
He would usually listen to me play guitar, and then we would talk about music
or listen to it.  I wasn't very good at guitar or singing.  I could only learn
one or two songs at a time, and had a bad sense of rhythm.  George had studied
piano as a kid, but when I tried to get him to play at the piano on the top
floor of the dorm he would silently attempt to play a piece of classical music
for about fifteen minutes, and then suddenly give up.  

\section{}

I took it upon myself to re-read the Odyssey before I started reading Ulysses.
It seemed like my only chance at understanding and originality was via
rigorously exploring of all of the dependencies of important pieces of
literature.  I was unsettled by my experience with the required reading book for
incoming freshman: ``Do Androids Dream of Electric Sheep?'' There was an essay
contest and a reading group.  I had put many hours and hours into my essay, and
had to run through the rain to turn it in, late to the group discussion.  But
listening to my peers talk about the book, which I loved, I felt confused and
defeated.  In the back of my head, I knew that the essay wasn't very good.

The night before the first day of classes, I posted an excerpt from the Odyssey
on Facebook: ``I will leap out of bed, sling a sharp sword over my shoulder,
strap a stout pair of boots onto my lissom feet, and go forth from my chamber
like a young god to face 8am calc.''

The Ulysses class met in one of the classic seminar classrooms in
Goldwin-Smith.  These classrooms had windows the size of a dining room table.
I was the only freshman in the class.  Most of the other students were
attractive: there were upperclassmen girls with golden jewlery, beautiful
sweaters, and slim bodies.  There was a handsome Irish student and a
crazy-looking woman with a hunchback. 

I considered joining an A Cappela group or the sailing team.

I was also in Advanced-Intermediate Chinese, a class on Globalization, and a
media Communications Class.  I was a Communications major.  I had fallen apart
writing my application essays alone on a foreign exchange in Taiwan.  I chose
to apply for the Communications major because I imagined running a
multinational media company.  Cornell was the only school I got into, besides my 
safety school.

\section{}

I liked to talk to George about my schemes for college.  It seemed like George
had a quiet interest in my more weird or risky ideas.  For example, in high
school the Internet had inspired me to go urban exploring.  I used urban
exploring as a way to get people to hang out with me outside of school.  We
explored the basement of our high school, and then abandoned factories,
theaters, and hospitals.  I dreamed of exploring Cornell's tunnels and rooftops
until I found a place to build a secret room where people could meet.  I
imagined it would be a grimy version of the ``secret'' rooms that exclusive
societies like Quill and Dagger had. 

George helped me look for a suitable place to build this room.  We went out in
the warm nights of mid September, when the quads were still perfectly green but
empty.  George watched my back while I opened hatches that looked promising.  I
was impressed by his bravery and calm.  We didn't have much luck finding a good
place.  The hatches led into dark, tiny spaces filled with loud machinery.  I
promised him that we'd eventually find a good place, that it was like
hitchhiking: the ride always came when you least expected it.

\section{}

The first week of classes, Seth texted me, asking if I wanted to have lunch.
Seth was a ``nerdy'' friend from high school, who was planning to be a doctor.
I was surprised when I found Seth sitting with a bunch of good-looking,
articulate people. They had met during orientation, playing ultimate frisbee.  I
was instantly attracted to Morgan, who was tall, brunette, and classically good
looking in a somewhat masculine way.  She was from Virginia. 

I made a point to eat lunch at the same time and place, so I could join their
table.  We sat in the back of Okensheild's, the old faux-gothic dining hall
overlooking the slope.  Morgan showed me Nicki Manaj, saying she was the best
female rapper since Lil' Kim, and would happily discuss all of the stuff I was
researching as part of reading Ulysses, like Gematria and manichaeism.

\section{}

I decided to stay in Ithaca for the Labor Day long weekend.  I went for a jog on
the gravel path around Beebee lake.  Strange birds stood in the water, and I
stopped to look at a group of turtles.  I walked back to my dorm, feeling a
sense of well-being, depsite the confusion of the first and a half of classes.
Mommy called me on my silver Motorola RAZR flip phone, my dad's old phone.  I
told her things were going pretty well. 

\section{}

In the middle of September, George and I went out together on a Saturday night.
A bunch of people from my floor were trying to go to a frat party, and I figured
George and I could go on our own, outside of their group.  

``Well, it'll either be intersting and worth going to and we'll meet people,''
I said to George, ``or it will be awful and embarassing, and we'll feel
superior and learn a lesson...about how things don't work.''  I was trying to
get him to think that going out would be a good idea.  He eventually agreed to
go to the party.

George had a new Android phone with Google maps, and looked up the address.
When we arrived, there was a long line up the hill leading to the house.  I saw
my floormates somewhere ahead of us in line.  It took at least an hour to get
to the front of the line.  It hurt my pride to have to wait like everyone else,
and watch drunk people making friends or talking.  George and I talked about
his rhetoric class, and the strange, unique rhetoric of the period right after
9/11.
 
The fraternity brothers at entrance to the house yelled at everyone.  George and
I were silent.  The brothers kept letting people in ahead of us.  One guy yelled
``get...the fuck...back'' over and over again, and his friend mutterred
``fucking faggots'' under his breath.  I loathed the overly serious faces of the
other students waiting in line with me, pushing towards the door.

Eventually someone from my floor yelled my name from inside the house, and the
brothers waved George and I in.  I followed the people from my floor up the
crowded stairs into a blacklit room, and did a few shots.  After a while, George
and I sat on a couch, watching the people from my floor laughing and talking
with a brother.  It was too loud to hear what they were saying.

The next day, George and I met up to study in the Graduate English Lounge, which
I had told him about.  I liked to story there, partly because undergraduates
technically werent allowed inside, and partly because the room had a large
collection of classic ``dry'' literary books that were interesting to pick up
and look at.  George worked on his readings for his rhetoric class, which we
talked about sometimes.   Most were texts from Aristotle and other rhetoricians. 

When it got dark, we gradually stopped working on school work and started
showing each other websites.  I showed him trailers for some of my favorite
Chinese movies and he told me that he couldn't read Chinese characters, but
wrote his name out for me.  George showed me some of the blogs where he read
about music, and we talked about how having good music taste was basically just
reading the right blogs now.  Before we walked back to North Campus, I showed
him Hipster Runoff, my favorite website.  I made him read some of my favorite
articles, like ``U, me, and every relevant concert we attend'' and ``My
job/career does not align with my true personal brand''.  I watched his face.
We talked about it on the walk home, and I felt happy.

\section{}

My first essay was due in class on September 20th, Monday.  I had spent most
of the previous week thinking about Morgan and reading ``The Unbearable
Lightness of Being'', which she had recommended.  On Friday, I realized I had
the weekend to finish the essays.  I went to bed early.

I didn't use my time very well.  I woke around 11am on Saturday morning, and
felt a sense of luxury and anxiety at the thought of all of the reading and
writing I had to do.  I assembled all of the things I needed and went to the
Appel dining hall, the dining hall that I didn't normally go to.  It was at the
highest point on north campus, behind a large soccer field.

Breakfast was my favorite meal.  I enjoyed having a glass of Mountain Dew with
my cereal, and eating lots of warm muffins and pastries.  As I was waiting in
lne for bacon and french fries, I ran into Finn.  I was very much in my own head
after a week of reading, and it felt weird talking to Finn suddenly.

``Steadmaan, how's it going?  Come upstairs and have breakfast with us.''  I
joined Finn and a surprisingly large group of people he knew.  They had all gone
out the night before.  Most of the people were the very forgettable type of
Cornellian I had started to expect: a sort of heavy, dirty blonde, somewhat
athletic male or female from the tri-state area.  They always wore t-shirts from
their former high school or athletic team.

I was instantly interested in Marie, though.  Finn was obviously interested in
her too, and treated her like an old friend.  She was wearing a black dress and
a leather jacket.  Finn told me Marie was an architect student, from Romania.  I
asked her what she thought of brutalist architecture and Cornell.  Ian continued
making jokes involving sex, physics or stuff we talked about in high school.  In
these situations, I always felt that the type of person I was interested in
would understand my reserve and silence as a sign of my character.  I imagined
Marie going back to her dorm room and thinking of my calmly amused face, my
reaction to Finn's humor.  

I spent a while talking with them.  Finn invited me to go out with him and Marie
that evening.  As an afterthought, he invited everyone else at the table, who
had been talking amongst themselves about other things whole time.  I told Finn
that I had to work on my essay, but I'd text him.

\section{}

I started reading the Communications textbook from the begninning.  I looked at
the syllabus and made a list of all the chapters I had to read before starting
the essay.  The book started out with an overview of pyschology, sociology and
philosophy.  I started taking notes, skeptically.  The stuff in class was all
very obvious, disappointingly obvious.  Obviously the Internet was changing how
people intereacted.  Obviously people perceived their friends differently now
that everyone was on social networks.  The stuff in the textbook was more
interesting, because the textbook referenced the names an theories of past
thinkers, names and ideas that might be useful to me.  I started writing them
down. 

I felt like I was clay that had been sitting in the sun.  I was 18.  Life
experiences had given me a bit of a hard shell.  I thought of how I had run away
from home when I was 16, how my high school crush had rejected me in middle
school.  But my core was pliable, and reading the right things would shape it.
The hard crust of me would allow me to hold shape for a while, and do things
like accomplish my goals or finish essays.  I was heavy like clay.

I started fitting two lines of text into each line of my notebook.  I didn't
want the Communications notes to take up all the space that I needed for
Ulysses.  

Some of the concepts from psychology were interesting to me.  I was excited when
I read about the concept of the Fundamental Attribution Error: the tendency of
people to attribute failure to personality or character, instead of the
``external factors'' of life.  I thought back to riding the train home when I
was an exchange student in Taipei.  One day, I had noticed someone else from the
high school I was attending.  She was staring at a Patrick Star (from Spongebob)
keychain made out of colored plastic circles that had been melted together with
an iron.  She stared at the keychain for almost twenty minutes, her face sad and
empty.  She was a little shorter and chubbier than the average Taiwanese high
school student, and was wearing the same purple uniform that I was wearing.

I wondered if she was experiencing intense, painful understanding of her
situation as a citizen of the second world nation of Taiwan, in between the
glossy promise of the first world and the hope and misery of the third world.  I
imagined that she had a noble personality, but simply couldn't do anything about
it, and was stuck attaching significance to keychains of Patrick Star, the
stupid starfish.  I thought about the the things that I thought were important,
my own personal history, and wondered if they might be equally void.

As the sun set, I had finished two chapters, and taken a few pages of notes.  I
texted Finn, and walked back to my dorm to shower.  I felt excited, and thought
over the stuff I had read.  I felt happy that I could learn even in my stupid
Communications class.  Maybe it would be an okay major.

\section{}

We went to a party at my dad's fraternity, Rockledge.  

``Didn't your dad build these tables and stuff?'' yelled Finn, referring to the
wooden booth we were sitting in, drinking.  It was loud and dark.  Marie sat by
Ian's side.  

``Yeah.  I think he spent like a whole semester on them.''

We went to get drinks and I got separated from Finn and Marie.  I was sort of
drunk and danced with a few girls, one who was attractive.  I hoped I would get
a chance to make out with one of them, but the song would end and one of us
would drift away for another drink, or smile and step away, pushing through the
crowded basement.  

What is the force that keeps people apart, I wondered, thinking back to my
readings.  Socio-psychological vs socio-cultural factors.  

Finn grabbed me, and pulled me upstairs.  We sat on the balcony for a bit.  Finn
was even drunker than I.  Rockledge sat at the edge of a gorge, and it was at
least a one hundred foot drop from the balcony down to the water and rocks
below.  It was a beautiful night, and I felt dazed and calmed by the sounds of
the music in the basement, the water falling over the waterfall at the end of
the gorge, and rich young voices yelling to each other somewhere in the dark.  I
could see across to the other side of the gorge, where small figures walked in
groups to parties.

Finn led us through west campus to another party.  As we walked I talked with
Marie about Romania.  I had a vague idea of the former dictator and his wife, I
had watched a video of them being shot against a wall.  I remembered that there
was a lot of yellow in that video.  

``Ceausescu,'' said Marie.  ``They recently dug up his body to do DNA tests.''  

``That's interesting,'' I said.  ``It's weird that we were born just a few years
after that.''  

``Yeah, my family is obsessed with it.  I came here when I was five, so I don't
know.''

At the party, I reluctantly agreed to play beer pong against Finn and Marie.  My
partner was a fraternity brother with a doughy face and body.  Beer pong made me
anxious, even though I was drunk.  Marie was getting louder and louder, and
screamed every time she made a shot, hugging Finn.

Eventually we left, and made our way back to North Campus.  We were going to
meet some people at Finn's dorm.  It was a long walk, over bridges and along
steep windy paths with wooden steps.

I noticed that Finn was behaving sort of strangely.  He was always a forceful
drunk, but now he seemed sort of angry.  He was talking about physics and
engineering, which always annoyed me when he was drunk.  Marie was arguing with
him.  He said that architecture was inferior to engineering, and repeatedly
referred to a building that had collapsed because architects had added extra
features to the building without doing the proper calculations.

``You architects could learn these things, but you \textit{don't}.  Architects
are too lazy to learn \textit{science},'' said Finn.  It seemed like Marie was
about to cry, or become very angry.

``That's not true!  We do learn how to calculate load and stuff!''

``When.  What class?  You don't take any physics classes.''  Finn was walking
quickly, and Marie struggled to keep up in her high heels.

``We still learn how, though.  There's a lot of math.''

``There's more to designing buildings than calculating load, Finn.  And computer
simulations can handle most of the structural stuff, right?'' I said.

``That's fucking lazy.  Marie, what would tell the parents and kids of the
people that die if your building collapsed?  Are you going to trust a computer
program?'' 

``Ugh,'' said Marie, and fell back from Finn.  Finn seemed to be breathing
heavily.  I wondered if he was going to fall apart during college.  

When we got back to Finn's dorm, there was just the typical stuff going on.
Someone was drunkenly playing the piano, songs they'd obviously learned as part
of their high school music program.  People were running around.  Marie left,
and I left soon after.  

\section{}

The next day I kept taking notes.

You don't \textit{have} to \textit{want} to write, I wrote.

I fantasized about getting a little drunk with someone and going to the library
and picking up random books and bringing them back to the graduate english
lounge to read.  

``Surrounded by deniers, he must deny them,'' I quoted.

I started writing in my notebook about what I'd need to do, to be able to write
this essay.  I'd done enough reading.  It was due the next morning in class, and
I still had a few chapters left to read.  I had enough notes though.  I looked
at the essay prompts.

If I was going to write this essay, I'd need to be able to bullshit, but in an
authentic way, like the articles I was reading on JSTOR.  To bullshit
effectively, productively, and easily (which I must learn to do, I wrote) I
needed a weighty, diverse, mass of fermenting mental matter in the sea of my
mind.  I needed facts, theories, jokes, friends, stories, personal issues,
obsessions (past and present), magic words, fancy sentences, quotes,
frameworks, trivia, lies, references, etc.  I also needed a prose style, a
persona, that would be easy to slip into, that wouldn't alientate a reader or
listnerer.  I would also need an intuition to where the essay would go, a sense
of structure...

\section{}

The night after the Communications essay was due, I left the dining hall and
started walking towards the woods, carrying my guitar.  I knew it wasn't a very
good use of my time, and that I should just be working on the essay, but I felt
that there was something in me that had to come out.  I walked on the boardwalk
down into the small valley containing the river that fed into Beebee lake.  Many
of the leaves had fallen, so that there were only pines in the electric lights.

I sat on a stone bridge and tried to make a up song for a while, singing to
myself.  When I got tired of that, I started thinking, trying to figure things
out.  What should I focus on, I asked myself.  What was I going to study in
college?  Obviously things weren't working out so well.  I had to focus.

I tried to rule things out, based on my personality.  I wanted to be creative.
I wanted to be able to show my work, and know it was good.  What if I failed?
What would failure mean in different careers?  I was fighting the feeling that
ultimately failure was my fate.  That was defeatist, what my dad would call ``a
self-fulfilling prophecy''.  

But if I studied business and became a businessman, maybe a startup
entreprenuer, failure would mean that all I'd have would be maybe a powerpoint
deck of my ideas.  If I was a writer or artist, at least I'd have my art.  What
else was there?  Being a businessman seemed like the worst thing to fail at.  I
repeated to myself: you don't want to just have a powerpoint.  You need a skill.
At least I have Chinese, I told myself.  I went to the library the next day and
checked out a few contemporary Chinese novels.

\section{}

I sat at a small desk in the Olin library stacks, listening to a piece of music
my high school friend had sent me.  Brian had been arrested in 10th grade for
``hacking'' the school's database, which included social security numbers.  We
had become friends when I was sent to the Harkness Center for five days of in
school suspension because I had installed a port scanner on a school computer.
Brian was stuck there for a year.  We both agreed it was a wonderful place, that
it was great to just read in peace.  Brian was also placed in the Gifted Math
Program, and joined my carpool, which raised eyebrows.

Brian and I liked the same kind of music, and some of my best memories of high
school were going into the swamp behind our high school with some of his very
fucked up friends and playing guitar for them.  In the fall of 2008, Animal
Collective had concvinced us that some type of pop music would soon destroy
indie music, or that indie music would become pop music, and that not many
people would be into lo-fi folk and laptop music, like us.  Brian emailed me
while I was in Taiwan, telling me what college was like and writing about his
girlfriend and drugs.

The music he sent me included a clip of myself talking.  I couldn't tell if I
sounded dumb.  I normally hated my own voice, but the music itself was pretty
beautiful.  I felt excited.  It was happening: my friends were making
\textit{good art}.  I would make good art too.  The email talked about the music
collective he was making, and how he wanted his parents to give him old
computers for his birthday instead of clothes.

It was very painful, trying to write the essay.  I listened to Crystal Castles,
to try to get pumped up, but it just made me more antsy.  I felt the sudden
realization that September was almost over.  I realized that Cornell was the new
normal for me, and would be for the next four years.  I had crossed some sort of
Rubicon: I had learned the meaning of the Rubicon from a reference in Ulysses.
I listened to the Green Day song ``Wake Me Up When September Ends'' over and
over again, watching the video for the first time.  Suburban High School life,
which I had always felt very annoyed and bored with, now seemed like a deeply
fraught emotional landscape.  I chose the essay prompt that involved the
difference between public and private personas.  Persona came from the Greek
word for ``mask'', which broke down into ``per'': for, and ``sona'': ``sound''.
Greek actors didn't have electricity, so they used masks that amplified their
voices.  I would write an essay about how we used personas in order to make
ourselves more effective in life, and get what we want.

At 2AM, Olin library closed, and I went over to Uris library, which was open all
night.  I felt a sense of despair, but told myself that I'd stay up all night,
and get the essay done.  I've got to fuck myself up, to be a special person, I
thought.  It's not going to be easy.  I've got to do great things, in order to
be great.  It's not just going to happen.

I sat in the ``fishbowl'' area of Uris, an area that my father had pointed out
to me once.  I explored the building until I found the vending machines, up near
the clock tower, and bought peanut MnMs and Mountain Dew.  Around 2:30AM a loud
alarm bell rang, causing my stomach to leap into my throat.  A man and woman in
grey suits who smelled strongly of cigarrettes and something else came around,
checking that everyone had Cornell IDs.  Then it was silent.  I slowly made
progress on the essay, spending most of my time on Wikipedia.  By 4:40AM, I had it
mostly done, and fell asleep in an armchair.  I had to finish that essay, and
then write two more.  We had to address three prompts in total.  But hopefully
the essay was very good.  I wondered how many points I'd lose for it being late.

The sound of bells woke me up.  My neck hurt and my eyes were gummy.  It was
around 7:30AM, and cold outside.  People were walking to class.  I thought it
was really beautiful.  I went back to Olin library to get a Buffalo Chicken
wrap and another Mountain Dew.  I read an article on Thought Catalog by Bebe
Zeva, who I was amazed to realize was actually younger than me.  It's happening,
I thought.  I've got to write.  It's happening.

\section{}

The Communications essay made me realize that I was struggling in college, too.
But I felt hopeful. I was learning properly now.  I let myself have a  day or
two to recover, after I finished my essay.  I \textit{had} finished it, after
all.  Late at night, I lay awake, loving the feeling of the window open a crack,
the sound of rain, the heater being on.  I felt alone, on my top bunk.  It was
the end of September, and I had suddenly realized that 2010 was almost over.  My
host mother in Taiwan had given me the planner; it was a reward for buying ten
boxes of donuts at Mister Donut.  My host mother bought me more donuts than I
even wanted.  I loved the planner, it had pictures of cartoon loins with donuts
as manes.  It had inpsired me to try to keep a daily planner.  WHen I started
using it, right before the beginning of 2010, I had imagine the decade of the
'10s as ``my decade''.  It was the decade I would have to be successful in, or
else.  I'd be 28 at the end of the decade.  1/10th of that decade was now almost
done.

The next morning I slept in and then went to have lunch at Okensheilds.  I took
a nap in the reading room, which was always warm and seemingly low on oxygen.
When I woke up I didn't feel bad.  Above me on the bookshelf was a copy of
Nixon's ``Five Crises''.  ``Life for everyone is a series of crises,'' it began.
I already liked Nixon.  I related to the phrases he used, ``derilict in one's
duty'', ``a marred life'', ``taking one's job but not oneself seriously.''  I
wrote down quotes into my notebook.

I went to the one class I hadn't missed, and then back to my dorm room.  It was
still raining.  Facebook somehow led me to Omegle.  I talked to a few people.
One was another young man, who seemed a lot like me but less advanced, or
educated.  We exchanged gChat usernames and talked a few weeks later, late at
night.  I also talked to a girl who eventually told me that she had cancer.  I
added her on skype.  Cancer reminded me of a Lorrie Moore story I had read about
a woman who got cancer and then decided to kill herself.  I told the Omegle girl
that I had a story she might be interested in, and went to the library to find a
copy of the book.  I checked it out, and started the walk back to my dorm.  It
was already late at night, and I felt manic.  On the bridge over the gorge, I
saw an Asian woman using a flashlight to look at something attached to the green
steel of the bridge.  I realized she was looking at clumps of spiders and taking
notes on them.  This was Cornell.

I found the story and typed it up on tumblr for the girl with cancer.  I
considered the fact that she might not actually have cancer, but decided that if
she didn't actually have cancer, she was still really fucked up for pretending
to have cancer on Omegle.  Her skype user photo was very homely, and looked like
someone with cancer from the midwest, so I was inclined to believe her.  I fell
asleep thinking about being one of the many people who had jumped off the green
bridge over the gorge.  I wouldn't want to do it unless I had something to leave
behind, like my laptop with a manuscript saved to the desktop.  If I had that, I
could do it.

\section{}

I got into the habit of doing homework in the laundry room of my dorm.  I liked
the big plastic table, the white noise, and the warmth.  I did a lot of calculus
homework there, or did readings for my classes classes.  I also read most of Tao
Lin's ``Richard Yates'' there.  

``Richard Yates'' had recently been published.  Conversations with George had
inspired me to actually read it.  George and I wanted to apply to Telluride
House, a ``intellectual community'' that had provided free housing to
neo-conservative thinkers like Allan Bloom, Francis Fukuyama, and Paul
Wolfowitz, back in the early 1960's.  It was now supposed to be much more
progressive.  George and I were already becoming reliant on each other for
discussion of the ideas most dear to us, like literature on the internet, ``the
classics'', and Hipster Runoff. 

The Telluride House essay required five essays.  George and I stayed in the
graduate English lounge late on the days leading up to the deadline.  I wasn't
making very much progress on my essays.  I wrote one that described the sounds
of the beginning of a class in middle or elementary school: the unzipping of
binders, etc.  I showed it to George and he sort of smiled and asked if it was
supposed to be like James Joyce.  I felt discouraged. 

George didn't know what to write about.  I was procrastinating by looking at
stuff on Hipster Runoff and asked if George had heard of Tao Lin.  ``Of
course,'' said George, surprising me.  ``Maybe you should write your essay about
Tao Lin,'' I said.  ``But I haven't read any of his books,'' said George.  ``So?
That's perfect.''

While George worked on his essays, I read everything that I could find online
about Tao Lin, or written by Tao Lin.  I had become aware of  Tao Lin when I was
living in Taipei as a foreign exchange student.  During winter break I was
living with a nervous old woman in the misty, hilly northern part of the city.
I was very isolated, and my mental health rapidly deteriorated as I tried to
write my college application essays before New Years, when they were due.  It
would be the start of a new decade, hopefully my decade.

I became obsessed with Hipster Runoff, feeling that it somehow cut closer to the
heart of my life than anything else.  I first felt contempt for the excerpts of
``Shoplifting from American Apparel'' that I read, but I realized that some of his
writing accurately captured some of the strange feelings I was having in Taipei,
feelings that I had thought were unique.

I told that George that my essays were ``almost finished'', not wanting to
discourage him.  At midnight, I sent in an application that consisted only of
blank essays.  I told myself that I would finish them all the next day and
pretend that I had forgetten to attach them.  George seemed exhausted, and not
particularly happy with his essays.  Jokingly, I ordered a copy of ``Richard
Yates'' from Amazon.

\section{}

Sometimes I got dinner with Morgan and David on North Campus.  It was
disconcerting spending time with them in the dark, surrounded by freshman, in
the modern dining halls.  David would often come to dinner after rowing practice
with some of his teammates.  David had joined the lightweight rowing team.  I
was sort of annoyed by how Morgan would go along with the jokes of David and his
teammates.  The jokes were pretty funny, usually repetively referencing pop
culture in some way.  Morgan would just draw these jokes on and on, and I would
occassionally join in, but my heart wasn't into it.  I vaguely wondered what it
was like, doing something like rowing.  I knew that the rowing machine was very
difficult: my dad had bought a rowing machine when I was in 7th grade,
and had told me that I'd be allowed to use his Laser sailboat when I could row
5000 meters faster than my mom.  I remembered turning on the radio and rowing
for a long time, until I broke 10,000 meters, and how my arms had developed hard
muscle after doing this for a few days.  I also remembered how the erg was next
to my parent's little bookshelf, and when I got bored of rowing I'd sit down and
read ``The Joy of Boys'' or other parenting books, and think about how weird it
was that my parents thought of me this way, like every other boy, turning into a
young man, with sexual feelings and a personality.  

\section{}

I took my first Calculus prelim.  I was surprised at how hard it was.  I mostly
understood the material, I thought, but I was having more and more trouble
paying attention in lecture.  I got a score in the 70s, which was the mean for
the test.  I felt moderately angry, and promised myself that I'd study harder
next time.

\section{}

I had to go back to Buffalo for Fall Break, because my uncle was getting
married.  I got a ride back with Paola and George.  Paola's mom picked us up and
took us to lunch at the hotel on campus, part of the hotel administration
school.  ``My mom still really likes you,'' said Paola as we waited for her mom
in the hotel.  I was properly dressed, and felt good.  

The morning after I returned to campus after break, my first essay in the
Ulysses class was due.  I felt like I'd be able to finish it over break.  The
wedding was in the hills of Virginia near Camp David, and I would have lots of
time in the car to read and write.  I had learned my lesson from the
Communications essay.  More importantly, I thought, I actually cared about the
Ulysses class.  I never missed it, and the professor seemed to really like what
I said in class.  I found the other students in the class interesting.  The most
interesting other student was a woman with a strange, hunched back and gravelly
voice named Keri.  She often wore camoflauge, and seemed like she shouldn't be
allowed in the beautiful old classroom.  I thought her contributions to the
discussion were pretty decent thouhg, and was intrigued.  One time, she saw my
notebook and got very excited.  I explained how it was color-coded based on the
type of information I was writing down: black for my own thoughts, green for
quotes, blue for facts and things to look up, and red for important things.  She
cooed over it for a while, and I started to feel strangely attacted to her.

I enjoyed my turkey sandwich with apple and mustard, and felt confident talking
to Paola's mother.  I napped during most of the ride back to Buffalo.  

Later that night, my mom drove Paola, George and I to the movie theater to see
``The Social Network''.  It felt strange and calming to be back in my hometown,
away from the buildings and trees that I saw every day at Cornell.  

We all agreed it was a really good movie.  I felt strongly affected by the
scenes of Harvard at night, with students walking quickly to and from buildings
in a type of lighting that I couldn't quite put into words.  It seemed to
capture a type of greatness or importantance that I cared deeply about.  I also
remembered the rowing race scene, showing a type of struggle that was almost
removed from time, removed from ``business''.  I felt stressed about the fact
that I was now in college, and needed to achieve the same things that Mark
Zuckerberg had, and meet the people that I needed to meet in order to make
things happen.  I needed the leverage that he had, the ability to program or
create.  Could writing even be like that?  It was hard to tell if writing was
good; it was hard to get people to respect or admire writing.

In the drive down to Virginia for the wedding I mostly slept.  I thought of all
the car rides growing up, when I had promised to read or write or think about
something, but only stared out the window and thought about myself.  We stayed
at a motel, and I tried to do some of the reading.  It was weird spending time
with relatives that I'd never met before, my dad's side of the family.  Many
were very similar to us, very awkward.  I could see myself in them.  I felt very
aware of my body at the wedding ceremony, which was outside at the edge of a
lake.  I had spent as much time as possible with my book, telling people that
college was busy and hard work.  We went to a brunch the next morning at a lodge
in the woods near Camp David, and then drove back to New York.

\section{}

My parents dropped me off back at Ithaca.  I went straight to the library to
start writing.

I had my thesis: that James Joyce had been heavily inspired by Bruno, and that
Bruno's thought had informed many of the most humanistic elements of the novel's
structure, like the last chapter from Molly Bloom's perspective, where her
nighttime thoughts mirror her husbands, closing the loop of love.

I wrote down the things I needed to do for my parents: 1) get good grades, 2)
don't die/suicide, 3) don't hurt other people, 4) support self, 5) get married,
6) talk to people, 7) eat sleep well.  I would be calm, tolerant, but not
indifferent.  It's going to hurt, I told myself.  That was the consequence of
fucking around all weekend.


\section{}

I stayed up most of the night trying to write the Ulysses paper.   Around 3am, I
subconsciously gave up, but spent time reading on the internet for a few more
hours until going to sleep in disgust.  I slept through my calculus class but
made myself go to the Ulysses class, wearing a green polo shirt.  I smiled at
myself in the mirror.  I had only a few paragraphs written.  It was a Monday,
and I figured I'd be able to turn it in before the next class, and just take
some late points.  I didn't talk to the professor, and slipped out while
eveyrone else was turning in their essays.  

I felt strange the rest of the day.  I talked brightly with Morgan and David at
lunch, and worked on a crossword with them.  Everything that happened felt
critical.  I had two full days to finish the essay.  I would research Giordano
Bruno and turn in a paper that made original contributions to the study of
Ulysses. The quality I could add in the next two days would more than make up
for the lateness.  

\section{} 

One time I was working in the laundry room and Paola texted me, wondering what I
was doing.  I invited her over.  I wondered if she still felt attracted to me.
I high school relationship had collapsed simply due to awkwardness.  Maybe
things had changed.  

It was one of the first dark, cold nights.  Paola seemed happy to see me.  We
talked about our school work.  It seemed like the first month of college had
changed Paola.  I felt like I was still the same.  One of the things that had
made we want to break up with Paola was the fact that she was almost constantly
cheerful and bubbly, but her words now seemed to be heavier, as she spoke about
how difficult her statistical methods class was.  She was considering changing
her major, or studying abroad.  I said that calculus was hard too. 

``I'm really tired,'' she said, and put her head down on the folding table.

``This is a tricky problem set,'' I said, looking down at the book.  We were
silent in a way that reminded me of silences that sometime happen when drinking
and talking with friends. Then I realized that Paola was making crying sounds.
For a second I thought it might a noise from one of the laundry machines, but I
listened harder until I felt sure that it was from Paola.  I wondered if she
was going to tell me what was going on, or if she knew that I could hear her.
I felt paralyzed and disturbed.  

I imagined patting her back, hugging her, or saying something, but I couldn't
actually make myself do these things, because of some sort of strange
embarassment.  I felt ridiculous and weak, pretending that I was oblivious.

I hear her lift her head from her arms, and straighten herself.

``I think I'm going to go back to my dorm,'' she said in flat voice.  I looked
over in her direction, not meeting her eyes.   

``Oh yeah, good luck.'' I said.

``Good luck,'' she said on her way out the door.


\section{}

By the middle of October, I was spending almost every evening in the graduate
English lounge with George.  I knew I had a problem with procrastinating on the
internet, so I left my laptop at home.  But I just started using the library
search terminals, old Dell Inspirons that I had to stand up to use.  

Using a library computer, I ordered a set of lockpicks from a PayPal shop, and
showed the excitedly to George.  I thought that the lockpicks would reignite our
stalled ``secret room'' plan.  I practiced using the lockpicks on my own dorm
room door.  When I finally was able to lock and unlock the door, I excitedly
showed my new skill to the floormates.  I 

On Craigslist, I saw a personal that was obviously posted by ``Alice Munro'',
the person who had responded to my James Agee email saying that I was too old
and that I should hit on girls in my dorm.  I sent another email to her, but
using the email I had created to buy the lockpicks with, instead of my Cornell
email.

When I recieved a response, and sent a few emails back and forth with her until
she sent me a picture unprompted, I showed the emails excitedly to George.  She
was 23 year old Cornell grad, living in Trumansburg, a small town about 25 miles
away from Ithaca.  I imagined the five years of time between me and her, all of
the experiences she must have had.

George seem bemused, a little worried, maybe disgusted.  I felt thrilled.  It
was breakfast for dinner night, and the sun was now setting around dinnertime
instead of at 9pm, like in the childhood summers.  The sun filled Okensheilds,
and I had to shield my eyes as I made my way back to where George was sitting,
carrying pancakes, eggs, and sausage. 

te I would go out with either people from my floor, or Bebe Zeva crystle
castles, uffie, ableton

\section{}

Finally Morgan and I were studying together, like I studied with George.  I
fantaized about her joining George and I, and three three of us becoming a
dependable,  growing intellectual group.  Morgan and I studied on the top floor
of Uris library, where the sound of the belltower was noticably louder, and the
sound of the wind often made me sad.  I was starting to do research for my
midterm essay in the Ulysses class.  

On a particularly dark October day, Morgan fell asleep, still sitting up.  I
stared at her, wondering if what I felt for her was real.  She was very
beautiful, in a way.  

\section{}

AM and I emailed back and forth a few times.  She sent me a picture of herself,
and I took spent a half hour, late at night, trying to take a good picture of
myself at my desk.  I eventually sent her a picture of myself making a pouty
face, with a pen behind my ear.  I was initially surprised by AM's picture.  She
had short hair.  I told her that she looked kind of like Virginia Woolf.  


\section{}

I was not going to class, because I needed to finish the essay.  I went the the
Appel dining hall with a book every morning, after most people had already left.
It was bright and clean.  I felt a strange sort of morning sickness that was
helped by the book and listening to catchy, pretty songs over and over.  

On Thursday morning I was interrupted from my uneasy morning routine when a man
in a ROTC uniform came up to my small table.  ``Is anyone sitting here?'' he
asked me.  The skin around his mouth was scaly, tight and slightly red.  He wore
glasses with small lenses, and made eye contact with me.  He pointed at the seat
across from me.  

``No,'' I said.  He sat down, making the noise that people make when they've
``been on their feet all day''.  After a moment, he stood up and said ``I'll be
right back''.  

I thought that he was brave, forcing himself to socialize with people.  I had
stopped doing that.  I probably should make myself do stuff like this, I
thought.  But it's so \textit{annoying}. 

When he came back, I watched him eat.  His eyes looked like little blobs of
shit.

\section{}

I had an Oral exam in Chinese.  I had been up until 4am reading about Aristotle
and my face felt stiff.  I smiled and apologized when the teacher asked why I
was missing class.  We began going through the pre-defined dialogue, but I
found myself having more and more trouble.  I felt far away, dissociated.  We
were talking about paper cranes, and I was forgetting words, feeling more and
more transfixed by the desire to excuse myself and leave.  The teacher gave my
a pitying smile.  

George and I were getting more interested in what I called ``laptop music''.  I
wanted to use my laptop as my primary instrument.  I wasn't having much luck
with memorizing songs on my classical guitar.  I wasn't going to be able to make
an impact online doing that.  

What finally convinced me to download Ableton was finding the song ``Is This
Really What You Want?'', made by Tao Lin and Carles.  In an effort to focus
better, I'd been going to the computer lab at Robert Purcell Community center
after finishing dinner, instead of back to my dorm.  But instead of working on
math I'd generally just use the computers there to look up information about Tao
Lin and other internet things.  I was thrilled when I found ``Jesus Christ (The
Indie Band)'', which was just Tao Lin and Carles talking over a synth-pop
background.  They only had one song.  I showed it to George and we both listened
to it many times, and eventually agreed that it was a great song, and showed
pretty conclusively that Tao Lin and Carles were distinct human beings.  I
pirated the Ableton music production software and started learning how to use
it.  

I had started listening to Crystal Castles a lot more when one of the people in
the Ulysses class emailed everyone a link to one of their songs, ``Air War'',
which included a remixed a recording of an Irish woman reading from the
``Sirens'' chapter of Ulysses.  Before Crystal Castles had seemed too poppy for
me, I had heard it on one of the Canadian alternative radio stations I could
sometimes get in Buffalo during the summer.  But I was realizing more and more
that I loved pop music, even music like Nicki Manaj that Morgan had shown me.
Very late at night I would sit at my desk with my headphones and listen to
``Bottoms Up'' over and over again, probably loud enough so that my sleeping
roommates might of heard the sound leaking out of my default apple earbuds.


\section{}

\textit{Professors,}

\textit{My name is not Patrick, and the email below is not directed to me.  It seems
that you are trying to contact a Patrick Steadman.  I do not know who that is,
but I have occasionally received emails intended for him.  My name is Peter
Steadman, and my email address is psteadman@gmail.com.  I'm guessing that
Patrick has sometimes made a mistake when giving out his email address.  I
imagine it must be something quite close to mine.  Please note that he has not
received this message, and if you could let him know that I have received
another email intended for him, I would appreciate it.  I have received
several others, including one or two from his mother, so I don't imagine that
he has given an erroneous address on purpose.
}

\textit{Well, this explains some of the non-responsiveness, I suppose.  I just
forwarded my message to Patrick's Cornell e-mail address:  pts44@cornell.edu.}


The assistant professor of the Ulysses class had forwarded me this email, asking
me to ``please let us know how you are and where you are with this''.  They were
wondering where my essay was, I hadn't been receiving their emails because they
were sending them to the wrong email address.  I hadn't been to class in a few
days.

A few minutes later I received an email from someone named Pamela.

\textit{
He reports that you still have not turned in your paper, despite a meeting last
Wednesday where you told him you would complete it and turn it in.  It seems as
though you are struggling with something that you are having trouble
overcoming, and I am urging you to set an appointment with Lisa Ryan or
Catherine Thompson in my office.  We can help you. }

\textit{
As our concern for you rises, our method of trying to reach you will escalate 
we need you to respond so we do not have to take such measures.}



I took these emails as a wake-up call.  The feeling of the cold part of fall
coming made me feel hopeful.  I'd study for the Calculus prelim on Thursday.  I
hadn't been very good about going to class because it was early in the morning,
but it'd be good to focus on math.  There was also an essay due in the
Communications class.  I'd work on that after the Calculus prelim, and turn it
in a bit late again.  It'd be okay.

\section{}

I stared at the Calculus prelim.  I didn't know how to do anything.  I hadn't
studied, really. I had forced myself to go to the exam anyways, which was hard
enough.  But still, I didn't know how to do it.  Tears came to my eyes.  The
test room was a large, warm apitheater with red carpeting. I looked through the
test booklet for about a half hour, attempting some of the problems on scrap
paper.    Then I turned in the empty test.  It was a beautiful night, and I
walked home with other testtakers, but feeling removed from them.  I felt that I
might really be in trouble now, grade wise.  I thought about how time had passed
over the past week, and how I was now deep in the semester.  I dreaded the
thought of my parents.  I couldn't let the semester keep going like this, I had
to turn it around.

\section{}

On Halloween, Sunday night, George and I went out together.  We were looking
forward to going the art dorm party.  Some of my floormates were also planning
on going.  A group of girls were standing in the entrance to my suite, talking
to my Asian roommate.  The girls were wearing skintight black dresses and cat
ears.  My roommate was drunk, sitting at his desktop computer.  I borrowed the
robe from his Jedi costume, since he said he was too drunk to go out. 

I felt excited.

George and I walked towards Risley.  A group of guys yelled ``faggot'' at me,
probably because of my robe and mannerisms.  I turned around and screamed
wordlessly at them, unable to see their faces.  George laughed.  I felt
anxiously happy and hopeful.

There was a long line for the party.  George and I waited in it silently.  Most
other people were wearing pretty elaborate costumes.

When we got inside, it felt more like an ``activity night'' than a party.  There
was a volunteer massage therapy station and an empty dancefloor.  It was not
worth the effort to get drinks.  It was kind of fun to wander around the
hallways of the dorm, looking at the drawings and paintings on the walls.  The
art kids seemed to all know each other and rushed in groups through the halls,
laughing and talking in a way that seemed to cause George pain.  We eventually
found a quiet place in the moonlit courtyard, and talked about ???. 

I forced myself to talk to a few people, but the conversations died quickly.
George and I eventually left, and managed to get into another party.

\section{}
On Monday, I woke up feeling very anxious.  I had an email from my
Chinese, in English for the first time, that simply said ``Patrick, you've been
absent for a long time, please set up a time when we can talk.''

I hadn't responded to the emails from the professors or the advisor.  I had
planned to take care of the Calculus prelim and make a good start on the essay
before I talked to them.  That had not happened.

I buried my head in the pillow.  It was 4pm in the afternoon, and bright
outside, almost summery.  I'm doing the Raskolnikov method of problem sovling, I
thought: sleeping and hoping the problem is solved in my sleep.

\section{}

I don't know if I can return to my family like this, I thought.  At the same
time, I didn't feel like I belonged in College.  I belonged in my parents
basement.  The one possibility that I hadn't expected was coming true.  I was a
totally ordinary person with nothing special to offer the world.  I was torn
between routine and the need to get shit (specifically the essay) done.    

Sometime the week before, when I was at the library after midnight, I started
drawing my hand on a sheet of printer paper.  I was trying to make it a routine.
The drawings seemed to be improving; they were good, at least.  I drew on the
back of handouts and printer paper.  I imagined compiling them into a book, and
sending them to my high school crush Ally, who had messaged me saying she was
having a rough time, as a way to prove that a person could improve.  It took
over an hour to make one drawing.  I tried to figure out how I could fit that
hour into the day by making a list of the things I needed to do each day. 

I was in a big mess.  But I was starting to feel a euphoric sense of the
largeness of the world.  One of the benefits of my failure to be a success was
that there were so many other places I could go.  I talked to my old friends
from high school and various programs I had taken part in on facebook.  Facebook
had just released a chat feature.  It would change things, I thought.

\section{}

On Tuesday morning, I got an email from the TA of my Calculus class, saying we needed to
talk about my prelim.  I also had an email from Mommy, saying she had to drive
to Syracuse for a meeting, and could have dinner with me on the way home.  I had
sorted of hinted at my problems on the phone.  

``Do you think people can change?'' I asked George, making him look up from his
book.  We were in the Graduate English Lounge.  The email from my mother had
made me feel better, and I got dressed and headed to the library, and decided to
meet with my Chinese professor in the afternoon.

George thought about it for a bit.

``I don't know.  Probably not.''

``I'm really afraid of that.  I hope people can change.''

``Well, I think situations can change them.''

``But they can't like consciously change themselves, you mean?''

``Yeah.''

I kept thinking about this for the rest of the day, especially when I was
walking to meet Teng Laoshi.  It was a muggy, foggy day, and the afternoon still
felt like morning.  Teng Laoshi, like everyone else, seemed sympathetic.  She
agreed that it was still possible for me to pass the class, just barely, but
that I should talk to my advisor and try to drop it.  I left her and went to the
library.

A few days ago I had started reading ``100 Years of Solitude''.  It seemed to
reference many of the things I had researched while reading Ulysses.  I felt
like I was becoming aware of the nature of literature, something made up of many
specific things that make up humanity, somehow beyond words but hinted at by
words, only carried along by the feeling created by good sentences, or the
feeling one got at night, walking from place to place.  I decided to finish
reading the book that night.

\section{}

The next morning, when I returned to my dorm after staying overnight at the
library, the cringey male RA was hovering outside of my door.  A week or so ago,
before the prelim, he had told me to email back the advisor asking about me, and
had also mentioned that since I was an artist, maybe I could make a mural for
the common room?

``Hey Patrick.  Uh, just so you know, some people were looking for you.  You
should definitely respond to them, as soon as you can.  Alright?''

He gave me a business card with Pamela's name and phone number on it, and I went
into my room.

I opened up my laptop and checked my email.  I immediately knew I was dealing
with a crisis: 

\textit{If you do not respond to this email within 24 hours, we will ask
the Cornell Police to do a ``well visit'' to your class and/or dorm room.}

I quickly responded with one line:

``I'm fine.  'Well visits' sound embarrassing/traumatic.'' 

I also responded to a few other emails.  I looked forlornly at an email from the
Communications professor, asking if I was going to turn in a paper, and then
a few days later informing me that the paper would be an F even if I turned it
in, at this point.  I responded to the math TA, telling him it had been a bad
few weeks, but that I thought I could pass the class.

Sweating heavily, I called Pamela's office and set up an appointment for the
next day. 

I spent a lot of the evening talking to people on gChat.

\section{}

I woke up early and made myself look normal.  I felt very strange and detached.
I knew the meeting would go well.  One of the few things I was born with, I
believed, was the ability to make adults trust me and think I was the ``right''
sort of person.  I always remembered adults, especially older adults, praising
me.

I smiled and shook the hand of the advisor, and followed her into her office.
She talked about how her and some of my professors were worried about me, but
felt that I had a lot of great potential.  I explained that a lot of my troubles
were stemming from trouble writing my essays, and not really liking what I was
studying in the College of Agriculture and Life Sciences.  I lied a little bit,
telling her that I'd been going to my other classes, and that I would stop
working on the English class.  I also showed her the emails that my professors
had sent to the wrong email address, explaining that it was part of my
unresponsiveness, but that I was sorry for the problems it had caused.

By the end of the meeting, it seemed like the advisor was concerned, but
confident in my ability to turn things around.  I left the office feeling
better, and felt that I had a chance to fix things.  I just needed to pass my
classes, I told myself.  This semester wasn't going well, and wouldn't look good
on my transcript, but I'd face the consequences, and past most of my classes,
and get on track.

\section{}

My hair was too long, and I didn't want it to look like a ``jew fro''.  Instead of
going to get my hair cut, I ordered hair clippers from Amazon.  I cut my own
hair in the dorm bathroom, the hot metal of the clippers burning the nape of my
neck.  

A little after midnight, I was walking home from the library.  I saw a guy who
seemed like he was crying.  He looked pathetic, I was disgusted.  When I got
back to my room I realized that I was disgusted because I was that way too, that
ugly and alone, but I didn't allow it to show. 

I knew I needed to get my shit together.  I needed to manage my life.  Things
were a big mess, I recognized.  It was too unstable.  This was a big change in
my thinking.  I needed a clearly-stated philosopy of life, that was
well-articulated, but devoid of sophistry.

I needed to fully believe certain truths.  I wrote that I needed to believe that
there was nothing inherently special about being me, Patrick Steadman.  Nothing
was ``meant to be'' in my life, in any objective way.  It would be dangerous,
false, and inefficient to believe otherwise.  It'd make me hate other people.

I needed to believe that I was probably not a ``genius'', whatever a ``genius''
was.  I needed to believe and accept that other people were better than me, at
things.  Even if I was a genius I wouldn't be a genius at every moment.  My true
self couldn't be expressed at every moment.  

I needed to recognize that what I was doing, wasn't working.  But I also had to
realize that I didn't have to ``remake'' myself in some epiphanic way.  Some
things \textit{were} working.  And remaking myself was impossible, I had to
remember that, even though it was a seductive idea. 

The one thing I \textit{hadn't} tried was diligent, patient, disillusioned
effort.

I thought of the pitying, longing, insecure feeling I had when I saw people
studying the sciences or math.  I felt hopeful about this new approach, and went
to sleep. 

\section{}

On Thursday night, my mother pulled up to Robert Purcell Community center in the
Volkswagen Jetta.  It was one of the first cold, wet evenings, and north campus
seemed quiet and subdued.  I got into the car.  She commented about my hair,
which was sort of lopsided because I cut it myself, but she seemed cheerful.
She seemed like she had absorbed positive energy from her day of meetings in
Syracuse.  

We drove down the the Commons, and walked around in the rain a bit before we
chose the Mexican place.  The Commons smelled of spiced apple cider, cooked on
the stove.  My parents had taken me to visit the Ithaca Science Center when I
was a kid, which was the center of a scale model of the solar system that
extended out through the long plaza of the Commons.  Even as a kid, Ithaca
seemed to have the maximum amount of ``culture'' that a small town could have,
mixed with a sort of run-down upstate counterculturalism.

Soon we were talking about school stuff.  I had talked to her about my essay
problems over the phone a few weeks earlier. 

``I talked to an advisor and stuff,'' I said.   '' I'm handling this.''

I also told her I hadn't managed to finish my Telluride house essays, and that I
was sorry.  

``Next year,'' she said, and talked about some of the times she had had problems
in college.  It seemed that she didn't really ``get'' what I was dealing with,
but I wasn't upset.  It was my problem.

When she dropped me off back at Jameson hall, it was as dark as the middle of
the night and raining.  She noticed I was limping as I walked out of the car,
the water from puddles splashing over my sandals.

``Is your ingrown toenail still a problem?'' she asked, smiling.

``Yeah,'' I said, embarassed.  It had been getting worse again.  

``You should go to Gannett and get it looked at.  Maybe you could talk to
someone there about your essay writing problems too.''

``Okay, it'll be fine though.  Don't worry.''

I just need to pass, I reiterated to myself.

\section{}

I was spending a lot of time reading and writing in the dining hall.  I think I
was inspired by Thomas Pynchon, who was known to get a big plate of spaghetti
and a coke and spend hours studying.  If I stayed at the dining hall most of the
evening normally someone I knew would see me and sit down with me.  Spending the
whole evening in the dining hall would drain me though, and I would leave as the
work-study students began cleaning.  One evening I heard people singing in one
of the rooms of RPCC.  The SA had organized a karaoke night; they were playing
videos on youtube for people to sing over.  I thought of my host family in
Taiwan, and the simplicity I felt living there, singing Taiwanese songs on the
karaoke machine in their living room.  

I said that I wanted to sing, and found the Teresa Teng song I wanted.  There
were about five people lounging on the couches in the room, most of them
overweight.  I sang the song, enjoying the sound of my voice, and the ease that
the Chinese lyrics came to me and the feeling I felt for each character.  After
the song no one made eye contact or said anything to me, and I walked away.  For
the first time at Cornell I felt sad and defeated.  I wondered what this new
feeling meant.  My thoughts went to my mother, my father, and Morgan.  Maybe I
was realizing that I was nothing special, and that was the lesson of college.  I
was fated to be with Morgan, who reminded me of my mother, and was perfect.
Soon enough I would kiss Morgan and she would be my first girlfriend and calm my
heart, and I would be able to write my essays like a normal student.  It was my
fate.  I went back to my room and fell asleep without taking my contacts out,
tears at the edge of my eyes.

\section{}

Multiple times in the same week I lost the four-colored bic pen that I needed in
order to take notes.  I went to the the Cornell store to buy a new one, and
generally spent at least an hour there.  The first book I bought there was ``The
Literature Student's Survival Guide'', which contained summaries of the Bible
and other canonical texts.  While reading Ulysses late at night in the library,
only managing to finish a page every fifteen minutes or so until I fell asleep,
I suddenly awoke and realized I wasn't really ready for the book yet.  I needed
to learn more, and discipline my character before I could write about Ulysses.  

I learned where many of the words I was using in my essay came from: the word
Shibboleth, used by Gileadites (balm of Gilead people) to identify and kill
Ephraimites in their midst, who couldn't pronounce with word \textit{shibboleth}.
It was a word I recognized from the URL querystrings that I saw when signing out
of the various academic portals.  A.M. had mentioned that I should take
advantage of my access to JSTOR: I had seen SHIBBOLETHs being passed around when
connecting to JSTOR.  I looked up a bunch of articles about Bruno and Joyce on
JSTOR, and was amazed by the new wealth of knowledge.

I made myself stop reading all the other books I had borrowed from the library.
I had checked out a large stack of old books about Anglo-Saxon mysticism, books
that I would have found fascinating as a child, when I lay awake in bed at night
hoping desperately that magic was a real thing, hoping so hard that tears came
into my eyes.  I remembered that this type of behavior had continued into middle
school.  Now I felt like magic was still possible, that certain experiences I
had crafted for others must have seemed magical: when I had created a hidden
shrine in the basement of our high school and led some friends and acquaintances
down there to ``discover it''; when I had spent weeks planning for a party my
freshman year that ended up being a good party where things happened; and
hopefully I would be able to create an experience like that for George and I,
finding the right party.  But maybe the lesson I was learning was that
``magical'' experiences could never be magical for the person creating them.
George regarded the big stack of books with skepticism as I brought it back to
my dorm.  In early November I had received an email notifying me that I owed a
\$80 fine to the library, because the books were overdue.

I didn't have much to show for all my agonizing over the Ulysses essay.  What
did I have to show for the last week, I asked myself.  It was a lacuna, lacuna
was a word I had learned from Ulysses. I had screwed up my semester over the
essay.  I decided to make a final push on the Ulysses.  It would be the easiest
way to solve my problems: if I could just show my advisors and parents a
finished essay, or even semi-finished essay (I had to be realistic), they would
understand where my efforts had been focused.  Transferring to a different
program within Cornell was probably necessary, anyways: hopefully one of the
state schools, since they were much cheaper.

I went to the library with only my copy of Ulysses and my notebook, and started
working my way through the last few hundred pages of my book.  From research on
JSTOR I knew that my thesis was heavily dependent on the last few chapters of
the book, where everything miraculously came together.

My notebook had been becoming more and more organized as the semester passed.
Now I wrote down every word I didn't know in blue, and added a definition, so
that I could easily find and review them.  Joyce referenced so many beautiful
things, the things I wanted my vocabularly and life to be filled with: rich
looking green and yellow drinks, feminine calm, and very specific types of
plants like jonquils.  I went to the computer and looked up a picture of a
jonquil, and then drew it in my notebook.  I wanted all the things the ancients
had.  I googled ``Frankencense and Myrrh'', to see if I could buy it.


\section{}

I finished reading Ulysses on a Saturday morning in the dining hall.   I spent
almost all of the nights during that week in the library, until after the 2AM
when the security people came around to check student IDs.  I had made slow
progress on the book, but the final chapter seemed like the perfect
consummation, the imaginations of two perfectly real characters treading the
same memories, affirming love.  It fit perfectly with my essay, and my thoughts:
I wanted to live a consummated life, not just a life of books.  I would write
the essay and fix the semester.

To reward myself, I met up with Paola (she had texted me).  I went to the Cornell
store with her, and bought one of the books I had sampled online: ``Foucault's
Pendulum''.  It had strange, silky pages.  Paola and I went to sit outside on
the quad, something I had never really done.  It was windy, and almost too cold.


\section{}

Morgan, Seth, David and I always talked about going out, but it never really
happened.  Sometimes Morgan and David would talk jokingly about things that
happened on the weekend, but those nights out seemed to be with their
floormates.  We didn't actually all go out together until a chilly night in the
middle of November.  

I had high hopes for the night.  I figured something would happen between Morgan
and I.  I had no idea how to negotiate the physical space between her and I, and
saying something to her seemed impossible in her presence.  I figured alcohol
would make things happen.

We ended up going to a few fraternities around the new West Campus dorms.  I
watched Morgan play beer pong with Seth.  I had a few shots.  Morgan, Seth,
David and I danced.  Seth was being goofy, and Morgan laughed.  

I was starting to feel more tired than drunk.  A few times, Morgan and I talked
about our schoolwork, but when Morgan was drunk, talking to her about Ulysses
and her child development stuff felt hollow.  The converstations didn't last
long.

I was falling behind the group, which had swelled to include some other loud
guys, as we headed to an afterparty in Collegetown.  I had no connections at
these places so I was just along for the ride.  I realized suddenly that it was
November, and I didn't have the social connections that other people were
developing, despite my best efforts.  I stalked behind Morgan, listening to her
talk to a group of loud drunk guys.  I felt more and more drunk and unhappy, and
felt angry and disappointed with Morgan and also myself.

I dropped away behind a wall, and then ran away, and stopped again, feeling
ashamed.   I was inside the arches of the World War II memorial.  I waited,
breathing hard.  Suddenly Morgan appeared.  

``What are you doing, Patrick?'' she said, seeming annoyed, but a little amused.
I stared at her in mute shame, knowing that I didn't have words for what I was
feeling.  ``I think you should go home,'' she said.  ``Can you get home?'' she
said, in an overly concerned voice.

``Yeah,'' I said.  I set off on the direction of our dorms, but then turned
towards the suspension bridge.  I felt a sort of thrill at my unhappiness and
disappointment, watching my feet move quickly in the orange streetlight.  It
felt like something was broken and I was being pushed on by a force.  I walked
over the bridge, thinking about climbing over the fence.  I told myself I was
fine and that I was free of Morgan now, and that tomorrow I'd write.

\section{}

On Monday morning I went to the library early.  On the way there, a girl I had
noticed before rode by me on a bike, her hair shining in the morning light.  I
felt moved.  I had seen her reading a book in the dining hall, once.  I had
tried to see the title.  It seemed to be about gender issues.

At the library I wrote a long poem-like thing about her in the margins of my
notebook.  Don't flinch away from a beauty you can't avoid, I told myself.  Life
was beautiful, just like most books said.  I would be saved by desire for sex and
happiness.  I didn't want to have a miserable shitty life.  I wanted beauty and
happiness.  I would work hard to get them.  I didn't want to have a miserable
life.  I was reading ``The Garden of Eden'' by Hemingway, because AM had recommend it to
me.  It was really beautiful.  I wrote down sentences from it in green pen,
which described the the ``fun'' that Hemingway and his young wife had had, by
the blue ocean in their rough sweaters.  More than anything I wanted beauty like
that.  Tears came to my eyes.  I was sick of this.

I forced myself to stare at the blinking cursor until I thought of a sentence to
write.  I wouldn't let myself look away from the cursor for a single second.
This seemed to work.  I wrote and rewrote the first paragraph of the essay,
uniterruptedly, until it made no sense when I woke up the next morning.  In
dismay I went back to sleep, and woke up at 4pm, disoriented.  

\section{}

I wanted to discipline myself.  For my parents, for her.  In my notebook I wrote
that I would spend seven hours sleeping, one hour on the internet, one hour
drawing, one hour playing my guitar, one hour for eating and hygeine, and the
rest studying.  I would get myself out of the hole.  But some things were simply
too hard or painful for me to apply discipline to.  I could only think about
writing the Ulysses essay for about five minutes at a time before I would find a
way to stop and do something else.

\section{}

I lay facedown on the upholstered cushions on a bench in Uris library.  I had
just awoken from a short nap, after making no progress on the Ulysses essay.  It
was late, and my core felt cold.  I kept my eyes closed, thinking of strange
things from the past: the gerbil warrior king I had imagined as a childhood.  I
tried to rememeber the name I had given him: GerbilAwfulous.  My own life felt
strange to me.  I remembered how, at last Christmas, my parents had told me that
I was terrified by The Nutcracker as a child.  I thought about AM and Jennifer,
the skype girl with cancer, all of the people I talked to online.  Eventually it
became too painful to continue lying there, and I sat up and went on my
computer.  Jennifer was online and we skyped for a few hours.  She was being
strangely combative, talking about her experience at Yale.  She was talking
about a Danish guy she had met, and how attractive and ... brooding he was.  He
was a Junior, and wrote for the literary magazine.  I asked her if I wasn't
attractive in the same way.  She laughed.  Jennifer was still angry at me for
unempathetic things I had done while talking to her on the Internet in Taiwan.
``You can't be attractive, \textit{in that way}, if you know what I mean?''
``Why?'' I asked repeatedly, wanting to know. ``It's hard to explain in words,''
she said.  ``You're too...\textit{hairy} to be that way, you know?''

I had done everything but my duty that day.  At worst I would end up like a very
overcultured Chinese gentleman, if I continued like this.  I thought of my host
father in Taipei, with his beautiful house built into the hill, full of shelves
of books and DVDs.  

\section{}

``I feel like I keep putting off the beginning of a 'long climb'.  Or I start
and just quit, which is much worse,'' I wrote in an email to AM.  I said that I
loved ``The Garden of Eden''.  It was November 7th. She responded that she was
laying in bed after driving for four hours, and was tired but awake.  I never
responded to her email, and we never talked again.

At the library, I started reading a pdf of ``The Secret History'', about a
student who is seduced by a group a wealthy classics students at a small liberal
arts college in Vermont.  I wanted to write seductively.  I stayed up the whole
night reading.  The only time I stopped was to get Snickers and Mountain Dew and
go stand in the glass coupula that overlooked Libe slope, and the city of
Ithaca.  On the gently sloping blackness of the hill across the lake I saw the
bright line made by a long country road, where individual cars were visible in
the morning.  It was like looking down from an airplane, a scene that contained
too much information, and I tried to make sense of the shifting geometry and
light but only felt a deep thrill of emotion.  I felt I was destined for
something.  

\section{}

A few days later, I hadn't really made progress on the essay.  What did I have
to show for the week? I woke up in the middle of the night with a intense desire
to shave my head, or half of it.  I lay awake in bed, sweating.  Somehow, I went
back to sleep.

\section{}

I would fall into a shallow sleep in dining halls or the library.  In
half-dreams I remembered the cold one felt when one swam just a meter or two
down into a lake or the ocean.  I remembered how it felt to grab another boy in
the water, or for someone to jump into the water on top of you.  In middle
school I had joined an Existentialism discussion board on MySpace.  I remembered
reading MySpace stories written by one member of the board that described
intentionally drowning someone weak at summer camp when he was a boy.  The story
described how the struggling vicim had suddenly released his bowels.  I wondered
if I had any deep dark secrets like that, secrets that I could never tell
anyone, even my wife someday.


\section{}

I started reading the copy of ``Foucault's Pendulum'' again.  The book had even
more references than Ulysses, almost: the references were almost made in a very
tounge-in-cheek way, which supported my growing feeling that literature was a
sort of massive, half-serious network of very old buzzwords and concepts.

Sometimes I realized my real life situation was bad.  Using the library
computers, I talked with old friends from middle school on facebook, who told me
that they believed in me.  

``Foucault's Pendulum'' led me to read a lot of Wikipedia articles while
listening to classical music.  I read about ``Pecca Fortiter'', the idea that
men should sin \textit{greatly} and without fear because God's grace was
infinite.  I read about the Church of the Holy Sepulchre, where a ladder had
stood against a wall since the 1800's because moving it would upset the status
quo between the Jews, Christians and Muslims.  I read about Euler's Identity:

\begin{equation}
  e^{i\pi} + 1 = 0
\end{equation}
      
Gauss is reported to have commented that if this formula was not immediately
obvious, the reader would never be a first-class mathematician, I read.  I spent
hours trying to understand it.  I would have to work hard.

Giordano Bruno had been intrigued by the Ars Memoria, a system of mnemonics
based on the Sepherot that allowed practicioners to systematize and remember
their knowledge.  I found old texts on the subject and began writing out a table
of the letters of the alphabet and the core concepts in my life that they
responded to.

My research into religion led me to read the Inferno, where I read about the 
\textit{relapsi}, those who realize the errors of their way, but continue.  They
were said to be doubly hated among sinners: hated by God, and hated by
themselves.


\section{}

I was spending a lot more time with the people from my floor.  It seemed like
they enjoyed hanging out with me now, amused by the things I was interested in.
I would stay up many nights with the floormates, talking.  I showed a prefrosh
the massive collection of art pngs I had torrented.  My computer got viruses,
and was booted from the network, so I had to go to the computer lab, where I
obsessively read articles about Tao Lin, or talked to a 23 year old I had met on
craigslist.  

``You know, in the beginning of the semester, I thought you were a scrub,'' said
Jonny, one of the athletic guys on the floor.  ``But you're like the opposite of
a scrub,'' he said.  

``What's a scrub?''  I asked, intuiting the meaning of the word instantly.

Jonny seemed embarassed.  ``You know, \textit{that} guy.  Because you like
didn't talk to any of us.''  

A bunch of us were sitting around, making ramen.  One of the Asian students was
cracking eggs into the boiling water.  It was a cold morning.  I had stayed up
the whole night before, and had finally finished ``Foucault's Pendulum''.  A
significant amount of snow had fallen in the dark.  I felt a sense of stillness
and doom, knowing that Thanksgiving break was next week.  I listened to
exclusively classical music now, a lot of Schubert and Ravel.  Morgan had given
me her gChat username and we occassionally talked late at night, mostly about
the stuff we were reading, and pop music.  She always signed off before me,
around 2AM.

I felt like I was impressing my floormates.  One of the girls told me she was an
English major.  I was surprised, she didn't look like an English major at all.
She had a soft face that reminded me of a seal or otter and talked like everyone
else from Westchester.  I showed her Mumford and Sons and played some of their
songs on guitar, and she came to my room a few days later and told me that she
was obsessed with them.  She stood in the doorway for a while talking to me,
causing me a vague sense of pain and unease.

On the phone at the library I told Jennifer about the English major girl, and
made fun of her for liking ``Mumford and Sons''.  I told her that there was no
one at Cornell for me, at least that I could find.  Jennifer said that there
were lots of guys that were attractive at Yale, at least from a distance, and I
felt a sense of unease.  I sent her a picture of the English major girl.
Jennifer told me that she had already left for Thanksgiving break, and was
staying at a friends house in New England, and had pimples.  Jennifer was
annoyed at me for always being gross about things.  I told her it was the
reality of the situation.  She was hiding in the bathroom to avoid being around
her friend, who had a boyfriend.  I knew them all from the summer program.  I
remembered one night near the end of the summer program Jennifer had stayed up
all night talking.  We lay on separate couches, talking about our dreams and
hopes, and our understandings of the world.  Before we stopped talking and fell
asleep I suddenly felt very aroused, more turned on than I probably had at any
point in my life.  I was confused: at the time I did not find Jennifer all that
attractive, or didn't think she was what I needed in a girlfriend.  

On the phone Jennifer told me that she was doing so many things because she
didn't want to be unlovable, and I didn't understand.  I told her that I
absolutely identified with wanting to feel lovable.  Jennifer wanted to be
objectively good and write in pretty ways.

The next day I woke up in the afternoon and ordered a book stand of Amazon.  I
wanted to deal with my neck and upper back pain, and be able to read more
without feeling tired.  My mother had commented that I was twitching my neck, it
was a sort of tic. 

I don't think I consciously thought much about my academic situation.  I talked
to my friend, a guy who had also dated Paola, about Deep Springs.  I had told
him to apply to the school, a free two year school where you worked on a Cattle
ranch and discussed philosophy.  It was very selective, and fed directly to
Harvard, Yale, and UofC.  I started telling people on my floor that I was
planning to transfer to Deep Springs.  I had until the new year to write the
application essays.

I bought Bulfinch's mythology, and read it, waiting for fall break.  I also
started downloading TV shows that I had read about online.  I downloaded ``Mad
Men'', and found most of the pilot episode pretty dumb and shallow.  But I felt
myself being moved by the last scene, where Don Draper held his wife and both of
them looked at their sleeping daughter, in a Connecticut bedroom that looked
like an oil painting.

\section{}

My mother drove Paola, Finn and I back to Buffalo.  She was in a good mood and
took us to eat at Moosewood.  After, we walked around the Brentwood Mall, which
isn't really a mall, it's the basement of an old institutional building.  It
smelled like fall in Ithaca, a hazy smell of upstate New York and water in
basements. 

I bought a beautiful embossed version of ``The Aeneid'' at the rare and used book
store. 

In the car, I could feel the stress of the past months affecting how I talked
to Mommy. I  wanted her to stop talking to Finn about his physics major.  I
felt that he'd definitely flake out of it soon.  The way Finn talked about his
physics major, it seemed very obvious that he wanted to impress people with it. 

My plan would work out better in the long run.

As we drove through the amber fields overlooking Cayuga Lake Mommy stopped
talking to Finn and Paola and quietly asked me how things were going.

``I think I really hate Cornell,'' I said.

``Your classes still aren't going well, huh?'' Mommy said. 

``They're going badly but it's not that.  It's the people.  I'm really bored and
so...disappointed.''

I could tell what I was saying was hurting her.  But I was sick of her
upbeatness.

``Patrick, where else would you go?  I think you're being unfair.  It'll get
better.  I remember how stressful it was.  I still have bad dreams sometimes
where I have a prelim and a bunch of homework I haven't turned in.''

``I don't think you understand.  I don't know if I can go back there.''

I said the communications classes were such bullshit and that I wished I was
studying something in the Arts and Sciences College.  

``You can't just blame us for your problems,'' Mommy hissed at me, even though Finn
and Paola were probably asleep in the back.

We drove by an abandoned house that I noticed every time we made the drive from
Buffalo to Ithaca, a dark farmhouse close to the road.  I wanted to
explore it someday.

When we got home I put the Aeneid on my book stand.  The Aeneid was good and
important to read.  Of arms and men.  I could hear Mommy and Daddy arguing
downstairs as they made dinner.   The golden light of childhood autumns filled
my room.

I was called down for dinner a few times until I heard anger.  I sat down at the
table feeling shaky and tired, and drank the milk.  It felt chalky in my mouth.
Avery made a face when she saw my hair.

We started arguing around the end of dinnner, and my sisters left.  We ended up
in our ``dining room''.

``You have to figure out what you can do to recover your classes, Patrick.  Talk
to your professors,'' said Daddy.

``I can't, I can't.'' 

``What do you mean 'you can't'?  You don't really have a choice.  College is an
endurance thing, you just have to get through this semester.''

``You can't take this for granted, Patrick.  Grandma and Grandpa have been saving
for a long time so you can go to Cornell and not the University at Buffalo.  We
thought you'd be happy, once you got to a place like Cornell, with challenging
classes and people,'' said Mommy, her voice self-pitying.

``Can we please not talk about money?  It doesn't help, right now.''

I paced around the dining room table.

``You guys don't understand.  I think I'm going to fail all of my classes, or at
least all but one or two, no matter what I do,'' I said, finally.

``What?''

``Great,'' said Mommy.

``There are things you can do, Patrick,'' said Daddy after a few seconds.

``No, there's not.  I've already tried.  It's too late.  I just couldn't write my essays.  I'm
sorry.''

``Then forget you essays!  Focus on the other classes.  Have you been going to
them?''

I felt the pain coming.

``Not since, October, maybe?  I was really focused on trying to write that essay
and was at the library all night.  I'm sorry.  It's a writing problem.  I think
I'm getting closer to figuring it out.''
 
``Your essays aren't the issue here, Patrick.  You need to drop that class and
figure out how to pass the other classes.''

``I'm going to fail them all, except for maybe one.  There's no chance.  I'm
almost sure.''

I could see tears coming to the edges of Mommy's eyes.

``Oh great,'' she said.  ``You really screwed up.''

``Yeah, basically.''

My father was silent for a period of time.

``You need to talk to your advisor,'' said Daddy.  ``And figure out how to drop the
courses.''

``It's past the drop deadline.  I'm just going to fail those classes.''

``No, that's not acceptable.''

``You're going to need to get a job,'' said Mommy.

``What?  No, I'm going back to try again.  Even if I fail my classes this semester.''

``No, you're not,'' said Daddy.  ``Not right now.  We can't let you go back like this.''

``I can't \textit{not} go back, okay?''

``I know...'' said Daddy, ``that you're afraid that if you leave, you won't ever go
back.  That makes sense.''

``I don't even want to go back, I'm so miserable there.  I need to figure out
this essay thing, though.''

``It's not essays I think, Patrick.  You need to understand that you won't always be the
best in your class.''

``I know that, stop saying that,'' I said.  ``That's not my problem.''

``I think you're not turning your essays in because you're afraid of not being
the best,'' said Mommy.

``I can't even really get them started.  I don't know why.  I've been trying
\textit{so hard}, I hope you understand that.''

``You're going to have to pay for this semester you wasted,'' Mommy said.

``Please stop talking about money!  It's not helping...me write my essays.
Cornell is so freaking stressful.''

``Tomorrow we need to call your advisor and figure out how to drop those
classes.''

``First, I have never talked to my advisor, and second, it's way past the end of
the drop period.''

``Why didn't you talk to your advisor?  That's just stupid, Patrick,'' said Mommy.

``Because he's in Communications, and I don't want to be a Communications major!
You have no idea how dumb my Communications class is.  It's like the opposite of
Legally Blonde.  I'm the only person with a black Windows computer surrounded by
a bunch of sorority girls and bros with MacBooks.''

They laughed.

``Look, I know that might not be the best match for you, but that doesn't mean
you don't need to talk to your advisor or go to class.  You're being stubborn
and can't blame us,'' said Mommy.

``We can call the CALs office tomorrow and figure out what to do.''

``He should call up La Scala's and see if he can get his job back,'' said Mommy to
Daddy.

I scoffed, and made my way up to my room.  I could hear my parents arguing
downstairs.

\section{}

The next day my parents called the College of Argiculture and Life Sciences.  I
stood at the head of the stairs in my underwear, listening to them talk on the
phone.  I wasn't sure why they hadn't made me make the call, but I was vaguely
grateful.  I was so sick of it all.  I shivered in bed at night, watching Mad
Men.  My sister had moved into my childhood bedroom and so all my stuff was
crammed into a small room with large windows that let in the cold.  
