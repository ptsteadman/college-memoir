\chapter{The Bell Curve Part II}


\chapter{Un-Quitting}
The combination of the yelling, pain and sight of others failing around me would
make me expect that someone with authority would end this.


\chapter{Reticule}
During my first fall on the Cornell rowing team, I had a constant sense that
all of my teammates were calling me ``faggot'' or ``pussy'' behind my back.  I
imagined that if I showed my ``true self'' their friendliness would turn into
open hate, like when you kill allied soldiers in a video game, and they start
attacking you and your crosshairs go from green to red.  


\chapter{Hadoop}
When I arrived at Cornell for my sophomore year, I was determined to set myself
up for a prestigious internship.  In early September I put on my blazer and went
to the career fair.  I spent most of the day in the hot indoor track building
talking to employers. 

I found that I enjoyed talking to the tech companies more than the banks.  When
talking to someone from a financial services company, it always seemed like both
the recruiter and I didn't really know what the work was.  With the tech
companies, I was able to stretch my superficial knowledge of programming and
have some interesting discussions with the recruiters.  Many of them seemed to
want to believe that they were doing something really special, technology-wise.
An overweight neckbeard representing an ad tech company made a lasting impression 
by asking if I knew what Hadoop was.  

``No, what is it?'' I asked.

``Oh, man.  The next big thing.  The current big thing,'' he said,
rocking back and forth.

``Like, what is it though?''

``It's Apache's distributed MapReduce framework.  It's how we can process
millions of ad slots a minute.  You should really look it up.''

I did Google it, and felt the familiar sense of wondering what was really going
on: were terms like Hadoop, and NoSQL, or NodeJS, just buzzwords, like the names
of bands in high school, or the different sectors at an investment bank?  Or did
they \textit{actually matter}?  Over a year later, I would smile when my Database Systems class
required us to implement the PageRank algorithm using the Hadoop framework.
Computer Science had turned out to be more than emptiness, for me.

And a year after \textit{that}, one of the men indirectly responsible for the
creation of Hadoop spoke in my Information Retrieval class.  I was fascinated,
because I had watched a video where he talked about the role of 9/11 in Google's
history.  I had developed a pet theory: that the attack on the Twin Towers had
led to the creation of the Map Reduce programming paradigm, and in turn the
Hadoop framework that enabled many of the ``big data'' applications of the later
2000s.

Amit Singhal, director of search quality at Google, was away at a conference on
September 11th, 2001. As the public searched for news about the attacks on the
Twin Towers and the Pentagon, Amit and his colleagues realized that Google was
dramatically failing to meet the nation's information need. Searches for ``World
Trade Center'' led to web pages detailing the architecture of the now-destroyed
buildings, or real estate listings.  This was due to the fact that Google was
only able to index the internet about once a month: the index used to fulfill
searches did not reflect the current, dramatically different reality.  Over a
conference call, Amit and the Google engineers decided on a hacky solution: they
simply added links to relevant news articles on Google's homepage.  This didn't
work: the massive amounts of traffic directed to these articles caused the news
network's servers to crash almost instantly.

Amit and Krishna Bararat, a search architecture engineer, were trapped at the
conference center in upstate New York until planes were allowed to fly again.
Over the next few days, they sketched out the architecture of what would become
Google News, a system that would index news websites constantly, ensuring that
Google would be able to provide information about events that had just happened.
Building this system would require rethinking Google's entire data pipeline.
New programming models for distributed systems would need to be perfected in
order to have enough computing power to simulataneously index thousands of news
websites.

Over the next few years, Google News was developed, but Amit and Google realized
that everything else also needed to be indexed in ``real time''.  In order to
index the whole internet, every day, the programming methods used to create
Google News would have to be formalized.  In addition, Google realized that the
rest of the internet had to catch up with Google.  To address both of these
concerns, in 2003 Krishna released ``the MapReduce paper'', which detailed the
abstractions used by Google to think about their complicated distributed
systems.  

Something about the nature of MapReduce always felt very current and zeitgeist
to me, perhaps because of the neckbeard at career fair, or perhaps because the
paradigm reminded me of the way ``real life'' felt: many many isolated units of
data being mapped, fragmented by a hash function, and then reduced to useful
key-value pairs.


\chapter{LA}


\chapter{Winter and Spring}

I stared at the math book in dismay.  I started to read the third of nine
chapters that were going to be tested.  I had to reread sentences, I was
not understanding.  After a few minutes every few sentences my eyes would close
and I'd shake myself awake.  My eyes closed for a while.

I soon woke up angry.  I couldn't do this.  It was stupid.  I bit my tongue and
took out a notebook, and restarted reading the paragraph, writing down the
progression of equations.  I felt more awake and filled with a little
adrenaline.   

Snow blew around outside as I worked on a Business Managment homework
assignment.  The next day was the Valentine's Day Massacre, a 10k erg followed
by a 4 mile run that included a steep hill.  I was afraid.  The assignment was
tiring and dull but I felt surrounded by beauty.

I sat next to Kyle for the 10k.  I started out holding a 1:45 split.  It
was not too hard to hold it.  Seeing that I was beating Kyle was motivating.
There was nothing terribly painful at all.  I felt warm and soon enough there
were 3 kilometers to go, and I ground through them and got off the erg.  I was
going to put on pants but someone told me to start running.

Kyle was a faster runner than me.  "Football", I thought.  At the same time I
passed a few of the really large rowers, like C and A.  They were
running very slowly.  

\chapter{IRAs}


\chapter{Priceline}
Shake Carlo.  


\chapter{Wall of Real Depression}
In August my relationship with my grandparents worsened, and I rowed less.  I
worked on programming until after 2AM most nights, and my movements downstairs
woke up my grandfather.  I would hear his footsteps on the stairs and dread what
he said: "Do you know what time it is, Patrick?  What on \textit{earth} are you
doing?".  His face was tired and angry. "Work," I'd reply, ashamed.

I was, in fact, working, and I was enjoying it.  I didn't enjoy much of anything
else.  Every week Pop invited me to do various things with him.  Did I want to
go stand-up paddleboarding?  Did I want to go waterskiing with my cousins-once-
removed?  Did I want to get dinner after work?  

I had little or no interest in doing these things, and it hurt me.  I was
starting to wonder where my life had gone.  It seemed like my new thing was
a single desire to become ``a real software engineer'', so that I didn't have to
deal with so much bullshit.  

The sense of hope I had felt about Connecticut and adult life was starting to
unravel.  I was least happy during my morning commute.  My air conditioner was
failing and would occassionally emit hot humid air.  Every morning I drove past
a group of Mexican immigrants waiting for contractor pickup trucks.  My
grandfather also picked up Mexicans.  When he drove up to the bridge that they
waited under, they would try to pile into the back of his truck, because they
thought he was a contractor.  I got angry each time my grandfather repeated this
story, or talked about how he repsected the Mexicans because they at least
worked, unlike the other residents of South Norwalk.  One time when we were
driving down the to yacht club to go stand-up paddleboarding, I said that
it probably made more economic sense for American citizens to collect welfare
than to do hazardous, low-paying, uninsured work, so maybe he should respect the
residents of South Norwalk for being smart, like bankers. 

On the weekend of July 4th, I helped my grandfather sail his boat to Newport,
Rhode Island.  Every year since I was five or six, I'd done the "overnight sail"
from Connecticut to Rhode Island.  

Hot tears, disbelief

Relief and doom. 
Hackathon?

Bradley Manning was sentenced to 35 years in prison.  I read this news walking
through the rows and rows of brightly-colored desks at Priceline.  I left work
early and drove back to my grandparents, and read the chat messages that
eventually got him caught.   I felt that ``real life'' might be talking to
someone you barely know late at night at the Internet when you're far away from
home and have witnessed things that are too big for you to swallow, or dooming
yourself because you want to talk.  When I met up with some of my childhood
hacker friends back in Buffalo around Thanksgiving, we were always careful to
correctly gender Manning.  

I was most happy at the office, programming and listening to EDM.  I didn't want
the summer to end.


Awkwardness that did not go away.

\section{}

President Obama was visiting upstate New York in the driving rain and the I-90
was blocked completely.  I texted William and asked if I could sleep at the
Knoll, and he said sure.  I was excited to get back to Cornell.


There was a party at the Knoll the night I arrived, or maybe the night after.  I
drove Jim and V into downtown Ithaca to get kegs.  Seeing my teammates felt
awkward but good.  I'm not sure why I felt awkward.  It was difficult to go back
to the level of intimacy we had felt as a boat, or something.  Also my head was
still in New York, thinking about programming, to some degree.    

The party started.  I felt expectant.  I gradually got drunk and put my energy
into bro-ing out with my teammates.  Soon the house was totally packed.  But
nothing \textit{happened}, beyond being greeted happily by my teammates.  I
didn't talk to any girls.  As the night progressed I felt more and more
irrelevant and I told myself that the simple and good solution was to row
harder, get better.  Then things would happen.

***

I was having trouble pushing myself.  During 2013, even in the most miserable
workout the perfect EDM song might come on, and I would feel cold adrenaline go
down my neck and feel unexpectedly good and want to push more.  Something about
the reality of the situation made me embrace it, even if meant going to the
bathroom between pieces and seeing sweat roll off my legs my scared red face in
the mirror.  

***
First time I got high
Driving home
Rowing robotically the next day

\section{}
blue lips (cyanosis) after fastest 5k

\chapter{Bushwick}
getting the staticy blanket and feeling calm and good
programming on 11 inch macbook air
winter
smoking synthetic weed with Iraq vet
Justine
LSD on new years
new hope
dislocating knee

Feeling comfortable with my sexuality was sort of like discovering a hidden room
in the basement of my childhood home.  The beautiful confident place I had found
with Lisa and Anna was a place I knew I could go back to.  My past worries were
gone, I didn't have to ``come to terms with my sexuality'', I didn't have to be
afraid or ashamed. 

\chapter{Anna}
and I dropped away behind a wall, and then ran away, and stopped again, feeling ashamed.
balance beam

\chapter{Moat}


\chapter{Facebook}


\chapter{Graphics}


\chapter{Rowing Machine}
The rowing machine was the only sacred object in my apartment.  It was situated
so that I could see the Manhattan skyline while I rowed, and in the late fall
my heart was moved by the slow pulsing of the warning lights across the river.

My parents were confused by the young adult I became in college, and when I
walked on to the rowing team, they had clung to my identity as a rower as one of
the few positive facts they knew about me.  As a result, almost all of my
Christmas gifts during those years were rowing related.  These gifts hung on the
wall next to the rowing machine: a hand drawn poster from the 1920's
commemorating Ivy League crews, parts of antique rowing shells, and,
``embarassingly'', pictures of me racing at Eastern Sprints and the IRAs.  When
people made fun of my shrine, I reminded myself that these items were proof that
my parents loved me more than any of them.


\chapter{Abuse of Trust}
The methods used to accomplish most cracking-style hacks are almost embarassing
to me. 

Abuse of trust is one of the most effective ways to gain access to valuable
data.  Part of Facebook's culture is that Facebook ostensibly places a high
level of trust in its employees.  I think most people are unwilling to steal
data from their employers due to organizational loyalty, intense punishments
when caught, and the difficulty of finding a buyer for that information.  On a
personal level, something else seems to stop people from violating each other's
personal data.

What does it mean, that most people, if left alone in a room with one of their
friend's diaries, would respect that friend's privacy and leave it alone?  To
me, this is strange.  The information in that diary could potentially really
benefit them in the long run, perhaps deepen the relationship with that friend,
or make you realize that you should stop being friends with that person.  The
friend is unlikely to be hurt by your having read the diary.  Does this mean
that people would rather ~not~ know the truth about the relationships in their
lives, or believe that not knowing the truth is ideal?  Or, does this mean that
many people are fundamentally uninterested in the mental lives of their friends?
These explanations all seem plausible to me.  Or, there may just be very
effective taboos in place that prevent any well-socialized individuals from
reading private information.

\chapter{Old Messages}
While reading through the old messages of friends and lovers, I sometimes ended
up reading conversations from my own past.  I usually felt the familiar dread
and pain that came  with seeing myself as others saw me.  None of the messages
were old enough to allow for the possibility that I'd changed since sending
them.  A person with more willpower might have used these messages as motivation
to change himself.  A more humble person might have realized that she should be
kinder to others.  I felt pain but realized that I had to ignore it.

\chapter{Live the Dream of 2012 in 2014}
facebook interview

\chapter{Buffalo 2015}

I spent my last winter break in Buffalo, feeling mildly euphoric most of the
time.  I went to work with Daddy.  He had decided to buy a fully electric
Volkswagen Golf, and I went with him to pick it up.  We worked strange hours,
driving to the factory around noon and leaving at eight in the evening, when it
was freezing cold and dark.  During the drive, we discussed the project, and
then often continued talking in the kitchen when we got home.  Sometimes it
seemed like we were both a little embarassed by the amount of passion we
had for the project. 

My mother was angry that Daddy spent so much time at work.  One time she
came down and interrupted us.  ``Your father's workplace is so dysfunctional,''
she said.  ``It's given him a stomach ulcer.''

I think my father and I were motivated by the the vision of a Buffalo factory
that was not dysfunctional. 

My family went out for dinner a lot, to new restaurants in revitalized areas of
Buffalo.  They were nice restaurants.  Walking in, I was often silently amazed
that the brassy, dark, spacious glow of New York restaurants could be replicated
in Buffalo.  These restaurants had a selection of good beers and I would get two
and be quite buzzed.

One time, my family was waiting for a table at the bar.  My mother was talking
about her drink, part St. Germaine and part white wine.  I hoped she was
overhearing the handsome older men talking about real estate investments in
Colorado.  One man owned a warehouse that he was now leasing to medicinal
marajuana growers.  I imagined my friends from high school meeting for drinks in
this restaurant, by the edge of the old Erie canal, discussing the wealth we had
built over our lives, wealth that extended beyond Buffalo but was still somehow
still rooted in Buffalo.  I thought of sixty years from now.  The future
seemed...nice.

Many nights I had nightmares about rowing, where I missed practice and shouted
at Coach Kennett.  I had another strange dream about Ocean Palace.  It was
disturbing.  Somehow we became close, and she was at my place after going out.
I was touching her, we were about to have sex.  Then she stopped me and said
``stop, I'll never have sex with you, you disgust me, so many aspects of your
body and personality gross me out.''  Then I experienced a telepathic slideshow
of the parts of my body that she found disgusting.  I silently backed away, full
of pain, knowing it would be hard to foget this moment, and feeling betrayed.
At the same time, I knew it was somehow deserved.  Later on in the dream I ended
up having a threesome with Ocean and her friend.  I woke up and thought of the
dream I had in the fall about a beautiful relationship with Ocean, and felt a
sense of trepidation towards the coming semester.

Other things to write about in this chapter: IMG1772, Zoe from tinder, New York
trip

You guys need to give me more love, attention, and respect.

\chapter{Amazon}
The decision to lie.

\chapter{Marissa}
It was surprising that within a month of moving to New York, I was in a
relationship, sleeping with someone that really lit up my life.

Marissa was my first adult relationship, I think.  It was more than a little
surreal: us, on Christmas eve, dancing in front of our small Christmas tree,
staring into each other's eyes, only half drunk.  The amount of comfort and
pleasure we managed to enjoy during any given weekend never ceased to surprise
me.  We had perfectly achieved our ``goals'', and only dimly remembered our
dreams.

Marissa was a banker, but she made being a banker look easy.  She didn't really
buy into the whole ``insane work hours'' thing.  She was effective and composed,
and had a moderate, genuine interest in the financial services industry.  Also,
she was attractive.  Blonde.  She looked like a female banker. 

The only thing that hinted at her past was the fact that she wore too much
perfume.  I think I have a thing for women \section{} (I always make a point to call
the ``girls'' of our generation ``women'') \section{} I think I have a thing for women
who wear too much perfume.

New York social life.  I was sick of the fact that all of my friends besides
William didn't seem to care for me much at all, or were very flaky.  But I also
realized the paradox in the fact that I still wanted to be popular and have many
interesting impressive friends.  I felt that this postgrad time in New York was
my last chance to realize this ideal.  I had failed in high school and college.
In certain moments I saw I was just as calculating and cruel as the
acquaintances that disappointed me.  I told myself that I wanted a social life
that was brighter and more whole, but I also realized that this might just be a
cowardly excuse to avoid the social realities that came with my appearance,
income, lack of relevance or fame, and personality.

a world doomed by the power of sexual attraction, complexity, time, entropy,
death, luck

\chapter{End of Rowing}
Losing to Columbia.
Conversation re: loss to Japanese team at Henley.

\chapter{Showcase}


\chapter{The End}
Alumni regatta

I made my way out of Schoelkopf Stadium, aware of the presence of rowers and
other people I knew.  Parents and family came down from the bleachers and mixed
with the students.  Some parents reunited with their ``first generation''
college students, now graduates.  I felt tired as some sentimentality welled up
in me, seeing these sweaty parents and relatives with their homemade signs and
food.  The people I came into the stadium with were all around me, but we stared
forward silently now.  I walked directly towards the place where I had agreed to
meet my family.  I saw a few people that I knew, some people that I had wanted
to get closer to in the past, but I consciously realized that it was over, and
that I didn't have to talk to them any more, or even acknowledge their existence.

My family greeted my happily, and I didn't mind taking a few more pictures
before I returned my robe and hat.  We spent about an hour getting hot dogs and
talking.  It wasn't very crowded in the Cornell store. 

``Come back to Donlon with us, we have some presents for you,'' said my mom. 

``Okay,'' I said.  

We made the long walk back to the dorms.  I was in a group with Pop and father.
It was hot, and I felt a little dehydrated.  My grandfather said something about
how he remembered work seeming like a vacation after college, and that I could
start making real money now.

``I'm not so sure I'll be focusing on that, right after graduation,'' I said,
and immediately regretted it, thinking about how when he was my age, he had a
four year old daughter.  He was silent.  My dad and I talked about my plans for
the next couple months.  I again had the feeling that something small and
inarticulate in my grandfather despised me, or just thought I was annoying.  I
pushed the thought to the side, thoughts like that weren't any good for a 23
year old.

Back at the dorm I sat around while my sisters changed, and everyone reconvened.
My sisters offered me some gummy worms.  I felt relaxed.

I opened presents from my aunt and great aunt first.  Both had given me \$100
checks.  I was surprised and happy.  My parents had given me new shoes.  They
would later send me an e-card for \$1000, that said they were proud of my hard
work.  Finally I opened the present from my grandparents.  The card on the
outside contained a check for \$1000.  Looking at the card and smiling, I
realized that I wouldn't have to worry about finding the money to pay my deposit
and first month's rent.  ``Thank you so much,'' I said, and started opening the
actual present.  It was a cheap blue plastic toolbox.  ``For your apartment,''
said Pop.  ''Open it.''

I opened up the toolbox.  It reminded me of a slightly better version of
toolboxes I had recieved as a child, with a flimsy saw and light hammer.  A
\$100 bill was taped to the inside of the case.

``The best tool of all!'' said Pop loudly, reading the handwritten note attached
to the billfold.  I laughed.


\section{}

\chapter{Justification} 

People have never seemed particularly 'into' my artistic or literary work, but I
want to believe that if I stay focused and honest that the writing will be
beautiful and useful.  If not, I will begin my working life knowing that I tried
writing, and that writing is a way for me to communicate, but is not a viable
career option.

This writing is kind of impossible.  I feel very conscious that the time I have
now for writing, these bus rides and train rides across the state of New York,
are my chance to write.  If I let even one slip it is a problem.  Some walls you
can't get through, right?  Is there any point to getting through this wall?  The
only writing that seems to come naturally is this sort of journaling  it's what
I've practiced the last five years.  But it is useless when it comes to ``real
writing''.  The truth is that I have little understanding of what experiences in
my life were important and literary, and I also don't have an understanding of
how to present those experiences in a way that would be interesting to others.
There is noise and distraction and life will slip by as it always has in the
past.  Any other approach seems to lead to insanity and suffering.  Insanity and
suffering to no end, due to lack of talent, willpower, relevance, etc.  Okay,
done.  The only hope is that I can be like the shitty rowers who just keep
trying and trying and a way that you'd never expect them to succeed but they
kind of have their plan, and eventually it works.  Is that me?  I don't know.
But it kind of \textit{needs} to be me.
The fantasy that you will just work really hard and then it'll be perfect will
not happen.  Your goal should be to get this to a state where it can be
critiqued.

Are you really going to do this?  It's like sitting there before a 2k.  Remember
your true last 2k, the one you did after you got kicked out of the boat, with
L.  Going out at a pace you knew you wouldn't hold.  It would be too
painful.  At 500?  600 to go, knowing it would be too painful.  "It's good you
stopped, and didn't have a shitty piece," said L.  And remember your second last
2k, after you got sick during spring break.  That one was a thing of beauty in a
way.  

Should this be limited to the scope of the thoughts you actually had in college?

why are you telling me this?  why is this interesting to me?
bluntly stating is not always so interesting really
the only way you have of making your writing ``better'' rn is deleting parts of
it

now feeling that I might be writing the about the wrong things.
``I feel like I am respected in the majority of situations''
social context things vs. ``true self'', insofar as true self is internal, individual
internet as the outside/inside place, literary place

I just have this feeling in my body I remember

Telling yourself that the only real thing is your work will make you depressed
and undermine your work.  

I've got to have a ``game plan'' for getting this done.

When you start to fall in love
and your stomach turns to syrup
Why am I so basic and boring?

``I yearn to be fucked,''
jealousy

